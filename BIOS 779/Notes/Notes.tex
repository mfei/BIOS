
\section{Prior and Posterior Expectation and Variance}
 
\begin{align*}
	E(\theta) &= E \Big[ E[\theta | Y] \Big]
\end{align*}
Prior mean of $\theta$ = Average posterior mean of  $\theta$ over data distribution.

\begin{align*}
	Var(\theta) &= E \Big[ Var(\theta | Y) \Big] + Var \Big[ E[\theta | Y] \Big]
\end{align*}
Posterior variance of $\theta$ is, on average, less than prior variance of $\theta$.


\section{Prior Distribution}
When we do not know the precision, we can use the normal-gamma conjugate prior for $\theta, P$


\section{Bayesian in linear regression}

\begin{itemize}
\item[(i)] If we start with the non-estimable $X'X$ is not full rank, $X'X$ inverse does not exist. But if we have a proper prior, we could still estimate.

\item[(ii)] sensitivity analysis based on different variances, assume the $a_0$ to incorporate the variance, there is no just one the prior

\item[(iii)] Use discount likelihood as the prior specification, the power prior. Need to understand how we go from discount likelihood to discount likelihood times initial prior.

Bayesian Update
It is a hierarchy, prior today is the posterior of yesterday.

Stage 0: haven't observed historical data. $\pi_0(\beta, \gamma)$ - non-informative
Stage 1: $D_0 = (n_0, Y_0, X_0), a_0= 1$,  observed historical data, natural bayesian update. Then it is posterior of $\beta$ based on on $D_0$ with $\pi_0(\beta, \gamma)$ as initial prior.
Stage 2: $D= (n, Y, X), a_0 =1$, posterior of $\beta$ uses the posterior in stage 1 as the prior for $\beta$

Question:
When $\alpha=0$, the HPD included $H_0$, and as $\alpha$ increase, the HPD does not include $H_0$. Then how do we interpret it?


\end{itemize}

\section{AIC BIC L measure}
1. Y needs to be in the same scale, not only the X is the same.
2. Variable selection number - $2^p$


