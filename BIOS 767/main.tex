\documentclass[a4paper,11pt]{book}
\usepackage{import}
\usepackage{amsmath}
\usepackage{mathtools}
\usepackage[utf8]{inputenc} % set input encoding (not needed with XeLaTeX)
\usepackage[thinc]{esdiff}
%\usepackage{example}

%%% Examples of Article customizations
% These packages are optional, depending whether you want the features they provide.
% See the LaTeX Companion or other references for full information.

%%% PAGE DIMENSIONS
\usepackage{geometry} % to change the page dimensions
\geometry{a4paper} % or letterpaper (US) or a5paper or....
% \geometry{margin=2in} % for example, change the margins to 2 inches all round
% \geometry{landscape} % set up the page for landscape
%   read geometry.pdf for detailed page layout information

\usepackage{graphicx} % support the \includegraphics command and options

% \usepackage[parfill]{parskip} % Activate to begin paragraphs with an empty line rather than an indent

%%% PACKAGES
\usepackage{booktabs} % for much better looking tables
\usepackage{array} % for better arrays (eg matrices) in maths
\usepackage{paralist} % very flexible & customisable lists (eg. enumerate/itemize, etc.)
\usepackage{verbatim} % adds environment for commenting out blocks of text & for better verbatim
\usepackage{subfig} % make it possible to include more than one captioned figure/table in a single float
% These packages are all incorporated in the memoir class to one degree or another...

%%% HEADERS & FOOTERS
\usepackage{fancyhdr} % This should be set AFTER setting up the page geometry
\pagestyle{fancy} % options: empty , plain , fancy
\renewcommand{\headrulewidth}{0pt} % customise the layout...
\lhead{}\chead{}\rhead{}
\lfoot{}\cfoot{\thepage}\rfoot{}

%%% SECTION TITLE APPEARANCE
\usepackage{sectsty}
\allsectionsfont{\sffamily\mdseries\upshape} % (See the fntguide.pdf for font help)
% (This matches ConTeXt defaults)

%%% ToC (table of contents) APPEARANCE
\usepackage[nottoc,notlof,notlot]{tocbibind} % Put the bibliography in the ToC
\usepackage[titles,subfigure]{tocloft} % Alter the style of the Table of Contents
\renewcommand{\cftsecfont}{\rmfamily\mdseries\upshape}
\renewcommand{\cftsecpagefont}{\rmfamily\mdseries\upshape} % No bold!
\usepackage[english]{babel}
\usepackage{amsthm}
\theoremstyle{definition}
\newtheorem{definition}{Definition}[section]
\newtheorem{theorem}{Theorem}[section]
\newtheorem{corollary}{Corollary}[theorem]
\newtheorem{lemma}[theorem]{Lemma}
\theoremstyle{remark}
\newtheorem*{remark}{Remark}


\usepackage{makeidx}
\makeindex

\begin{document}

\frontmatter
\import{./}{title.tex}
\clearpage
%\thispagestyle{empty}

\tableofcontents

\mainmatter

\chapter{ANOVA}

\import{sections/}{section1-1-ANOVA.tex}
\clearpage
\import{sections/}{section1-2-Repeated ANOVA.tex}
\clearpage
%\import{sections/}{section1-3.tex}
\import{sections/}{section1-4-Linear.tex}
\clearpage
\import{sections/}{section1-5-LRT.tex}
\clearpage
\import{sections/}{section1-6-F.tex}

\chapter{Parameter Estimates}

\section{The Standard Exponential Distribution}

The standard exponential distribution family 

\begin{align*}
p(y| \theta) &= \phi \Big[ \exp \Big( y \theta - b(\theta) \Big) - c(y) \Big] - \frac{1}{2} s(y, \phi)
\end{align*}

We will explore the fun characteristics of the exponential family

\begin{itemize}
\item[(i)] Mean and Variance by derivatives

\begin{align*}
log  \int p(y| \theta) &=log  \int \phi \Big[ \exp \Big( y \theta - b(\theta) \Big) - c(y) \Big] - \frac{1}{2} s(y, \phi) dv = 0 \\
 log \int \exp \{( y \theta ) \} h(y) v(dy) &= b(\theta) \\
 \partial_{\theta} log \int \exp \{( y \theta ) \} h(y) v(dy) &= \partial_{\theta}  b(\theta) \\
\end{align*}

To proceed we need to move the gradient past the integral sign. In general derivatives can not be moved past integral signs (both are certain kinds of limits, and sequences of limits can differ depending on the order in which the limits are taken). However it turns out that the move is justified in this case by an appeal to the dominated convergence theorem. 

\begin{align*}
\partial_{\theta}  b(\theta) &= \partial_{\theta}  log \int \exp \{( y \theta ) \} h(y) v(dy)\\
 &=  \frac{\int y \exp \{( y \theta ) \} h(y) v(dy) }{\int \exp \{( y \theta ) \} h(y) v(dy)} \\
 &= \int y \exp \{ y \theta - b(\theta) \} h(x) v(dy) \\
 &= E[y] 
\end{align*}

Also we can see that the first derivative of $b(\theta)$ is equal to the mean of the sufficient statistics. Similar for the variance.

Another proof is to use the Bartlett's identities

Suppose that differentiation and integration are exchangeable and all the necessary expectations are finite. We have the following results:

\begin{align*}
E\_{\xi} \Big( \partial_j l_n \Big) &= 0,\\
E_{\xi} \Big( \partial^2_{j,k} l_n \Big) + E_{\xi} \Big( \partial_j l_n \partial_k l_n \Big) = 0 \\
\end{align*}

By the above two equations, we can get the expectation and variance. 


\end{itemize}



\section{The Bernoulli Distribution}

The standard exponential distribution family 

\begin{align*}
p(y| \theta) &= \phi \Big[ \exp \Big( y \theta - b(\theta) \Big) - c(y) \Big] - \frac{1}{2} s(y, \phi)
\end{align*}

For Bernoulli distribution,
\begin{align*}
p(x| \pi) &= \pi^{x} (1- \pi)^{1-x} \\
&= \exp \{ \log \Big( \frac{\pi}{1- \pi} \Big) x + \log (1 - \pi) \}
\end{align*}

We see that Bernoulli distribution is an exponential family distribution with 

\begin{align*}
\theta &= \log \Big( \frac{\pi}{1- \pi} \Big) \\
b(\theta)&=- \log (1 - \pi) =  \log \Big( 1 + \exp(\theta) \Big) x \\
\phi & = 1
\end{align*}

\subsection{Mean and Variance}

For a univariate random variable $Y$, in this case, all the $Y_i$ have the same $\pi$
\begin{align*}
\diffp{b(\theta)}{\theta} &= \frac{\exp(\theta)}{1 + \exp(\theta) } = \frac{1}{1 + \exp(-\theta)} = \mu = E(Y) \\
\diffp{b(\theta)}{\theta \theta}  &= \frac{\exp(\theta)}{\Big[ 1 + \exp(\theta) \Big]^2} = \mu(1-\mu) =Var(Y)
\end{align*}

In regression model, $logit (\pi) = X \beta$, which $\beta$ is a vector, then we will use the chain rule. And each individual $y_i$ has its own equation that $\pi_i$ is different.

\begin{align*}
\theta & = X \beta, \qquad \theta_i = x_i^{T} \beta \\
\partial_{\beta}{b(\theta_i)} &= \partial_{\theta_i}{b(\theta_i)} \partial_{\beta}{{\theta_i}} \\
&= \frac{\exp(\theta_i)}{1 + \exp(\theta_i) }  x_i= \frac{1}{1 + \exp(-\theta)} x_i= \mu_i x_i\\
\partial^2_{\beta}{{b(\theta_i)}} &= \frac{\exp(\theta_i)}{\Big[ 1 + \exp(\theta_i) \Big]^2} x_i^{\otimes 2}= \mu_i(1-\mu_i) x_i^{\otimes 2}
\end{align*}

And we will need to connect this with the Fisher Information or Newton-Raphson algorithm

\begin{align*}
\theta_i & = k \Big(x_i^{T} \beta \Big) = x_i^{T} \beta \\
\xi &= (\beta, \phi)\\
ln(\xi) &= \sum_{i=1}^n \phi \Big[ y_i k \Big(x_i^{T} \beta \Big) - b \Big( k \Big(x_i^{T} \beta \Big)  \Big) - c(y_i) \Big] - \frac{1}{2} s(y_i, \phi) \\
\dot{ln}(\beta) &= \diffp{ln(\beta) }{\beta} = \phi \sum_{i=1}^n \Big[ y_i - \dot{b} \Big( k \Big(x_i^{T} \beta \Big)  \Big)  \Big] \dot{k} \Big(x_i^{T} \beta \Big) x_i \\
&= \sum_{i=1}^n \Big[ y_i - \mu_i \Big] x_i \\
\ddot{ln}(\beta) &= \diffp{ln(\beta) }{\beta \beta} = -\phi \sum_{i=1}^n \ddot{b} \Big( k(x_i^T \beta) \Big) \dot{k}(x_i^T \beta)^2 x_i x_i^T + \phi \sum_{i=1}^n \Big[y_i - \dot{b}(k(x_i^T \beta)) \Big] \ddot{k}(x_i^T \beta) x_i x_i^T \\
&= -\sum_{i=1}^n \ddot{b} \Big(\theta_i \Big) x_i x_i^T = -\sum_{i=1}^n V(\theta_i) x_i x_i^T, \qquad \partial^2_{\beta}{{b(\theta_i)}} = V(\theta_i)
\end{align*}

let 
\begin{align*}
V(\theta) & = diag \{ V(\theta_i) \} , \qquad e_i = y_i - \mu_i\\
\sum_{i=1}^n V(\theta_i) x_i x_i^T &= X V(\theta) V^T\\
\mu_i &= \dot{b}(\theta_i), \qquad v_i = \ddot{b}(\theta_i)\\
\dot{\theta}_i &= \partial_{\beta} \theta_i = \dot{k}(x_i^T \beta) x_i, \qquad \ddot{\theta}_i = \partial^2_{\beta} \theta_i = \ddot{k}(x_i^T \beta) x_i x_i^T \\
\dot{b}(\theta_i) &= \partial_{\theta} b(\theta) \Big |_{\theta = \theta_i}, \dot{k}(\eta) = \partial_{\eta} k(\eta), \ddot{k}(\eta) = \partial^2_{\eta}(\eta)
\end{align*}

So
\begin{align*}
E \Big[ - \ddot{l}n(\beta) \Big] & = \phi \sum_{i=1}^n v_i \dot{\theta}_i^{\otimes 2}
\end{align*}

Another set is to use $E(y_i), Var(y_i)$ which is also used commonly as that are the information we generally get. It is used a lot in GEE. 
\begin{align*}
\partial_{\mu} \theta &= \partial_{\theta} \mu ^{-1}, \qquad \partial_{\mu} \mu = \partial_{\theta} \mu \partial_{\mu} \theta = 1\\
\partial_{\theta} \mu &= \partial_{\theta} b(\theta) = \ddot{b}(\theta) \\
\partial_{\mu} \theta &= \Big( \partial_{\theta} \mu \Big)^{-1} =  \ddot{b}(\theta)^{-1} \\
\end{align*}

Then we have the connection between the two system
\begin{align*}
\partial_{\beta} \theta &= \partial_{\beta} \mu_i \partial_{\mu_i} \theta_i = \partial_{\beta} \mu_i \Big[ \ddot{b}(\theta_i) \Big]^{-1} \\
\partial_{\beta}^2 \theta_i &= \Big( \partial^2_{\mu_i} \theta_i \Big) \Big( \partial_{\beta} \mu_i \Big)^{\otimes 2} + \partial_{\mu_i} \theta_i \Big( \partial_{\beta}^2 \mu_i \Big) \\
&= - \dddot{b}(\theta_i) \ddot{b}(\theta_i)^{-3} \Big( \partial_{\beta} \mu_i \Big)^{\otimes 2} + \Big[ \ddot{b}(\theta_i) \Big]^{-1} \Big( \partial^2_{\beta} \mu_i \Big)
\end{align*}

The generalized estimation model
\begin{align*}
V(\beta) &= \text{diag} \Big( v_1(\beta), …, v_n(\beta) \Big) \\
e(\beta) &= (y_1 - \mu_1(\beta), …, y_n- \mu_n(\beta))^{'} \\
D_{\theta} (\beta)^{'} &= \Big( \partial_{\beta} \beta_1(\beta),…,  \partial_{\beta} \beta_n(\beta)\Big)_{p \times n} \\
D (\beta)^{T} &= \Big( \partial_{\beta} \mu_1(\beta),…,  \partial_{\beta} \mu_n(\beta) \Big)_{p \times n} \\
\dot{l}_n(\beta) &= \phi D_{\theta}(\beta)^{T} e(\beta) = \phi D(\beta)^{'} V(\beta)^{-1} e(\beta) \\
E \Big[ -\ddot{l}_n(\beta) \Big] &= \phi D_{\theta}(\beta)^{'} V D_{\theta}(\beta) = \phi D(\beta)^{'} V(\beta)^{-1} D(\beta) 
\end{align*}



\chapter{Convergence Theorem}

\section{Conditional MLE}
The conditional maximum likelihood estiamte (CMLE) of $\psi$ is not calculated directly from the conditional distribution of $\psi$. While we get it from $p(n_{11}| n_{1+}, n_{+1}, n, \psi)$.

We should be able to get the MLE from the log-likelihood conditional.

$\hat{\psi}_c$ is the solution to 
\begin{align*}
	n_{11} &= P_1(\hat{\psi}_c)/P_0(\hat{\psi}_c) = \mu\\
	log P(n_{11}| n_{1+}, n_{+1}, n, \psi) &=  n_{11} log \psi - log P_0(\psi) + c\\
	\diffp{log P}{\psi} &= \frac{n_{11}}{\psi} -\frac{P_0(\psi)'}{P_0(\psi)} = 0\\
	n_{11} &= P_1(\hat{\psi}_c)/P_0(\hat{\psi}_c)
\end{align*}
The variance of $\hat{\psi}_c$ can be approximated by the inverse of the Fisher information matrix $I_n(\hat{\psi}_c)$, which is given
\begin{align*}
	I_n(\hat{\psi}_c) &= E\{[ \partial_{\psi} log P(n_{11}| n_{1+}, n_{+1}, n, \hat{\psi}_c)]^2 \} = \frac{Var(n_{11}| n_{1+}, n_{+1}, n, \hat{\psi}_c)}{\hat{\psi}_c^2}
\end{align*}


\item[(g)] Suppose that $\pi_{11}, \pi_{12}$ are parameters of interest and the rest of the parameters are treated as nuisance. Derive the conditional likelihood of $(\pi_{11}, \pi_{12})$ and the conditional MLE's of  $(\pi_{11}, \pi_{12})$.
If not specified, we treat as general contingency table that total n is fixed. If only $\pi_{11}, \pi_{12}$ are parameters of interest and the rest of the parameters are treated as nuisance, then we will set the rest of the parameters as one parameter, and get its distribution, which is to find the sufficient statistics for rest of the parameters.
Write the Multinomial distribution in exponential family distribution.\\
We can find marginal distribution by summing over along all possible values of $(n_{11}, n_{12})$. Note that $n_{11} \leq \min{n_{1+} - n_{12}, n_{+1}}$ for a given value of $n_{12}$. Similarly, $n_{12} \leq \min{n_{1+}- n_{11}, n_{+1}}$ for a given value of $n_{11}$. \\
Additionally,
\begin{align*}
	n & \geq n_{1+} + n_{+1} + n_{+2} - n_{11} - n_{12} \\
	n_{11} + n_{12} & \geq \max{ 0, n_{+1} + n_{1+} + n_{+2}}
\end{align*}
Let
\begin{align*}
	S(n_{11}, n_{12}) &= \{(n_{11}, n_{12}): n_{11} + n_{12} \geq \max{ 0, n_{+1} + n_{1+} + n_{+2}},\\
	&  n_{11} \leq \min{(n_{1+} - n_{12}, n_{+1})}, n_{12} \leq \min{(n_{1+}- n_{11}, n_{+1})}   \} 
\end{align*}

The conditional distribution
\begin{align*}
	p(n_{11}, n_{12}|n_{13}, ...n_{IJ}, n) &= \frac{p(n_{ij}}{p(S_n)}\\
	&= \frac{\frac{1}{n_{11}! n_{12}! } \pi_{11}^{n_{11}} \pi_{12}^{n_{12}}}{\sum_{(x, y \in S_n)} \frac{1}{x! y!} \pi_{11}^x \pi_{12}^y}
\end{align*}
And $\hat{\pi}_{11}, \hat{\pi}_{12}$ are the CMLE that maximize $p(n_{11}, n_{12}|n_{13}, ...n_{IJ}, n)$.

\end{itemize}



\backmatter

\import{./}{bibliography.tex}


\end{document}
