\section{linear model} 

Suppose that Y is a $4 \times 1$ vector with $E(Y ) = \mu, \mu \in C(E)$, where E is the set $E = \{ \mu: \mu' = (\beta_1 + \beta_2 - \beta_3, \beta_2 + \beta_3, -\beta_2 - \beta_3, -\beta_1-\beta_2 + \beta_3) \} $ where the $\beta_i$ are real numbers, $i = 1, 2, 3$. Further assume that $Cov(Y) = \sigma^2I_{4 \times 4}$, where $\sigma^2$ is unknown.

\begin{itemize}
\item[(a)] Derive $\hat{\mu}$, the ordinary least squares estimate of $\mu$, by carrying out the appropriate projection.
    
E(Y) is in the column space of C(E), we need to find the o.p.o on C(E). Also $Cov(Y) = \sigma^2 I_{4 \times 4}$, we can use ordinary least squares estimator as i.i.d.
\begin{align*}
    \mu' &= (\beta_1 + \beta_2 - \beta_3, \beta_2 + \beta_3, -\beta_2 - \beta_3, -\beta_1-\beta_2 + \beta_3) = X \beta\\
    \beta &= (\beta_1, \beta_2, \beta_3)^T\\
    X &= \begin{bmatrix}
           1  & 1 & -1  \\
           0 &  1 & 1\\
           0  & -1 & -1  \\
           -1 &  -1 & 1\\
         \end{bmatrix}\\
    C(X) &= X_1 = \begin{bmatrix}
           1  & 1   \\
           0 &  1 \\
           0  & -1  \\
           -1 &  -1 \\
         \end{bmatrix}\\
    M_{\mu} &= X_1(X_1'X_{1})^{-1} X_{1}^T = \begin{bmatrix}
           1  & 1   \\
           0 &  1 \\
           0  & -1  \\
           -1 &  -1 \\
         \end{bmatrix}\begin{bmatrix}
           1  & -1/2   \\
           -1/2 &  1/2 \\
         \end{bmatrix}\begin{bmatrix}
           1 & 0 & 0 & -1    \\
           1 & 1 & -1 & -1
         \end{bmatrix}\\
       \hat{\mu} &= M_{\mu} Y =  1/2 \begin{bmatrix}
           1  & 0 & 0 & -1   \\
           0 &  1 & -1 & 0 \\
           0  & -1 & 1 & 0 \\
           -1 &  0 & 0 & 1 \\
         \end{bmatrix} (y_1, y_2, y_3, y_4)^T = 1/2 (y_1- y_4, y_2-y_3, y_3-y_2, y_4-y_1)^T
\end{align*}

\item[(b)] Find the BLUE of $\beta_2 - \beta_3$ or show that it is nonestimable.
\begin{align*}
    \lambda &= (0, 1,-1)^T, \qquad \lambda^T \beta = \beta_2 - \beta_3\\
    \lambda^T &= \rho^T X = (\rho_1, \rho_2, \rho_3, \rho_4)  \begin{bmatrix}
           1  & 1 & -1  \\
           0 &  1 & 1\\
           0  & -1 & -1  \\
           -1 &  -1 & 1\\
         \end{bmatrix} = (0, 1,-1)\\
         \rho_1 &= \rho_4, \qquad \rho_2 - \rho_3 = 1, \qquad \rho_2 - \rho_3 = -1
\end{align*}
The contradict of $\rho_2 - \rho_3$ indicate that  $\beta_2 - \beta_3$ is not estimable.
\item[(c)] Consider testing $H_0: \beta_2 + \beta_3 = 0$  versus $H_1: \beta_2 + \beta_3 \neq 0$ . Let $E_0$ denote the set E assuming that $H_0$ is true. Explicitly give the sets $E_0$ and $E \cap E_0^{\perp}$.\\
Find the $M_{0}, M_{1}$ are the o.p.o onto C(X) for $H_0, H_1$.$\textbf{Not on C(Y)}$.Then the $C(M_{0}), C(M_{1})$ and the sets $E_0$ and $E \cap E_0^{\perp}$ relationship needs attention.
\begin{align*}
    \lambda &= (0, 1, 1)^T, \qquad \lambda^T \beta = \beta_2 + \beta_3\\
    \lambda^T &= \rho^T X = (\rho_1, \rho_2, \rho_3, \rho_4)  \begin{bmatrix}
           1  & 1 & -1  \\
           0 &  1 & 1\\
           0  & -1 & -1  \\
           -1 &  -1 & 1\\
         \end{bmatrix} = (0, 1, 1)\\
         \rho &= (1, 2, 1, 1)^T
\end{align*}
$ \beta_2 + \beta_3$ is estimable with one $\rho = (1, 2, 1, 1)^T$. Then we can have $H_0: \rho^T M Y = 0$ that $\rho^T M \perp C(E_0)$. 
\begin{align*}
    M_1 &= ( M\rho) [(M\rho)^T ( M\rho)]^{-1} ( M\rho)^T\\
    M\rho &=  \rho_N= (0, 1, -1, 0)^T \\
    M_1 &= \rho_N [(\rho_N)^T (\rho_N)]^{-1} (\rho_N)^T =1/2 \begin{bmatrix}
           0  & 0 & 0 & 0  \\
           0 &  1 & -1 & 0\\
           0  & -1 & 1 & 0 \\
           0 & 0 & 0 & 0\\
         \end{bmatrix} \\
         & = 1/2 (0, 1, -1, 0)^T
\end{align*}
And the complement of $M_0 = M- M_1$ 
\begin{align*}
    M_0 & = M- M_1 = 1/2 \begin{bmatrix}
           1  & 0 & 0 & -1  \\
           0 & 0 & 0 & 0\\
           0 & 0 & 0 & 0\\
           -1  & 0 & 0 & 1  \\
         \end{bmatrix} \\
         & = 1/2 (1, 0,  0, -1)^T
\end{align*}
Also we can just look at C(Y) column space
\begin{align*}
    E_0 & = span \{(\beta_1+2\beta_2, 0, 0, -\beta_1-2\beta_2)^T \}= span \{(1, 0,  0, -1)^T \}
\end{align*}


NOTE: THE CONTRAST TEST WE GENERALLY USE 

\begin{align*}
    F-test & = \frac{(\Lambda' \beta -d)^T \Big (\Lambda' (X'X) \Lambda \Big)^{-1}  (\Lambda' \beta -d) \Big / r(\Lambda)}{\sigma^2}
\end{align*}
In this way, we don't need to calculate the $\rho, P'$ matrix.

\item[(d)] Assuming normality for Y , construct the simplest possible expression for
the F statistic for the hypothesis $H_0: \mu \in E_0$ versus $H_1: \mu \not\in E_0$, where $E_0$ is
specified in part (c), and give the distribution of the F statistic under the null and
alternative hypotheses.\\
\begin{align*}
    M\rho &=  \rho_N= (0, 1, -1, 0)^T \in M, \qquad r(\rho_N) = 1\\
    M_{\rho} &= \rho_N [(\rho_N)^T (\rho_N)]^{-1} (\rho_N)^T=1/2 \begin{bmatrix}
           0  & 0 & 0 & 0  \\
           0 &  1 & -1 & 0\\
           0  & -1 & 1 & 0 \\
           0 & 0 & 0 & 0\\
         \end{bmatrix}\\
    MSE &= \lVert (I-M)Y \rVert = 1/2 (y_1+y_4)^2 + 1/2 (y_2 + y_4)^2 \\
    F &= \frac{Y^T M_{\rho} Y/r(\rho)}{MSE} = \frac{2(y_2 - y_3)^2}{(y_1+y_4)^2 + (y_2 + y_4)^2} \sim F(1,2, \gamma), \qquad r(M-M_{\rho}) = 1, r(I-M) = 2
\end{align*} 
In which, under $H_0, \gamma = 0$, and under $H_1$.
\begin{align*}
    \gamma &= \frac{\lVert (M_1) X\beta \rVert}{2 \lVert (I-M)Y \rVert/2}\\
    &= \frac{(\beta_2 + \beta_3)^2}{\sigma^2}
\end{align*} 

\item[(e)] Assuming normality for Y , construct an exact 95$\%$ confidence interval for $\beta_2 + \beta_3$.
From part(d), we have
\begin{align*}
 \lambda' = (0, 1, 1)\\
 \rho &= (1, 1, 0, 1)^T\\
 \lambda^T \beta &=  \rho M Y = M_1 Y= 1/2 (y_2 - y_3)\\
  F &= \frac{\lVert \lambda^T \beta \rVert/r(\rho)}{\sigma^2} = \frac{\lambda'\beta [\lambda'[X'X]^{-1} \lambda]^{-1} (\lambda'\beta)^T }{\sigma^2} \sim F(1,2, \gamma)\\
  [\lambda'[X'X]^{-1} \lambda]^{-1} & = 2\\
 &\{\beta: \frac{\lambda'\beta [\lambda'[X'X]^{-1} \lambda]^{-1} (\lambda'\beta)^T }{\sigma^2} \leq F(0.95, 1,2) \} 
\end{align*} 

\end{itemize}