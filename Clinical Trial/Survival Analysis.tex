% !TEX TS-program = pdflatex
% !TEX encoding = UTF-8 Unicode

% This is a simple template for a LaTeX document using the "article" class.
% See "book", "report", "letter" for other types of document.

\documentclass[11pt]{article} % use larger type; default would be 10pt

\usepackage[utf8]{inputenc} % set input encoding (not needed with XeLaTeX)

%%% Examples of Article customizations
% These packages are optional, depending whether you want the features they provide.
% See the LaTeX Companion or other references for full information.

%%% PAGE DIMENSIONS
\usepackage{geometry} % to change the page dimensions
\geometry{a4paper} % or letterpaper (US) or a5paper or....
% \geometry{margin=2in} % for example, change the margins to 2 inches all round
% \geometry{landscape} % set up the page for landscape
%   read geometry.pdf for detailed page layout information

\usepackage{graphicx} % support the \includegraphics command and options

% \usepackage[parfill]{parskip} % Activate to begin paragraphs with an empty line rather than an indent

%%% PACKAGES
\usepackage{booktabs} % for much better looking tables
\usepackage{array} % for better arrays (eg matrices) in maths
\usepackage{paralist} % very flexible & customisable lists (eg. enumerate/itemize, etc.)
\usepackage{verbatim} % adds environment for commenting out blocks of text & for better verbatim
\usepackage{subfig} % make it possible to include more than one captioned figure/table in a single float
\usepackage{amsmath}
\usepackage{mathtools}
\usepackage[thinc]{esdiff}
% These packages are all incorporated in the memoir class to one degree or another...

%%% HEADERS & FOOTERS
\usepackage{fancyhdr} % This should be set AFTER setting up the page geometry
\pagestyle{fancy} % options: empty , plain , fancy
\renewcommand{\headrulewidth}{0pt} % customise the layout...
\lhead{}\chead{}\rhead{}
\lfoot{}\cfoot{\thepage}\rfoot{}

%%% SECTION TITLE APPEARANCE
\usepackage{sectsty}
\allsectionsfont{\sffamily\mdseries\upshape} % (See the fntguide.pdf for font help)
% (This matches ConTeXt defaults)

%%% ToC (table of contents) APPEARANCE
\usepackage[nottoc,notlof,notlot]{tocbibind} % Put the bibliography in the ToC
\usepackage[titles,subfigure]{tocloft} % Alter the style of the Table of Contents
\renewcommand{\cftsecfont}{\rmfamily\mdseries\upshape}
\renewcommand{\cftsecpagefont}{\rmfamily\mdseries\upshape} % No bold!
\title{Survival Analysis}
\author{Mingwei Fei}

\begin{document}
	
	\maketitle
	
	\section{Sample Size}
	The $ln(HR)$ follows a normal distribution, we use this to calculate the sample size.
	
	\begin{align*}
	ln (\hat{\Delta}) & \sim N(ln(\Delta), \left( \frac{1}{d_1} + \frac{1}{d_2} \right)) \\
	\left( \frac{1}{d_1} + \frac{1}{d_2} )\right)^{-1} &= \left[ \frac{(z_{\alpha/2} + z_{\beta})^2}{(ln \Delta_0)^2} \right] 
	\end{align*}
	
	If hazard ratio set at 2.1, then 
	\begin{align*}
	\left( \frac{1}{d_1} + \frac{1}{d_2} )\right)^{-1} &= \left[ \frac{(1.96 + 0.58)^2}{(ln 2.1)^2} \right] = 11.7\\
	\frac{1}{d_1} + \frac{1}{d_2} = \frac{1}{11.7} = 0.085\\
	d_1 = d_2 = 23.44
	\end{align*}
	
	Assume the overall event and censored rate is $20\%$, then the sample size is $48/0.2 = 240$. If overall event rate (including censoring) is $18\%$, then the sample size is $48/0.18 = 266$.
	
	\subsection{Non-inferiority margin Hazard ratio $\Delta_0$ = 2.1}
	The assumption is that control group (C) event rate $10\%$ and treatment group (T) event rate $20\%$ at 6 months. Assume survival function is an exponential distribution:
	\begin{align*}
	S_t(t) &= exp(-\lambda_1 t), \qquad t= 0.5, S_t = 0.8 , -\lambda_1 = ln(0.8)/0.5\\
	S_c(t) &=  exp(-\lambda_2 t), \qquad t= 0.5, S_c = 0.9, -\lambda_2 = ln(0.9)/0.5  \\
	\Delta_0 &= \frac{\lambda_1}{\lambda_2}= \frac{ln(0.8)}{ln(0.9)} = 2.117
	\end{align*}	

	\subsection{Hazard ratio actual = 0.55}
The control group survival $76.8\%$ and treatment group survival $86.2\%$ at 6 months. 
Assume survival function is an exponential distribution:
\begin{align*}
	S_t(t) &= exp(-\lambda_1 t), \qquad t= 0.5, S_t = 0.862 , -\lambda_1 = ln(0.862)/0.5\\
	S_c(t) &=  exp(-\lambda_2 t), \qquad t= 0.5, S_c = 0.768, -\lambda_2 = ln(0.768)/0.5  \\
	HR = \frac{\lambda_1}{\lambda_2}\\
	&= \frac{ln(0.862)}{ln(0.768)} = 0.56
\end{align*}




\end{document}