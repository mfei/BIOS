% !TEX TS-program = pdflatex
% !TEX encoding = UTF-8 Unicode

% This is a simple template for a LaTeX document using the "article" class.
% See "book", "report", "letter" for other types of document.

\documentclass[11pt]{article} % use larger type; default would be 10pt

\usepackage[utf8]{inputenc} % set input encoding (not needed with XeLaTeX)

%%% Examples of Article customizations
% These packages are optional, depending whether you want the features they provide.
% See the LaTeX Companion or other references for full information.

%%% PAGE DIMENSIONS
\usepackage{geometry} % to change the page dimensions
\geometry{a4paper} % or letterpaper (US) or a5paper or....
% \geometry{margin=2in} % for example, change the margins to 2 inches all round
% \geometry{landscape} % set up the page for landscape
%   read geometry.pdf for detailed page layout information

\usepackage{graphicx} % support the \includegraphics command and options

% \usepackage[parfill]{parskip} % Activate to begin paragraphs with an empty line rather than an indent

%%% PACKAGES
\usepackage{booktabs} % for much better looking tables
\usepackage{array} % for better arrays (eg matrices) in maths
\usepackage{paralist} % very flexible & customisable lists (eg. enumerate/itemize, etc.)
\usepackage{verbatim} % adds environment for commenting out blocks of text & for better verbatim
\usepackage{subfig} % make it possible to include more than one captioned figure/table in a single float
\usepackage{amsmath}
\usepackage{mathtools}
\usepackage[thinc]{esdiff}
% These packages are all incorporated in the memoir class to one degree or another...

%%% HEADERS & FOOTERS
\usepackage{fancyhdr} % This should be set AFTER setting up the page geometry
\pagestyle{fancy} % options: empty , plain , fancy
\renewcommand{\headrulewidth}{0pt} % customise the layout...
\lhead{}\chead{}\rhead{}
\lfoot{}\cfoot{\thepage}\rfoot{}

%%% SECTION TITLE APPEARANCE
\usepackage{sectsty}
\allsectionsfont{\sffamily\mdseries\upshape} % (See the fntguide.pdf for font help)
% (This matches ConTeXt defaults)

%%% ToC (table of contents) APPEARANCE
\usepackage[nottoc,notlof,notlot]{tocbibind} % Put the bibliography in the ToC
\usepackage[titles,subfigure]{tocloft} % Alter the style of the Table of Contents
\renewcommand{\cftsecfont}{\rmfamily\mdseries\upshape}
\renewcommand{\cftsecpagefont}{\rmfamily\mdseries\upshape} % No bold!
\title{Survival Analysis}
\author{Mingwei Fei}

\begin{document}
	
	\maketitle
	
	\section{Sample Size}
	The $ln(HR)$ follows a normal distribution, we use this to calculate the sample size.
	
	\begin{align*}
	ln (\hat{\Delta}) & \sim N\left(ln(\Delta),  \frac{1}{d_1} + \frac{1}{d_2} \right) \\
	\left( \frac{1}{d_1} + \frac{1}{d_2} \right)^{-1} &= \left[ \frac{(z_{\alpha/2} + z_{\beta})^2}{(ln \Delta_0)^2} \right] 
	\end{align*}
	where $d_i$ is the number of observed events. 
	
	If hazard ratio set at 2.1, then 
	\begin{align*}
	\left( \frac{1}{d_1} + \frac{1}{d_2} )\right)^{-1} &= \left[ \frac{(1.96 + 0.842)^2}{(ln 2.1)^2} \right] = 14.26\\
	\frac{1}{d_1} + \frac{1}{d_2} &= \frac{1}{11.7} = 0.07, \qquad	d_1 = d_2 = 28.5
	\end{align*}
	
	The one-sided significance level 0.25, power is 0.8. Note that $Z_{\alpha/2}$ is the z score for the probability $1-\alpha/2$, and $z_{\beta}$ is the z score for the probability $1-\beta$. Assume the overall event and censored rate is $20\%$, then the sample size is $57/0.2 = 285$. The total number in the paper is 276.
	
	
	\subsection{Non-inferiority margin Hazard ratio $\Delta_0$ = 2.1}
	The assumption is that control group (C) event rate $10\%$ and treatment group (T) event rate $20\%$ at 6 months. Assume survival function is an exponential distribution:
	\begin{align*}
	S_t(t) &= exp(-\lambda_1 t), \qquad t= 0.5, S_t = 0.8 , -\lambda_1 = ln(0.8)/0.5\\
	S_c(t) &=  exp(-\lambda_2 t), \qquad t= 0.5, S_c = 0.9, -\lambda_2 = ln(0.9)/0.5  \\
	\Delta_0 &= \frac{\lambda_1}{\lambda_2}= \frac{ln(0.8)}{ln(0.9)} = 2.117
	\end{align*}	

	\subsection{Hazard ratio actual = 0.55}
The control group survival $76.8\%$ and treatment group survival $86.2\%$ at 6 months. 
Assume survival function is an exponential distribution:
\begin{align*}
	S_t(t) &= exp(-\lambda_1 t), \qquad t= 0.5, S_t = 0.862 , -\lambda_1 = ln(0.862)/0.5\\
	S_c(t) &=  exp(-\lambda_2 t), \qquad t= 0.5, S_c = 0.768, -\lambda_2 = ln(0.768)/0.5  \\
	HR = \frac{\lambda_1}{\lambda_2}\\
	&= \frac{ln(0.862)}{ln(0.768)} = 0.56
\end{align*}


	\section{Sample Size Formula}
	The test hypothesis is 
\begin{align*}
	H_0: & \lambda_1 = \lambda_2\\
	H_1: & \lambda_1 \neq \lambda_2
\end{align*}	
Or equivalently, in terms of hazard ratio, $\Delta = \lambda_1/\lambda_2$
\begin{align*}
	H_0: & \Delta = 1\\
	H_1: & \Delta \neq 1
\end{align*}
A much simpler and quite accurate approximation for a reasonably large number of events is based on the approximate normality of th natural logarithm of the estimated hazard ratio in each treatment group:
\begin{align*}
	ln(\hat{\lambda}_i) & \sim  N(ln \lambda_i, \frac{1}{d_i})
\end{align*}
where $d_i$ is the number of observed events. Thus, the $ln \Delta = ln \lambda_1 - ln \lambda_2$ also follows a normal distribution with variance $\frac{1}{d_1} + \frac{1}{d_2}$.

\begin{align*}
	ln (\hat{\Delta}) & \sim N\left(ln(\Delta),  \frac{1}{d_1} + \frac{1}{d_2} \right) \\
	\left( \frac{1}{d_1} + \frac{1}{d_2} \right)^{-1} &= \left[ \frac{(z_{\alpha/2} + z_{\beta})^2}{(ln \Delta_0)^2} \right] 
\end{align*}

The calculation of sample size follows
\begin{align*}
	Z &= \frac{ln(\hat{\Delta})}{\sigma}, \qquad \sigma = \sqrt{\frac{1}{d_1} + \frac{1}{d_2}}, \qquad \delta = ln(\Delta_0) \\
 &	P(Z \geq Z_{1-\alpha/2}| H_0)  \leq \alpha/2 \\
 &	P(Z \leq Z_{\beta}| H_1= \delta)  \geq \beta
\end{align*}
So we set Z satisfy the below equation
\begin{align*}
	\frac{ln(\hat{\Delta})}{\sigma} &= Z_{1-\alpha/2}, \qquad &\text{ H }_0\\
	\frac{ln(\hat{\Delta}) - \delta}{\sigma} &= Z_{\beta},  \qquad &\text{H}_1
\end{align*}
So we have
\begin{align*}
	ln(\hat{\Delta}) &= Z_{1-\alpha/2} \sigma, \qquad	ln(\hat{\Delta}) = Z_{\beta} {\sigma} + \delta ,\qquad	Z_{1-\alpha/2} \sigma = Z_{\beta} {\sigma} + \delta \\
	\sigma &= \frac{\delta}{Z_{1-\alpha/2} - Z_{\beta}}, \qquad 	\frac{1}{d_1} + \frac{1}{d_2} = \frac{\delta^2}{(Z_{1-\alpha/2} + Z_{1-\beta})^2}
\end{align*}
\end{document}