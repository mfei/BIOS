% !TEX TS-program = pdflatex
% !TEX encoding = UTF-8 Unicode

% This is a simple template for a LaTeX document using the "article" class.
% See "book", "report", "letter" for other types of document.

\documentclass[11pt]{article} % use larger type; default would be 10pt

\usepackage[utf8]{inputenc} % set input encoding (not needed with XeLaTeX)

%%% Examples of Article customizations
% These packages are optional, depending whether you want the features they provide.
% See the LaTeX Companion or other references for full information.

%%% PAGE DIMENSIONS
\usepackage{geometry} % to change the page dimensions
\geometry{a4paper} % or letterpaper (US) or a5paper or....
% \geometry{margin=2in} % for example, change the margins to 2 inches all round
% \geometry{landscape} % set up the page for landscape
%   read geometry.pdf for detailed page layout information

\usepackage{graphicx} % support the \includegraphics command and options

% \usepackage[parfill]{parskip} % Activate to begin paragraphs with an empty line rather than an indent

%%% PACKAGES
\usepackage{booktabs} % for much better looking tables
\usepackage{array} % for better arrays (eg matrices) in maths
\usepackage{paralist} % very flexible & customisable lists (eg. enumerate/itemize, etc.)
\usepackage{verbatim} % adds environment for commenting out blocks of text & for better verbatim
\usepackage{subfig} % make it possible to include more than one captioned figure/table in a single float
\usepackage{amsmath}
\usepackage{mathtools}
\usepackage[thinc]{esdiff}
% These packages are all incorporated in the memoir class to one degree or another...

%%% HEADERS & FOOTERS
\usepackage{fancyhdr} % This should be set AFTER setting up the page geometry
\pagestyle{fancy} % options: empty , plain , fancy
\renewcommand{\headrulewidth}{0pt} % customise the layout...
\lhead{}\chead{}\rhead{}
\lfoot{}\cfoot{\thepage}\rfoot{}

%%% SECTION TITLE APPEARANCE
\usepackage{sectsty}
\allsectionsfont{\sffamily\mdseries\upshape} % (See the fntguide.pdf for font help)
% (This matches ConTeXt defaults)

%%% ToC (table of contents) APPEARANCE
\usepackage[nottoc,notlof,notlot]{tocbibind} % Put the bibliography in the ToC
\usepackage[titles,subfigure]{tocloft} % Alter the style of the Table of Contents
\renewcommand{\cftsecfont}{\rmfamily\mdseries\upshape}
\renewcommand{\cftsecpagefont}{\rmfamily\mdseries\upshape} % No bold!
\title{Survival Analysis}
\author{Mingwei Fei}

\begin{document}
	
	\maketitle
	
	\section{Sample Size}
	The $ln(HR)$ follows a normal distribution, we use this to calculate the sample size.
	
	\begin{align*}
	ln (\hat{\Delta}) & \sim N\left(ln(\Delta),  \frac{1}{d_1} + \frac{1}{d_2} \right) \\
	\left( \frac{1}{d_1} + \frac{1}{d_2} \right)^{-1} &= \left[ \frac{(z_{\alpha/2} + z_{\beta})^2}{(ln \Delta_0)^2} \right] 
	\end{align*}
	where $d_i$ is the number of observed events. 
	
	If hazard ratio set at 2.1, then 
	\begin{align*}
	\left( \frac{1}{d_1} + \frac{1}{d_2} \right)^{-1} &= \left[ \frac{(1.96 + 0.842)^2}{(ln 2.1)^2} \right] = 14.26\\
	\frac{1}{d_1} + \frac{1}{d_2} &= \frac{1}{14.26} = 0.07, \qquad	d_1 = d_2 = 28.5
	\end{align*}
	
	The one-sided significance level 0.25, power is 0.8. Note that $Z_{\alpha/2}$ is the z score for the probability $1-\alpha/2$, and $z_{\beta}$ is the z score for the probability $1-\beta$. Assume the overall event and censored rate is $20\%$, then the sample size is $57/0.2 = 285$. The total number in the paper is 276.
	
	
	\subsection{Non-inferiority margin Hazard ratio $\Delta_0$ = 2.1}
	The assumption is that control group (C) event rate $10\%$ and treatment group (T) event rate $20\%$ at 6 months. Assume survival function is an exponential distribution:
	\begin{align*}
	S_t(t) &= exp(-\lambda_1 t), \qquad t= 0.5, S_t = 0.8 , -\lambda_1 = ln(0.8)/0.5\\
	S_c(t) &=  exp(-\lambda_2 t), \qquad t= 0.5, S_c = 0.9, -\lambda_2 = ln(0.9)/0.5  \\
	\Delta_0 &= \frac{\lambda_1}{\lambda_2}= \frac{ln(0.8)}{ln(0.9)} = 2.117
	\end{align*}	

	\subsection{Hazard ratio actual = 0.55}
The control group survival $76.8\%$ and treatment group survival $86.2\%$ at 6 months. 
Assume survival function is an exponential distribution:
\begin{align*}
	S_t(t) &= exp(-\lambda_1 t), \qquad t= 0.5, S_t = 0.862 , -\lambda_1 = ln(0.862)/0.5\\
	S_c(t) &=  exp(-\lambda_2 t), \qquad t= 0.5, S_c = 0.768, -\lambda_2 = ln(0.768)/0.5  \\
	HR = \frac{\lambda_1}{\lambda_2}\\
	&= \frac{ln(0.862)}{ln(0.768)} = 0.56
\end{align*}


	\section{Sample Size Formula}
	The test hypothesis is 
\begin{align*}
	H_0: & \lambda_1 = \lambda_2\\
	H_1: & \lambda_1 \neq \lambda_2
\end{align*}	
Or equivalently, in terms of hazard ratio, $\Delta = \lambda_1/\lambda_2$
\begin{align*}
	H_0: & \Delta = 1\\
	H_1: & \Delta \neq 1
\end{align*}
A much simpler and quite accurate approximation for a reasonably large number of events is based on the approximate normality of th natural logarithm of the estimated hazard ratio in each treatment group:
\begin{align*}
	ln(\hat{\lambda}_i) & \sim  N(ln \lambda_i, \frac{1}{d_i})
\end{align*}
where $d_i$ is the number of observed events. Thus, the $ln \Delta = ln \lambda_1 - ln \lambda_2$ also follows a normal distribution with variance $\frac{1}{d_1} + \frac{1}{d_2}$.

\begin{align*}
	ln (\hat{\Delta}) & \sim N\left(ln(\Delta),  \frac{1}{d_1} + \frac{1}{d_2} \right) \\
	\left( \frac{1}{d_1} + \frac{1}{d_2} \right)^{-1} &= \left[ \frac{(z_{\alpha/2} + z_{\beta})^2}{(ln \Delta_0)^2} \right] 
\end{align*}

The calculation of sample size follows
\begin{align*}
	Z &= \frac{ln(\hat{\Delta})}{\sigma}, \qquad \sigma = \sqrt{\frac{1}{d_1} + \frac{1}{d_2}}, \qquad \delta = ln(\Delta_0) \\
 &	P(Z \geq Z_{1-\alpha/2}| H_0)  \leq \alpha/2 \\
 &	P(Z \leq Z_{\beta}| H_1= \delta)  \geq \beta
\end{align*}
So we set Z satisfy the below equation
\begin{align*}
	\frac{ln(\hat{\Delta})}{\sigma} &= Z_{1-\alpha/2}, \qquad &\text{ H }_0\\
	\frac{ln(\hat{\Delta}) - \delta}{\sigma} &= Z_{\beta},  \qquad &\text{H}_1
\end{align*}
So we have
\begin{align*}
	ln(\hat{\Delta}) &= Z_{1-\alpha/2} \sigma, \qquad	ln(\hat{\Delta}) = Z_{\beta} {\sigma} + \delta ,\qquad	Z_{1-\alpha/2} \sigma = Z_{\beta} {\sigma} + \delta \\
	\sigma &= \frac{\delta}{Z_{1-\alpha/2} - Z_{\beta}}, \qquad 	\frac{1}{d_1} + \frac{1}{d_2} = \frac{\delta^2}{(Z_{1-\alpha/2} + Z_{1-\beta})^2}
\end{align*}

\section{Hazard Rate Asymptotic Distribution}

\subsection{Likelihood Function}
If $T_i$ and $C_i$ are independent, which means non-informative censoring. We look at the cumulative conditional probability at time T:

\begin{align*}
	p(T \leq s + \epsilon | T \geq s) & \approx p(T < s+ \epsilon | T \geq s, C \geq s)
\end{align*}

Note that the above probability is not the hazard rate, it is the cumulative hazard rate. The hazard rate is as below

\begin{align*}
	 h(t) & = \frac{p(t)}{S(t)} = p( s \leq T \leq s + \epsilon | T \geq s)
\end{align*}

The key of success is to construct likelihood function. We use conditional probability in the situation when there are hidden variables that we can't or don't need to estimate. When there are censoring time, we don't know exactly what those censoring times are.

So in the presence of censoring, we only observe $(T_i, \delta_i), i=1,..n$. Let us suppose that $T_i$ is the survival time, which may not be observed and we observe instead $U_i = min(T_i, C_i)$, where $C_i$ is the potential censoring time. 

\begin{align*}
	\delta_i & =
	\begin{cases}
		1 \quad T_i \leq C_i, \qquad \text{Uncensored}\\
		0 \quad  T_i > C_i, \qquad \text{Censored}
	\end{cases}
\end{align*}

\subsubsection{Likelihood under Censoring}
The likelihood under censoring can be constructed using both the density and distribution
functions or the hazard and cumulative hazard functions. Both are equivalent. The loglikelihood will be a mixture of probabilities and densities, depending on whether the
observation was censored or not. 

Let us suppose that $T_i$ has distribution $f(x, \theta_0)$, where f is known but $\theta_0$ is unknown. The likelihood construction must be with respect to the bivariate, random variable $(U_i, \delta_i)$.

We observe $(U_i, \delta_i)$ where $U_i = min(T_i, C_i)$ and $\delta_i$ is the
indicator variable. In this section we treat $C_i$ as if they were deterministic, we consider
the case that they are random later.

We first observe that if $\delta_i = 1$, then the log-likelihood of the individual observation $U_i$ is $log f(U_i, \theta)$, since

\begin{align*}
	P(U_i= x| \delta_i=1) & = P(T_i= x| T_i \leq c_i) = \frac{f(x; \theta)}{ 1- S(x, \theta)} dx\\
	& = \frac{h(x) S(x,\theta)}{1- S(x, \theta)} dx
\end{align*}
where $S(x, \theta)$ is the survival function $1- F(T_i \leq x)$.

On the other hand, if $\delta_i = 0$, the log likelihood of the individual observation $U_i = c_i|\delta_i = 0$ is simply one, since if $\delta_i = 0$, then $U_i = c_i$ (it is given). Of course it is clear that $p(\delta_i= 1) = 1 - S(c_i, \theta)$ and $P(\delta_i = 0) = S(c_i; \theta)$. Thus altogether the joint density of ${U_i, \delta_i}$ is

\begin{align*}
	p(U_i, \delta_i) & = \left(\frac{f(x; \theta)}{1- S(c_i, \theta)} (1- S(c_i, \theta)) \right)^{\delta_i} \left(1 \times S(c_i, \theta) \right)^{1-\delta_i} \\
		&=f(x, \theta)^{\delta_i} [S(c_i, \theta)]^{1-\delta_i}\\
		&= h(u_i)^{\delta_i} S(u_i)\\
	p(\theta) &= \prod_{i=1}^n h(u_i)^{\delta_i} S(u_i)
\end{align*}

	
Therefore by using

\begin{align*}
	f(U_i, \theta) & = h(U_i, \theta) S(U_i, \theta) \\
	H(U_i, \theta) &= -log S(U_i, \theta)
\end{align*}

The joint log-likelihood of ${(U_i, \delta_i)}_{i=1}^n$ is

\begin{align*}
	ln(\theta) & = \sum_{i=1}^n \left(\delta_i log f(\theta) + (1-\delta_i) log (1- F(\theta)) \right) \\
	&= \sum_{i=1}^n \delta_i \left[log h(T_i, \theta) - H(T_i, \theta) \right] - \sum_{i=1}^n  (1-\delta_i) H(c_i, \theta)  \\
	&= \sum_{i=1}^n \delta_i log h(U_i, \theta) - \sum_{i=1}^n  (1-\delta_i) H(U_i,  \theta)
\end{align*}

We can get the MLE of $\theta$ by score function, Fisher Information to get the variance. 

\subsection{Exponential Distribution}
Suppose that $T_1, T_2, ... T_n$ are i.i.d $Exp(\lambda)$ and subject to noninformative right censoring. The exponential distribution

\begin{align*}
	f(x, \lambda) & = \lambda exp(-\lambda x)
\end{align*}

The survival function
\begin{align*}
	S(x, \lambda) & = 1- F(x, \lambda) =1- \int_{0}^x \lambda exp(-\lambda x) dx = exp(-\lambda x)
\end{align*}

The likelihood function

\begin{align*}
	ln(\lambda) & = \prod_{i=1}^n \lambda^{\delta_i} exp(-\lambda u_i) = \lambda^r exp(-\lambda W)
\end{align*}
where $r = \sum_{i=1}^n \delta_i$ are the number of failures; $W=\sum_{i=1}^n u_i$ is total followup time.

The Score function and observed information
\begin{align*}
	\diffp{ln(\lambda)}{\lambda} & = \frac{r}{\lambda} -W\\
	-\diffp{ln(\lambda)}{{\lambda}{\lambda}} &= \frac{r}{\lambda^2}
\end{align*}
$\hat\lambda$ approximately follows $N(\lambda, \lambda^2/r)$ for large n.

By delta method,

\begin{align*}
	log(\hat\lambda) \sim N(log(\lambda), r^{-1})
\end{align*}
The variance of log hazard ratio $r^{-1}$ is free of the unknown parameter $\lambda$. Similarly, we see that the log of odds ratio is used more common than odds ratio. 

\end{document}