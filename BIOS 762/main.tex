\documentclass[a4paper,11pt]{book}
\usepackage{import}
\usepackage{amsmath}
\usepackage{mathtools}
\usepackage[utf8]{inputenc} % set input encoding (not needed with XeLaTeX)
\usepackage[thinc]{esdiff}
\usepackage[english]{babel}
\usepackage{amsthm}
\theoremstyle{definition}
\newtheorem{definition}{Definition}[section]
\newtheorem{theorem}{Theorem}[section]
\newtheorem{corollary}{Corollary}[theorem]
\newtheorem{lemma}[theorem]{Lemma}
\theoremstyle{remark}
\newtheorem*{remark}{Remark}
%\usepackage{example}

%%% Examples of Article customizations
% These packages are optional, depending whether you want the features they provide.
% See the LaTeX Companion or other references for full information.

%%% PAGE DIMENSIONS
\usepackage{geometry} % to change the page dimensions
\geometry{a4paper} % or letterpaper (US) or a5paper or....
% \geometry{margin=2in} % for example, change the margins to 2 inches all round
% \geometry{landscape} % set up the page for landscape
%   read geometry.pdf for detailed page layout information

\usepackage{graphicx} % support the \includegraphics command and options

% \usepackage[parfill]{parskip} % Activate to begin paragraphs with an empty line rather than an indent

%%% PACKAGES
\usepackage{booktabs} % for much better looking tables
\usepackage{array} % for better arrays (eg matrices) in maths
\usepackage{paralist} % very flexible & customisable lists (eg. enumerate/itemize, etc.)
\usepackage{verbatim} % adds environment for commenting out blocks of text & for better verbatim
\usepackage{subfig} % make it possible to include more than one captioned figure/table in a single float
% These packages are all incorporated in the memoir class to one degree or another...

%%% HEADERS & FOOTERS
\usepackage{fancyhdr} % This should be set AFTER setting up the page geometry
\pagestyle{fancy} % options: empty , plain , fancy
\renewcommand{\headrulewidth}{0pt} % customise the layout...
\lhead{}\chead{}\rhead{}
\lfoot{}\cfoot{\thepage}\rfoot{}

%%% SECTION TITLE APPEARANCE
\usepackage{sectsty}
\allsectionsfont{\sffamily\mdseries\upshape} % (See the fntguide.pdf for font help)
% (This matches ConTeXt defaults)

%%% ToC (table of contents) APPEARANCE
\usepackage[nottoc,notlof,notlot]{tocbibind} % Put the bibliography in the ToC
\usepackage[titles,subfigure]{tocloft} % Alter the style of the Table of Contents
\renewcommand{\cftsecfont}{\rmfamily\mdseries\upshape}
\renewcommand{\cftsecpagefont}{\rmfamily\mdseries\upshape} % No bold!
\usepackage{makeidx}
\makeindex

\begin{document}

\frontmatter
\import{./}{title.tex}
\clearpage
%\thispagestyle{empty}

\tableofcontents

\mainmatter

\chapter{Matrix}
\import{sections/}{section1-1.tex}
\import{sections/}{section1-2.tex}
\import{sections/}{section1-3.tex}
%\import{sections/}{section1-4.tex}
\import{sections/}{section1-5.tex}

\chapter{Generalized Linear Model}


\subsubsection{Likelihood Ratio Test based on likelihood}

Derive the likelihood ratio test for the hypothesis in part (e) and derive its asymptotic distribution under $H_0$.
From part (e), we have the parameter estimates under $H_0$. While under alternative hypothesis, we have $\mu_{ij} = n_{ij}$. 
\begin{align*}
	LRT_n &= 2(LR(\pi_{H_1}) - LR(\pi_{H_0})) =2\left( \sum_{i=1}^I \sum_{j=1}^J n_{ij} log \pi_{ij} - \sum_{i=1}^I \sum_{j=1}^J n_{ij} log \pi_{i+} \pi_{+j} \right)\\
	&= 2\left( \sum_{i=1}^I \sum_{j=1}^J n_{ij} log \frac{\pi_{ij}}{\pi_{i+} \pi_{+j} }   \right)\\
	&= 2\left( \sum_{i=1}^I \sum_{j=1}^J n_{ij} log \frac{n_{ij} n}{n_{i+} n_{+j} }   \right) \sim \chi^2_{(I-1)(J-1)} 
\end{align*}
Note that the full model has $(IJ-1)$ parameters, and the null hypothesis has $(I-1)+ (J-1)$ parameters.
\begin{align*}
	df &= I \times J-1 - (I-1) - (J-1)\\
	&= (I-1)(J-1)
\end{align*}

\subsubsection{Conditional Probability}

Suppose that $\pi_{11}, \pi_{12}$ are parameters of interest and the rest of the parameters are treated as nuisance. Derive the conditional likelihood of $(\pi_{11}, \pi_{12})$ and the conditional MLE's of  $(\pi_{11}, \pi_{12})$.
If not specified, we treat as general contingency table that total n is fixed. If only $\pi_{11}, \pi_{12}$ are parameters of interest and the rest of the parameters are treated as nuisance, then we will set the rest of the parameters as one parameter, and get its distribution, which is to find the sufficient statistics for rest of the parameters.
Write the Multinomial distribution in exponential family distribution.\\
We can find marginal distribution by summing over along all possible values of $(n_{11}, n_{12})$. Note that $n_{11} \leq \min{n_{1+} - n_{12}, n_{+1}}$ for a given value of $n_{12}$. Similarly, $n_{12} \leq \min{n_{1+}- n_{11}, n_{+1}}$ for a given value of $n_{11}$. \\
Additionally,
\begin{align*}
	n & \geq n_{1+} + n_{+1} + n_{+2} - n_{11} - n_{12} \\
	n_{11} + n_{12} & \geq \max{ 0, n_{+1} + n_{1+} + n_{+2}}
\end{align*}
Let
\begin{align*}
	S(n_{11}, n_{12}) &= \{(n_{11}, n_{12}): n_{11} + n_{12} \geq \max{ 0, n_{+1} + n_{1+} + n_{+2}},\\
	&  n_{11} \leq \min{(n_{1+} - n_{12}, n_{+1})}, n_{12} \leq \min{(n_{1+}- n_{11}, n_{+1})}   \} 
\end{align*}

The conditional distribution
\begin{align*}
	p(n_{11}, n_{12}|n_{13}, ...n_{IJ}, n) &= \frac{p(n_{ij}}{p(S_n)}\\
	&= \frac{\frac{1}{n_{11}! n_{12}! } \pi_{11}^{n_{11}} \pi_{12}^{n_{12}}}{\sum_{(x, y \in S_n)} \frac{1}{x! y!} \pi_{11}^x \pi_{12}^y}
\end{align*}
And $\hat{\pi}_{11}, \hat{\pi}_{12}$ are the CMLE that maximize $p(n_{11}, n_{12}|n_{13}, ...n_{IJ}, n)$.



\section{Practice}
\subsection{Contingency table parameters}
\begin{itemize}
	\item [(a)] Get MLE of $\pi$ and prove CLT.\\
	The multinomial distribution based on total n. 
	\begin{align*}
		p(\theta) &=n! \prod_{i=0}^1 \prod_{j=0}^1  \frac{\pi_{ij}^{n_{ij}}}{n_{ij}!}, \qquad \theta = (\pi_{00}, \pi_{01}, \pi_{10}, \pi_{11})^T\\
		ln p(\theta) &=log n!+ \sum_{i=0}^1 \sum_{j=0}^1 n_{ij}log( \pi_{ij}) - log n_{ij}! \\
		&= log n!+ n_{00}log \pi_{00}  + n_{01}log \pi_{01}  + n_{10}log \pi_{10}  + n_{11}log (1-\pi_{00}-\pi_{01} - \pi_{10})  
	\end{align*}
	The MLE of the $\theta$ by taking derivative to the log-likelihood
	\begin{align*}
		\frac{\partial ln(\theta)}{\partial \pi_{00}} &= \frac{n_{00}}{\pi_{00}} - \frac{n_{11}}{1-\pi_{00}-\pi_{01}-\pi_{10}} = 0\\  
		\frac{\partial ln(\theta)}{\partial \pi_{01}} &=\frac{n_{01}}{\pi_{01}} - \frac{n_{11}}{1-\pi_{00}-\pi_{01}-\pi_{10}} = 0 \\  
		\frac{\partial ln(\theta)}{\partial \pi_{10}} &= \frac{n_{10}}{\pi_{10}} - \frac{n_{11}}{1-\pi_{00}-\pi_{01}-\pi_{10}} = 0\\ 
		\hat{\pi_{00}} & = \frac{n_{00}}{n}\\
		\hat{\pi_{01}} & = \frac{n_{01}}{n}\\
		\hat{\pi_{10}} & = \frac{n_{10}}{n}\\
		\hat{\pi_{11}} & = \frac{n_{11}}{n}, \qquad n= n_{00} + n_{01} + n_{10} + n_{11}
	\end{align*}
	Let $Z_i= I(X=x, Y=y) \sim $ multi $(1, \pi_{00}, \pi_{01}, \pi_{10}, \pi_{11})$.
	\begin{align*}
		Z_1 &= I[(X,Y)= (0,0)]\\
		Z_2 &= I[(X,Y)= (0,1)]\\
		Z_3 &= I[(X,Y)= (1,0)]\\
		Z_4 &= I[(X,Y)= (1,1)]\\
		p(\theta) &= \prod_k \pi_{k}^{I(Z_k=1)}\\
		M_Z(t) &= E[exp(t^TZ)] = E[exp(t^T(Z_1 + Z_2 +... Z_n))] = E[exp(t^TZ_1 + t^TZ_2 + ... t^TZ_n)]\\
		&= E[\prod_{i=1}^n exp(t^TZ_i)]\\
		&= \prod_{i=1}^n E[exp(t^TZ_i)]  \qquad (\text{by independence})\\
		&= \prod_{i=1}^n M_{Z_i}(t) = \prod_{i=1}^n P(Z_i= 1) e^{tz_i}\qquad  \text{by MGF of discrete variable $Z_i$}\\
		&= \left( \sum_{j=1}^J \pi_j exp(t_j)\right)^n \qquad \text{by MGF of multinoulli}
	\end{align*}  
	Then the covariance matrix of $\theta$ could be calculated by MGF.
	\begin{align*}
		E(Z_1 Z_2) &= \frac{\partial^2 M_Z(t)}{\partial Z_i \partial Z_j}|_{t_i = t_j = 0}\\
		&= \frac{\partial \left(n(\pi_ie^{t_i})(\sum_{k=1}^K \pi_ke^{t_k})^{n-1} \right)'}{\partial t_j}\\
		&= n(n-1)(\sum_{k=1}^K \pi_ke^{t_k})^{n-2}\pi_i\pi_j|_{t_i = t_j = 0} = n(n-1)\pi_i\pi_j\\
		E(X_i) &= n\pi_i\\
		Cov(Z_i, Z_j) &= E(Z_i Z_2) - E(Z_1)E(Z_j) = n(n-1)\pi_i\pi_j - n^2 \pi_i\pi_j = -n\pi_i\pi_j\\
		Var(Z_i) &= E(Z_i^2) - E(Z_i)^2 \\
		E(Z_i^2) &=  \frac{\partial \left(n(\pi_ie^{t_i})(\sum_{k=1}^K \pi_ke^{t_k})^{n-1} \right)'}{\partial t_i}\\
		&= n(\sum_{k=1}^K \pi_ke^{t_k})^{n-1}\pi_i e^{t_i}+ n(n-1)(\sum_{k=1}^K \pi_ke^{t_k})^{n-2}\pi_i\pi_i e^{2t_i}|_{t_i = 0} \\
		&= n\pi_i + n(n-1)\pi_i^2 = n\pi_i(1-\pi)\\
		Var(Z_i/n) &= \frac{1}{n^2} Var(Z_i) = \frac{1}{n}\pi_i(1-\pi_i)
	\end{align*}
	Thus the covariance matrix is
	\begin{align*}
		\Sigma &= \begin{bmatrix}
			\pi_{00}(1-\pi_{00}) &  -\pi_{00}\pi_{01}&  -\pi_{00}\pi_{10} &  -\pi_{00}\pi_{11}\\
			-\pi_{01}\pi_{00} & \pi_{01}(1-\pi_{01}) & -\pi_{01}\pi_{10}   & -\pi_{01}\pi_{11}  \\
			-\pi_{10}\pi_{00} & -\pi_{10}\pi_{01} &  \pi_{10}(1-\pi_{10})  & -\pi_{10}\pi_{11}  \\
			-\pi_{11}\pi_{00} &  -\pi_{11}\pi_{01} & -\pi_{11}\pi_{10}   & \pi_{11}(1-\pi_{11})  \\
		\end{bmatrix}= diag{(\pi_{ij}) - \theta \theta^T}
	\end{align*}
	By Central limit theroem, 
	\begin{align*}
		\sqrt{n} (\hat{\pi_{00}} - \pi_{00}, \hat{\pi_{01}}- \pi_{01}, \hat{\pi_{10}} - \pi_{10}, \hat{\pi_{11}}- \pi_{11} )^T & \xrightarrow[]{d} N(0, \Sigma)
	\end{align*}
	\item[(b)] Let R denote the odds ratio. Find the maximum likelihood estimate of log(R) and
	derive its asymptotic distribution.\\
	By invariance of MLE:
	\begin{align*}
		R & =  \frac{\pi_{00}\pi_{11}}{\pi_{01}\pi_{10}}\\
		g(R) &= log R = log \pi_{00} + log \pi_{11}- log \pi_{01}- log \pi_{10}\\
		log \hat{R} & = log \hat{\pi_{00}} + log \hat{\pi_{11}}- log \hat{\pi_{01}}- log \hat{\pi_{10}}\\
		&= log \frac{n_{00}n_{11}}{n_{01}n_{10}}
	\end{align*}
	
	By Central limit theorem, we have 
	\begin{align*}
		\sqrt{n} \left(\hat{g(R)} - g(R) \right) & \xrightarrow[]{d} N \left(0, \frac{\partial g(R)}{\partial \theta} \Sigma   \frac{\partial g(R)}{\partial \theta}^T \right) \\
	\end{align*}
	By delta method,
	\begin{align*}
		\frac{\partial g(R)}{\partial \theta} &= \left(
		\frac{1}{R} \frac{\partial R}{\partial \pi_{00}} ,  \frac{1}{R}\frac{\partial R}{\partial \pi_{01}},   \frac{1}{R}\frac{\partial R}{\partial \pi_{10}} ,  \frac{1}{R} \frac{\partial R}{\partial \pi_{11}} \right)\\
		& = \left( \frac{1}{\pi_{00}},  -\frac{1}{\pi_{01}},  -\frac{1}{\pi_{10}}, \frac{1}{\pi_{11}} \right)\\
		\Sigma^{R} &= \frac{\partial g(R)}{\partial \theta} \Sigma \frac{\partial g(R)}{\partial \theta}' \\
		&= \left( \frac{1}{\pi_{00}},  -\frac{1}{\pi_{01}},  -\frac{1}{\pi_{10}}, \frac{1}{\pi_{11}} \right) \begin{bmatrix}
			\pi_{00}(1-\pi_{00}) &  -\pi_{00}\pi_{01}&  -\pi_{00}\pi_{10} &  -\pi_{00}\pi_{11}\\
			-\pi_{01}\pi_{00} & \pi_{01}(1-\pi_{01}) & -\pi_{01}\pi_{10}   & -\pi_{01}\pi_{11}  \\
			-\pi_{10}\pi_{00} & -\pi_{10}\pi_{01} &  \pi_{10}(1-\pi_{10})  & -\pi_{10}\pi_{11}  \\
			-\pi_{11}\pi_{00} &  -\pi_{11}\pi_{01} & -\pi_{11}\pi_{10}   & \pi_{11}(1-\pi_{11})  \\
		\end{bmatrix} \begin{bmatrix}
			\frac{1}{\pi_{00}} \\
			-\frac{1}{\pi_{01}}   \\
			-\frac{1}{\pi_{10}}  \\
			\frac{1}{\pi_{11}}  \\
		\end{bmatrix}\\
		&= (\frac{1}{\pi_{00}} + \frac{1}{\pi_{01}} + \frac{1}{\pi_{10}} + \frac{1}{\pi_{11}})\\
	\end{align*}
	We have the asymptotic distribution of $log(R)$
	\begin{align*}
		\sqrt{n} (log\hat{R} - logR) & \xrightarrow[]{d} N \left(0, (\frac{1}{\pi_{11}} + \frac{1}{\pi_{12}} + \frac{1}{\pi_{21}} + \frac{1}{\pi_{22}}) \right) 
	\end{align*}
	\item[(c)] Construct an approximate 95$\%$ confidence interval for the odds ratio R.\\
	From part (b), we have the asymptotic normal distribution of $log R$. We have the asymptotic distribution of $R$.
	\begin{align*}
		f &= exp(g) = R, \qquad f(g)' = R\\
		\sqrt{n} (\hat{f(g)} - f(g)) & \xrightarrow[]{d} N \left(0, f(g)'(\frac{1}{\pi_{11}} + \frac{1}{\pi_{12}} + \frac{1}{\pi_{21}} + \frac{1}{\pi_{22}}) f(g)'^T \right)\\
		\sqrt{n} (\hat{R} - R) & \xrightarrow[]{d} N \left(0, R^2(\frac{1}{\pi_{11}} + \frac{1}{\pi_{12}} + \frac{1}{\pi_{21}} + \frac{1}{\pi_{22}}) \right)\\
		(\hat{R} - R) & \xrightarrow[]{d} N \left(0, \frac{1}{n} R^2(\frac{1}{\pi_{11}} + \frac{1}{\pi_{12}} + \frac{1}{\pi_{21}} + \frac{1}{\pi_{22}}) \right)
	\end{align*}
	The 95$\%$ confidence interval for the odds ratio R
	\begin{align*}
		\{R &: \hat{R} - 1.96\hat{R} \sqrt{\frac{1}{\pi_{11}} + \frac{1}{\pi_{12}} + \frac{1}{\pi_{21}} + \frac{1}{\pi_{22}}} \leq  R \leq \hat{R} + 1.96\hat{R} \sqrt{\frac{1}{\pi_{11}} + \frac{1}{\pi_{12}} + \frac{1}{\pi_{21}} + \frac{1}{\pi_{22}}} \}
	\end{align*}
	
	\item[(d)] Under the assumptions of part (a), further assume that$ \pi_{1+} = \pi_{11} + \pi_{10} = \frac{exp(\alpha)}{1+\exp(\alpha)} $ and $ \pi_{+1} = \pi_{11} + \pi_{01} = \frac{exp(\alpha + \beta)}{1+\exp(\alpha + \beta)} $ . Derive the maximum likelihood estimates of $(\alpha, \beta)$, denoted by $(\hat{\alpha}; \hat{\beta})$.\\
	\begin{align*}
		\pi_{01} + \pi_{11} & = \frac{exp(\alpha)}{1+\exp(\alpha)} \\
		exp(\alpha) &= \frac{\pi_{10} + \pi_{11}}{\pi_{01} + \pi_{00}}, \qquad \alpha = log \left( \frac{\pi_{10} + \pi_{11}}{\pi_{01} + \pi_{00}}\right)\\
		\pi_{10}+ \pi_{11} & = \frac{exp(\alpha + \beta)}{1+\exp(\alpha + \beta)} \\
		\alpha + \beta &= log \left( \frac{\pi_{01} + \pi_{11}}{\pi_{10} + \pi_{00}} \right)\\
		\beta &= log \left( \frac{\pi_{01} + \pi_{11}}{\pi_{10} + \pi_{00}} \right) - log \frac{\pi_{10} + \pi_{11}}{\pi_{01} + \pi_{00}}, \qquad \beta &= log \left(\frac{(\pi_{01} + \pi_{11})(\pi_{01} + \pi_{00})}{(\pi_{10} + \pi_{00}) (\pi_{10} + \pi_{11})} \right)
	\end{align*}
	By invariance of MLE,
	\begin{align*}
		\hat\alpha &= log \left( \frac{\hat{\pi_{10}} + \hat{\pi_{11}}}{\hat{\pi_{01}} + \hat{\pi_{00}}}\right) = log \left(\frac{n_{10} + n_{11}}{n_{01} + n_{00}} \right)\\
		\hat\beta &= log \left(\frac{(\hat\pi_{01} + \hat\pi_{11})(\hat\pi_{01} + \hat\pi_{00})}{(\hat\pi_{10} + \hat\pi_{00}) (\hat\pi_{10} + \hat\pi_{11})} \right) = log \left(\frac{(n_{01} + n_{11})(n_{01} + n_{00})}{(n_{10} + n_{00}) (n_{10} + n_{11})} \right)
	\end{align*}
	\item[(e)] Using the assumptions of part (d), derive the asymptotic distribution of $(\alpha, \beta)$ (properly normalized).\\
	By Central limit theorem and delta method,
	\begin{align*}
		\xi &= (\alpha, \beta)^T \\
		g(\xi) &= \{ log \left( \frac{\pi_{10} + \pi_{11}}{\pi_{01} + \pi_{00}}\right), log \left(\frac{(\pi_{01} + \pi_{11})(\pi_{01} + \pi_{00})}{(\pi_{10} + \pi_{00}) (\pi_{10} + \pi_{11})} \right)\}^T \\
		\sqrt{n} (\hat{g(\xi)} - g(\xi)) & \xrightarrow[]{d} N \left(0, \Sigma^{N} \right) \\
		\Sigma^{N} &= \frac{\partial g(\xi)}{\partial \pi} \Sigma \frac{\partial g(\xi)}{\partial \pi}^T
	\end{align*}
	
	$\Sigma^{N}$ is calculated by delta method,
	\begin{align*}
		\frac{\partial g(\alpha)}{\partial \pi_{00}} &= -\frac{1}{(\pi_{01} + \pi_{00})} = -\frac{1}{\pi_{0+}} \\
		\frac{\partial g(\alpha)}{\partial \pi_{01}} &= -\frac{1}{(\pi_{01} + \pi_{00})} = -\frac{1}{\pi_{0+}}\\
		\frac{\partial g(\alpha)}{\partial \pi_{10}} &= \frac{1}{(\pi_{10} + \pi_{11})}= \frac{1}{\pi_{1+}}\\
		\frac{\partial g(\alpha)}{\partial \pi_{11}} &= \frac{1}{(\pi_{10} + \pi_{11})}= \frac{1}{\pi_{1+}}\\
		\frac{\partial g(\beta)}{\partial \pi_{00}} &= \frac{(\pi_{10}-\pi_{01})}{(\pi_{01} + \pi_{00})(\pi_{00} + \pi_{10})} = -\frac{1}{(\pi_{10} + \pi_{00})} +\frac{1}{(\pi_{01} + \pi_{00})} = -\frac{1}{\pi_{+0} }  +\frac{1}{\pi_{0+}}\\
		\frac{\partial g(\beta)}{\partial \pi_{01}} &= \frac{1}{(\pi_{01} + \pi_{11})} + \frac{1}{(\pi_{01} + \pi_{00})}  \\
		\frac{\partial g(\beta)}{\partial \pi_{10}} &=- \frac{1}{(\pi_{10} + \pi_{00})} - \frac{1}{(\pi_{10} + \pi_{11})}\\
		\frac{\partial g(\beta)}{\partial \pi_{11}} &= \frac{(\pi_{10}-\pi_{01})}{(\pi_{10} + \pi_{11})(\pi_{01} + \pi_{11})} = - \frac{1}{(\pi_{10} + \pi_{11})} +\frac{1}{(\pi_{01} + \pi_{11})} \\
		\frac{\partial g(\xi)}{\partial \pi} &=\begin{bmatrix}
			-\frac{1}{\pi_{0+}} &  -\frac{1}{\pi_{0+}} &  \frac{1}{\pi_{1+}} &  \frac{1}{\pi_{1+}}\\
			\frac{1}{\pi_{0+} }  -\frac{1}{\pi_{+0}} & \frac{1}{\pi_{0+} } + \frac{1}{\pi_{+1}} & - \frac{1}{\pi_{+0} } - \frac{1}{\pi_{1+}} & \frac{1}{\pi_{+1} } -\frac{1}{\pi_{1+}}    \\
		\end{bmatrix}\\
		\Sigma^{N} &= \frac{\partial g(\xi)}{\partial \pi}\Sigma \frac{\partial g(\xi)}{\partial \pi}^T\\
		&= \left(\frac{1}{\pi_{11}} + \frac{1}{\pi_{12}} + \frac{1}{\pi_{21}} + \frac{1}{\pi_{22}} \right) 
	\end{align*}
	\item[(f)] Under the model of part (d), show that $(\pi_{1+}\pi_{0+})^{-1} + (\pi_{+1}\pi_{+0})^{-1} \leq (\pi_{1+}\pi_{+0})^{-1} + (\pi_{+1}\pi_{0+})^{-1}$.\\
	\begin{align*}
		&(\pi_{1+}\pi_{+0})^{-1} + (\pi_{+1}\pi_{0+})^{-1} - (\pi_{1+}\pi_{0+})^{-1} - (\pi_{+1}\pi_{+0})^{-1}\\
		&= \frac{\pi_{0+}- \pi_{+0}}{\pi_{1+}\pi_{+0}\pi_{0+}} + \frac{\pi_{+0} - \pi_{0+}}{\pi_{+1}\pi_{0+}\pi_{+0}}\\
		&= \frac{(\pi_{0+}-\pi_{+0})(\pi_{+1}-\pi_{1+})}{\pi_{1+}\pi_{+0}\pi_{0+}\pi_{+1}}\\
		&=  \frac{(\pi_{01}-\pi_{10})^2}{\pi_{1+}\pi_{+0}\pi_{0+}\pi_{+1}} \geq 0
	\end{align*}
	From above, we have $(\pi_{1+}\pi_{0+})^{-1} + (\pi_{+1}\pi_{+0})^{-1} \leq (\pi_{1+}\pi_{+0})^{-1} + (\pi_{+1}\pi_{0+})^{-1}$.
\end{itemize}

 

  
\section{Exercise}
Consider pairs of independent random variables $(y_{i1}, y_{i2}), i = 1, · · · , n$ such that both $y_{i1}$ and $y_{i2}$ follow a $N(\mu_i, \psi)$ distribution. Let $\psi$ be the parameter of interest and the $\mu_i$ are nuisance parameters.
\begin{itemize}
	\item [(a)] Show that the maximum likelihood estimate of $\psi$ is inconsistent.\\
	The joint density of $y_{i1}, y_{i2}$
	\begin{align*}
		P(y_{i1}, y_{i2}) &= \frac{1}{2\pi \psi} exp \left(-\frac{(y_{i1}-\mu_i)^2 + (y_{i2}-\mu_i)^2}{2 \psi} \right)\\
		P(y_{1}, y_{2}) &=\prod_{i=1}^n \frac{1}{(2\pi \psi)^n} exp \left(- \sum_{i=1}^n  \frac{(y_{i1}-\mu_i)^2 + (y_{i2}-\mu_i)^2}{2 \psi} \right)
	\end{align*}
	The log-likelihood function
	\begin{align*}
		ln(y_{1}, y_{2}) &= -n log(2\pi) - nlog\psi -\sum_{i=1}^n  \frac{(y_{i1}-\mu_i)^2 + (y_{i2}-\mu_i)^2}{2 \psi} 
	\end{align*}
	Obtain MLE of $\mu_i, \psi$  
	\begin{align*}
		\partial_{\mu_i}ln  &=-1/(2\psi) \sum_{i=1}^n -2 (y_{i1} - \mu_i + y_{i2} - \mu_i)= 0, \qquad \hat{\mu_i} \\
		\mu_i &= 1/2 (y_{i1}+ y_{i2})\\
		\partial_{\psi}ln  &= -n/\psi +  \frac{\sum_{i=1}^n  [(y_{i1}-\mu_1)^2 + (y_{i2}-\mu_2)^2]}{2\psi^2} = 0\\
		\hat{\psi} &= 1/2n \left( \sum_{i=1}^n  [(y_{i1}-\mu_1)^2 + (y_{i2}-\mu_2)^2] \right) \\
		&= \frac{1}{4n} \sum_{i=1}^n  (y_{i1}-y_{i2})^2
	\end{align*}
	As $E(y_{i1} - y_{i2}) = 0, Var(y_{i1} - y_{i2}) = 2\psi $
	\begin{align*}
		Var(y_{i1} - y_{i2}) &=  E(y_{i1} - y_{i2})^2 - [E(y_{i1} - y_{i2})]^2 = 2\psi, \qquad  E(y_{i1} - y_{i2})^2 = 2\psi\\
	\end{align*}
	By WLLN, 
	\begin{align*}
		\hat{\psi} &= \frac{1}{4n} \sum_{i=1}^n  (y_{i1}-y_{i2})^2 \xrightarrow{n \rightarrow \infty}  1/4 E(y_{i1} - y_{i2})^2 = \psi/2 \neq \psi
	\end{align*}
	So MLE of $\psi$ is not consistent.
	\item[(b)]  Construct a consistent estimate for $\psi$ based on the available information.\\
	From part(a), we can construct $\Tilde{\psi} = 2\hat{\psi} = \frac{1}{2n} \sum_{i=1}^n  (y_{i1}-y_{i2})^2$.
	By WLLN, the
	\begin{align*}
		\Tilde{\psi} &= \frac{1}{2n} \sum_{i=1}^n  (y_{i1}-y_{i2})^2 \xrightarrow[n \rightarrow \infty]{p} = \psi
	\end{align*}   
	
	\item[(c)]  Assume that $y_{i1}$ and $y_{i2}$ follow a $N(\mu_i, \psi_i)$ distribution for $i = 1, · · · , n$,
	where $\mu_i = \beta_0 + \beta_1(x_i - \bar{x})$ and $\psi_i = exp(\alpha_0 + \alpha_1(x_i - \bar{x}))$, in which $x_i$ is
	a covariate of interest and $\bar{x}$ is the mean of the $x_i$s. Derive the score test statistic for testing homogeneous variance.\\
	The hypothesis are
	\begin{align*}
		H_0 &: \alpha_1 = 0 \\
		H_1 &: \alpha_1 \neq 0 
	\end{align*}  
	The log-likelihood function
	\begin{align*}
		\xi &= (\beta_0, \beta_1, \alpha_0, \alpha_1)^T\\
		ln(y_{1}, y_{2}, \mu_i, \psi_i) &= -n log(2\pi) - \sum_{i=1}^n log\psi_i -\sum_{i=1}^n  \frac{(y_{i1}-\mu_i)^2 + (y_{i2}-\mu_i)^2}{2 \psi_i} \\
		ln(y_{1}, y_{2}, \xi) &= -n log(2\pi) - \sum_{i=1}^n (\alpha_0 + \alpha_1(x_i - \bar{x})) \\
		& -\sum_{i=1}^n  \frac{(y_{i1}-\beta_0 - \beta_1(x_i - \bar{x}))^2 + (y_{i2}-\beta_0 - \beta_1(x_i - \bar{x}))^2}{2 exp(\alpha_0 + \alpha_1(x_i - \bar{x}))}, \qquad \sum x_i -\bar{x} = 0\\
		&= -n log(2\pi) - n\alpha_0 -1/2 \sum_{i=1}^n  \frac{(y_{i1}-\beta_0 - \beta_1(x_i - \bar{x}))^2 + (y_{i2}-\beta_0 + \beta_1(x_i - \bar{x}))^2}{exp(\alpha_0 + \alpha_1(x_i - \bar{x}))}
	\end{align*}
	We will get the score function and Fisher information for $\xi$
	\begin{align*}
		\frac{ \partial ln(\xi)}{\partial \alpha_0} &= - n + 1/2 \sum_{i=1}^n  \frac{(y_{i1}-\beta_0 - \beta_1(x_i - \bar{x}))^2 + (y_{i2}-\beta_0 - \beta_1(x_i - \bar{x}))^2}{exp(\alpha_0 + \alpha_1(x_i - \bar{x}))}\\
		&= - n + 1/2  \sum_{i=1}^n \psi_i^{-1} [(y_{i1}-\mu_i)^2 + (y_{i2}-\mu_i)^2]\\
		\frac{ \partial^2 ln(\xi)}{\partial \alpha_0^2} &= -1/2  \sum_{i=1}^n \psi_i^{-1} [(y_{i1}-\mu_i)^2 + (y_{i2}-\mu_i)^2]
	\end{align*}
	\begin{align*}
		\frac{ \partial ln(\xi)}{\partial \alpha_1} &=  1/2  \sum_{i=1}^n \psi_i^{-1} [(y_{i1}-\mu_i)^2 + (y_{i2}-\mu_i)^2] (x_i-\bar{x})\\
		\frac{ \partial^2 ln(\xi)}{\partial \alpha_1^2} &= -1/2  \sum_{i=1}^n \psi_i^{-1} [(y_{i1}-\mu_i)^2 + (y_{i2}-\mu_i)^2](x_i-\bar{x})^2
	\end{align*}
	\begin{align*}
		\frac{ \partial ln(\xi)}{\partial \beta_0} &=  \sum_{i=1}^n \psi_i^{-1} [(y_{i1}-\beta_0-\beta_1(x_i-\bar{x})) + (y_{i2}-\beta_0-\beta_1(x_i-\bar{x}))] \\
		\frac{ \partial^2 ln(\xi)}{\partial \beta_0^2} &= - 2\sum_{i=1}^n \psi_i^{-1}
	\end{align*}
	\begin{align*}
		\frac{ \partial ln(\xi)}{\partial \beta_1} &=  \sum_{i=1}^n \psi_i^{-1} [(y_{i1}-\beta_0-\beta_1(x_i-\bar{x})) + (y_{i2}-\beta_0-\beta_1(x_i-\bar{x}))] (x_i-\bar{x})\\
		\frac{ \partial^2 ln(\xi)}{\partial \beta_1^2} &= - 2\sum_{i=1}^n \psi_i^{-1}(x_i-\bar{x})^2
	\end{align*}
	Other derivatives
	\begin{align*}
		\frac{ \partial^2 ln(\xi)}{\partial \alpha_0\alpha_1} &= -1/2  \sum_{i=1}^n \psi_i^{-1} [(y_{i1}-\mu_i)^2 + (y_{i2}-\mu_i)^2](x_i-\bar{x})\\
		\frac{ \partial^2 ln(\xi)}{\partial \alpha_0\beta_0} &= -  \sum_{i=1}^n \psi_i^{-1} [(y_{i1}-\mu_i)^2 + (y_{i2}-\mu_i)^2](x_i-\bar{x})\\
		\frac{ \partial^2 ln(\xi)}{\partial \alpha_0\beta_1} &= -  \sum_{i=1}^n \psi_i^{-1} [(y_{i1}-\mu_i)^2 + (y_{i2}-\mu_i)^2](x_i-\bar{x})\\
		\frac{ \partial^2 ln(\xi)}{\partial \alpha_1\beta_0} &= -  \sum_{i=1}^n \psi_i^{-1} [(y_{i1}-\mu_i) + (y_{i2}-\mu_i)](x_i-\bar{x})\\
		\frac{ \partial^2 ln(\xi)}{\partial \alpha_1\beta_1} &= -  \sum_{i=1}^n \psi_i^{-1} [(y_{i1}-\mu_i) + (y_{i2}-\mu_i)](x_i-\bar{x})^2\\
		\frac{ \partial^2 ln(\xi)}{\partial \beta_0\beta_1} &= - 2 \sum_{i=1}^n \psi_i^{-1} (x_i-\bar{x})
	\end{align*}
	Taking expectation as $I(\xi) = -E (\partial^2 \xi)$
	\begin{align*}
		E(y_{i1}-\mu_i)^2 &= \psi_i,\qquad E(y_{i1}) = E(y_{i2}) =\mu_i,\qquad \sum_{i=1}^n x_i- n \bar{x} = 0\\
		E[\frac{ \partial^2 ln(\xi)}{\partial \alpha_0^2}] &=-1/2  \sum_{i=1}^n \psi_i^{-1} [E(y_{i1}-\mu_i)^2 + E(y_{i2}-\mu_i)^2]=  -n\\
		E[\frac{ \partial^2 ln(\xi)}{\partial \alpha_1^2}] &= -\sum_{i=1}^n (x_i-\bar{x})^2\\
		E[\frac{ \partial^2 ln(\xi)}{\partial \beta_0^2}] &= - 2\sum_{i=1}^n \psi_i^{-1}\\
		E[\frac{ \partial^2 ln(\xi)}{\partial \beta_1^2}] &= - 2\sum_{i=1}^n \psi_i^{-1}(x_i-\bar{x})^2\\
		E[\frac{ \partial^2 ln(\xi)}{\partial \alpha_0\alpha_1}] &= -1/2  \sum_{i=1}^n \psi_i^{-1} [E(y_{i1}-\mu_i)^2 + E(y_{i2}-\mu_i)^2]E(x_i-\bar{x}) = 0\\
		E[\frac{ \partial^2 ln(\xi)}{\partial \alpha_0\beta_0}] &= 0,\qquad
		E[\frac{ \partial^2 ln(\xi)}{\partial \alpha_0\beta_1}] =  0\\
		E[\frac{ \partial^2 ln(\xi)}{\partial \alpha_1\beta_0}] &=  0,\qquad
		E[\frac{ \partial^2 ln(\xi)}{\partial \alpha_1\beta_1}] =  0\\
		E[\frac{ \partial^2 ln(\xi)}{\partial \beta_0\beta_1}] &=  - 2 \sum_{i=1}^n \psi_i^{-1} (x_i-\bar{x}) 
	\end{align*}
	Then
	\begin{align*}
		I(\xi) &= -E (\partial^2 \xi)= \begin{bmatrix}
			n & 0&  0 &  0\\
			0 & \sum_{i=1}^n (x_i-\bar{x})^2 & 0  & 0 \\
			0 & 0&  2\sum_{i=1}^n \psi_i^{-1}  & 2 \sum_{i=1}^n \psi_i^{-1} (x_i-\bar{x}) \\
			0 &  0& 2 \sum_{i=1}^n \psi_i^{-1} (x_i-\bar{x})   & 2\sum_{i=1}^n \psi_i^{-1}(x_i-\bar{x})^2  \\
		\end{bmatrix}
	\end{align*} 
	Under null hypothesis, we have score test statistics follows a chi-square distribution
	\begin{align*}
		\frac{\partial ln}{\partial \Tilde{\xi}}^T I(\Tilde{\xi})^{-1} \frac{\partial ln}{\partial \Tilde{\xi}} & \sim \chi^2(1)
	\end{align*} 
	So we have $\Tilde{\psi} = exp(\Tilde{\alpha_0}) $, then $\Tilde{\alpha_0} = ln(\Tilde{\psi}) $.\\
	From part (a) which $\psi$ is constant, we have $\psi = \frac{1}{4n} \sum_{i=1}^n (y_{i1}- y_{i2})^2$ and then,
	\begin{align*}
		\hat{\mu_i} &= 1/2 (y_{i1}+ y_{i2})\\
		\hat{\psi} &=  \frac{1}{4n} \sum_{i=1}^n  (y_{i1}-y_{i2})^2
	\end{align*}
	then the score function under $\Tilde{\xi}$
	\begin{align*}
		\dot{l}(\xi) &= \begin{bmatrix}
			\partial_{\alpha_0} l(\xi) &= - n + 1/2  \sum_{i=1}^n \Tilde{\psi}^{-1} [(y_{i1}-\mu_i)^2 + (y_{i2}-\mu_i)^2] =0 \\
			\partial_{\alpha_1} l(\xi) &= 1/2  \sum_{i=1}^n \Tilde{\psi}^{-1} 1/2 (y_{i1}-y_{i2})^2 (x_i-\bar{x}) = \frac{1}{4 \Tilde{\psi}} \sum_{i=1}^n(y_{i1}-y_{i2})^2 (x_i-\bar{x})  \\
			\partial_{\beta_0} l(\xi) &=\sum_{i=1}^n \Tilde{\psi}^{-1} [(y_{i1}-\beta_0-\beta_1(x_i-\bar{x})) + (y_{i2}-\beta_0-\beta_1(x_i-\bar{x}))] =0  \\
			\partial_{\beta_1} l(\xi)& =\sum_{i=1}^n \Tilde{\psi}^{-1} [(y_{i1}-\beta_0-\beta_1(x_i-\bar{x})) + (y_{i2}-\beta_0-\beta_1(x_i-\bar{x}))] (x_i-\bar{x})=0 \\
		\end{bmatrix} \\
		&= \begin{bmatrix}
			0\\
			\frac{1}{4 \Tilde{\psi}} \sum_{i=1}^n(y_{i1}-y_{i2})^2 (x_i-\bar{x})   \\
			0 \\
			0 \\
		\end{bmatrix}
	\end{align*} 
	Under null hypothesis, $2 \sum_{i=1}^n \psi_i^{-1} (x_i-\bar{x}) = 0$, then 
	\begin{align*}
		I_n(\Tilde{\xi}) &= \begin{bmatrix}
			n & 0&  0 &  0\\
			0 & \sum_{i=1}^n (x_i-\bar{x})^2 & 0  & 0 \\
			0 & 0&  2\sum_{i=1}^n \Tilde{\psi}^{-1}  & 0\\
			0 &  0& 0  & 2\sum_{i=1}^n \Tilde{\psi}^{-1}(x_i-\bar{x})^2  \\
		\end{bmatrix}
	\end{align*}
	The score test statistics
	\begin{align*}
		SCn &= \frac{\partial ln}{\partial \Tilde{\xi}}^T I_n(\Tilde{\xi})^{-1} \frac{\partial ln}{\partial \Tilde{\xi}}  = (0,\frac{1}{4 \Tilde{\psi}} \sum_{i=1}^n(y_{i1}-y_{i2})^2 (x_i-\bar{x}), 0, 0 )\\
		& \begin{bmatrix}
			n & 0&  0 &  0\\
			0 & \sum_{i=1}^n (x_i-\bar{x})^2 & 0  & 0 \\
			0 & 0&  2\sum_{i=1}^n \Tilde{\psi}^{-1}  & 0\\
			0 &  0& 0  & 2\sum_{i=1}^n \Tilde{\psi}^{-1}(x_i-\bar{x})^2  \\
		\end{bmatrix}^{-1} \begin{bmatrix}
			0\\
			\frac{1}{4 \Tilde{\psi}} \sum_{i=1}^n(y_{i1}-y_{i2})^2 (x_i-\bar{x})   \\
			0 \\
			0 \\
		\end{bmatrix} \\
		&= \frac{\left[ \frac{1}{4 \Tilde{\psi}} \sum_{i=1}^n(y_{i1}-y_{i2})^2 (x_i-\bar{x}) \right]^2}{\sum_{i=1}^n (x_i-\bar{x})^2}
	\end{align*} 
	With $\Tilde{\psi} = \frac{1}{4n} \sum_{i=1}^n (y_{i1}- y_{i2})^2$, we have
	\begin{align*}
		SCn &= \frac{\left[n^2 \sum_{i=1}^n(y_{i1}-y_{i2})^2 (x_i-\bar{x}) \right]^2}{[\sum_{i=1}^n (y_{i1}- y_{i2})^2]^2 \sum_{i=1}^n (x_i-\bar{x})^2} \sim \chi^2(1)
	\end{align*} 
	We will reject the $H_0$ if $SCn > \chi^2(1, 1-\alpha)$.
\end{itemize}



\subsection{e}
Suppose that the vector $Y = (Y_0; Y_1; Y_2)^T$ follows a multinomial distribution with total count m and probability vector $(\gamma_0; \gamma_1; \gamma_2)^T$ with
\begin{align*}
	\gamma_j &= {2 \choose j} \pi^j (1-\pi)^{2-j} \theta^{-j(2-j)} /f(\pi, \theta), \qquad j= 0,1,2
\end{align*} 
where
\begin{align*}
	f(\pi, \theta) &= \sum_{k=0}^2 {2 \choose k} \pi^k (1-\pi)^{2-k} \theta^{-k(2-k)}
\end{align*} 
and $0 \leq \pi \leq 1, \theta >0$ are parameters. Furthermore, define $\lambda = log \frac{\pi}{1-\pi}$ and $\psi = log \theta$.

\begin{itemize}
	\item [(a)] Derive a sufficient statistic for $\lambda$ assuming  $\psi = \psi_0$ is known. Derive a conditional
	likelihood for $\psi$.\\
	Write the joint distribution of Y
	\begin{align*}
		P(Y) &= {m \choose y_0, y_1, y_2}  \gamma_1^{y_1} \gamma_2^{y_2} \gamma_0^{y_0} \\
		&= exp \left[ log {m \choose y_0, y_1, y_2} + y_0 log \gamma_0 + y_1 log\gamma_1 + y_2 log \gamma_2 \right]
	\end{align*}    
	\begin{align*}
		\gamma_0 &= {2 \choose 0} \pi^0 (1-\pi)^{2} \theta^{0} /f(\pi, \theta)= (1-\pi)^2/f(\pi, \theta)\\
		\gamma_1 &= {2 \choose 1} \pi^1 (1-\pi)^{1} \theta^{-1} /f(\pi, \theta)= 2\pi (1-\pi) \theta^{-1}/f(\pi, \theta)\\
		\gamma_2 &= {2 \choose 2} \pi^2 (1-\pi)^{0} \theta^{0} /f(\pi, \theta)= \pi^2/f(\pi, \theta)
	\end{align*} 
	\begin{align*}
		log P(Y) &=  log {m \choose y_0, y_1, y_2} + y_0 [2log (1-\pi) - log f(\pi,\theta)] \\
		& + y_1 [log 2 \pi (1-\pi) - log \theta - log f(\pi,\theta)]+ y_2 [2 log \pi - log f(\pi, \theta) ]\\
		f(\pi, \theta) &= {2 \choose 0} \pi^0 (1-\pi)^{2} \theta^{0} + {2 \choose 1} \pi^1 (1-\pi)^{1} \theta^{-1} + {2 \choose 2} \pi^2 (1-\pi)^{0} \theta^{0}\\
		log f(\pi, \theta) &=2log (1-\pi) +log 2 \pi (1-\pi) - log \theta + 2 log \pi \\
		log P(Y) &= log {m \choose y_0, y_1, y_2} + (2y_0 + y_1) log(1-\pi) \\
		& - (y_0+y_1+y_2) log f(\pi, \theta) + (y_1+ 2y_2) log \pi + y_1 log2 - y_1 log \theta\\
		m &= y_0 + y_1 + y_2, \qquad y_1 = m- y_0 - y_2\\
		log P(Y) &= log {m \choose y_0, y_1, y_2} + (m + y_0 - y_2) log(1-\pi) - mlog f(\pi, \theta)\\
		&+ (m-y_0+y_2)log \pi + y_1 log2 - y_1 log \theta\\
		&= log {m \choose y_0, y_1, y_2} + m log\left[ \frac{e^{\lambda}}{1+e^{\lambda}}  \frac{1}{1+e^{\lambda}} \frac{(1+e^{\lambda})^2}{1+ 2e^{\lambda-\psi} + e^{2\lambda}} \right] \\
		& - (y_0- y_2) \lambda + y_1 log 2 - y_1 \psi\\
	\end{align*} 
	If assume $\psi = \psi_0$ is known, then a sufficient statistics is $m, y_0-y_2$.
	\begin{align*}
		log P(Y)  &= log {m \choose y_0, y_1, y_2} + m log\left[ \frac{e^{\lambda}}{1+ 2e^{\lambda-\psi} + e^{2\lambda}} \right]
		- (y_0- y_2) \lambda + y_1 log 2 - y_1 \psi
	\end{align*}   
	Let $y_2-y_0 =t$, 
	\begin{align*}
		P(t)  &= \sum_{t} {m \choose y_0, y_1, y_2} \left[ \frac{e^{\lambda}}{1+ 2e^{\lambda-\psi} + e^{2\lambda}} \right]^m
		exp(\lambda t)  2^{y_1} exp(-\psi {y_1})\\
		P(y_1|t)  &=  \frac{P(t, Y)}{P(t)} = \frac{{m \choose y_0, y_1, y_2}  \left[ \frac{e^{\lambda}}{1+ 2e^{\lambda-\psi} + e^{2\lambda}} \right]^m exp(\lambda t)  2^{y_1} exp(-\psi {y_1})}{ \sum_{t} {m \choose y_0, y_1, y_2} \left[ \frac{e^{\lambda}}{1+ 2e^{\lambda-\psi} + e^{2\lambda}} \right]^m
			exp(\lambda t)  2^{y_1}  exp(-\psi {y_1})} \\
		&= \frac{\frac{1}{y_0!y_1!y_2!}2^{y_1}exp(-\psi {y_1}) }{\sum_{y'_2-y'_0=t} \frac{1}{y'_0!y'_1!y'_2!}2^{y'_1} exp(-\psi {y'_1})}
	\end{align*} 
	The conditional distribution for $\psi$
	\begin{align*}
		P(y_1, \psi |t)  &=  \frac{\frac{1}{y_0!y_1!y_2!}2^{y_1}exp(-\psi {y_1}) }{\sum_{y'_2-y'_0=t} \frac{1}{y'_0!y'_1!y'_2!}2^{y'_1} exp(-\psi {y'_1})}
	\end{align*} 
	
	\item[(b)] The data $y_0 = 3; y_1 = 0; y_2 = 2$ were observed. Based on the conditional likelihood
	of Part (a), compute the exact one-sided p-value for testing $H0 : \theta = 1$ against $H_0 : \theta > 1$ with $\lambda$ unspecified.\\
	The null hypothesis could be written as 
	\begin{align*}
		H_0  &: \psi = 0 \qquad vs. \qquad H_1: \psi \neq 0
	\end{align*} 
	From $y_0 = 3; y_1 = 0; y_2 = 2$, we have $t= y_2 - y_0 = -1, m=5$. There are possible 3 combinations that t=-1 as below\\
	\begin{tabular}{l l l l l}
		$y_1$ &  $y_2$ & $y_0$ & t & case\\\hline
		0 & 2  & 3 & -1 & 1\\
		2 & 1  & 2 & -1 & 2\\
		4 & 0  & 1 & -1 & 3\\
		\hline
	\end{tabular}\\
	So under $H_0$, the conditional probability for $y_1$ in the above 3 cases are
	\begin{align*}
		denominator &= \frac{1}{0!2!3!}2^{0} exp(-\psi {0}) + \frac{1}{1!2!2!}2^{2} exp(-\psi {2}) + \frac{1}{0!4!1!}2^{4} exp(-\psi {4}) \\
		&= 2/3 exp(-4\psi) + exp(-2\psi) + 1/12 = 21/12\\
		P(y_1=0, \psi |t=-1)  &=  \frac{\frac{1}{0!2!3!}2^{0}exp(0) }{\sum_{y'_2-y'_0=t} \frac{1}{y'_0!y'_1!y'_2!}2^{y'_1} exp(-\psi {y'_1})} = \frac{1/12}{21/12} = 1/21\\
		P(y_1=2, \psi |t=-1)  &=  \frac{\frac{1}{1!2!2!}2^{2}exp(0) }{\sum_{y'_2-y'_0=t} \frac{1}{y'_0!y'_1!y'_2!}2^{y'_1} exp(-\psi {y'_1})} = \frac{1/12}{21/12} = 12/21\\
		P(y_1=4, \psi |t=-1)  &=  \frac{\frac{1}{0!4!1!}2^{4}exp(0) }{\sum_{y'_2-y'_0=t} \frac{1}{y'_0!y'_1!y'_2!}2^{y'_1} exp(-\psi {y'_1})} = \frac{1/12}{21/12} = 8/21
	\end{align*} 
	We will reject $H_0$ if $P(y_1|t=-1) < 0.05$. Under the current sample, one sided test p-value for $P(y_1=0|t=-1) = 1/21 = 0.0476$, that $\psi \neq 0$.
\end{itemize}



\subsection{b}Consider the following
\begin{itemize}
	\item[(a)] For an arbitrary model, consider the conditional score statistic
	\begin{align*}
		U_{\psi}(\xi) &= \frac{\partial l_c(\xi, \psi_0)}{\partial \psi} |_{\psi_0=\psi}
	\end{align*} 
	Show that the conditional score statistic for any model can be written as
	\begin{align*}
		U_{\psi}(\xi) &= \partial_{\psi} log p(Y|\xi)- E[\partial_{\psi} log p(Y|\xi)|s_{\lambda}(\psi_0)]|_{\psi_0=\psi}
	\end{align*} 
	The conditional score statistic is the derivative of the conditional distribution
	\begin{align*}
		U_{\psi}(\xi) &= \frac{\partial l_c(\xi, \psi_0)}{\partial \psi} |_{\psi_0=\psi}\\
		p(\textbf{Y}| \xi) &= p(\textbf{Y}|s_{\lambda}(\psi_0), \xi) p(s_{\lambda}(\psi_0) | \xi), \qquad p(\textbf{Y}|s_{\lambda}(\psi_0), \xi) = \frac{p(\textbf{Y}| \xi)}{p(s_{\lambda}(\psi_0) | \xi)} \\
		l_c(\xi, \psi_0) &= log p(\textbf{Y}|s_{\lambda}(\psi_0), \xi)= log p(\textbf{Y}| \xi) - log p(s_{\lambda}(\psi_0) | \xi)
	\end{align*}
	Then we need to prove 
	\begin{align*}
		U_{\psi}(\xi) &= \frac{\partial l_c(\xi, \psi_0)}{\partial \psi} |_{\psi_0=\psi} = \partial_{\psi} log p(\textbf{Y}| \xi) - \partial_{\psi} log p(s_{\lambda}(\psi_0) | \xi)\\
		\partial_{\psi} log p(s_{\lambda}(\psi_0) | \xi) &= E[\partial_{\psi} log p(Y|\xi)|s_{\lambda}(\psi_0)]|_{\psi_0=\psi}
	\end{align*}
	We can write
	\begin{align*}
		log p(\textbf{Y}| \xi) &= log  p(\textbf{Y}|s_{\lambda}(\psi_0), \xi) + log p(s_{\lambda}(\psi_0) | \xi)\\
		E \left( \partial_{\psi}[log p(\textbf{Y}| \xi)| s_{\lambda}]\right) &= E \left(\partial_{\psi}[log  p(\textbf{Y}|s_{\lambda}(\psi_0), \xi)|s_{\lambda}]\right) + E \left(\partial_{\psi}[log p(s_{\lambda}(\psi_0), \xi)|s_{\lambda}]\right)
	\end{align*}    
	in which, the integral and expectation can switch, then we have
	\begin{align*}
		E \left(\partial_{\psi}[log  p(\textbf{Y}|s_{\lambda}(\psi_0), \xi)|s_{\lambda}]\right) & = \partial_{\psi} E \left([log  p(\textbf{Y}|s_{\lambda}(\psi_0), \xi)|s_{\lambda}]\right) = \partial_{\psi} E \left([log  p(\textbf{Y}| \xi)]\right)= 0
	\end{align*}      
	So,
	\begin{align*}
		E \left( \partial_{\psi}[log p(\textbf{Y}| \xi)| s_{\lambda}]\right) &= \partial_{\psi}log p(s_{\lambda}(\psi_0),\xi)
	\end{align*}
	Then we show
	\begin{align*}
		U_{\psi}(\xi) &= \partial_{\psi} log p(Y|\xi)- E[\partial_{\psi} log p(Y|\xi)|s_{\lambda}(\psi_0)]|_{\psi_0=\psi}
	\end{align*} 
	\item[(b)] Suppose that $y_1;.. y_n$ are independent and $y_i$ follows a Poisson distribution with mean $exp(\lambda_0 + \lambda_1x_{i1} +  \psi x_{i2})$, where $(x_{i1}; x_{i2})$ are covariates, $\lambda = (\lambda_0; \lambda_1)$ is the
	nuisance parameter vector and $\psi$  is the parameter of interest. Derive the conditional
	likelihood of $\psi$   and show that this conditional likelihood is free of $\lambda$.\\
	The joint distribution of $(y_1, · · · , y_n)$ is given by 
	\begin{align*}
		P(Y|\lambda, \psi)&=  exp \left( \sum_{i=1}^n y_i(\lambda_0 + \lambda_1x_{i1} +  \psi x_{i2}) - \sum_{i=1}^n exp(\lambda_0 + \lambda_1x_{i1} +  \psi x_{i2}) - log y_i! \right)
	\end{align*}
	Thus, $S_0 = \sum_{i=1}^n y_i$ is the sufficient and complete statistics for $\lambda_0$, and $S_1 = \sum_{i=1}^n y_i x_{i1}$ is the sufficient and complete statistics for $\lambda_1$.\\
	The conditional distribution of $\psi$ given $S_0, S_1$ is given by
	\begin{align*}
		p(\textbf{Y}, \psi|S=(S_0, S_1)) &= \frac{exp \left( \sum_{i=1}^n y_i(\lambda_0 + \lambda_1x_{i1} +  \psi x_{i2}) - \sum_{i=1}^n exp(\lambda_0 + \lambda_1x_{i1} +  \psi x_{i2}) - log y_i! \right)}{\sum_{y' \in S} exp \left( \sum_{i=1}^n y'_i(\lambda_0 + \lambda_1 x_{i1} +  \psi x_{i2}) - \sum_{i=1}^n exp(\lambda_0 + \lambda_1 x_{i1} +  \psi x_{i2}) - log y'_i! \right)}\\
		&= \frac{exp \left( S_1 \lambda_0 + S_2 \lambda_1 +  S_3 \psi) - \sum_{i=1}^n exp(\lambda_0 + \lambda_1x_{i1} +  \psi x_{i2}) - log y_i! \right)}{\sum_{y' \in S} exp \left( S'_1\lambda_0 + S'_2 \lambda_1 + S'_3 \psi) - \sum_{i=1}^n exp(\lambda_0 + \lambda_1 x_{i1} +  \psi x_{i2}) - log y'_i!\right)} \\
		&= \frac{exp \left( S_3 \psi  - log y_i!\right)}{\sum_{y' \in S} exp \left( S'_3 \psi - log y'_i! \right)}, \qquad S_3 = \sum_{i=1}^n y_i x_{i2}, S'_3 = \sum_{i=1}^n y'_i x_{i2}
	\end{align*}
	which is independent of $\lambda$. \\
	\item[(c)] Derive the conditional score statistic for part (b) and write out a Newton-Raphson algorithm for obtaining the conditional maximum likelihood estimate of $\psi$  based on $U_{\psi}(\xi)$.\\
	The log likelihood of the conditional distribution is
	\begin{align*}
		l_c(\psi) &= S_3 \psi  - log y_i! -log \left[ \sum_{y' \in S} exp \left( S'_3 \psi - log y'_i! \right) \right], \qquad S_3 = \sum_{i=1}^n y_i x_{i2}, S'_3 = \sum_{i=1}^n y'_i x_{i2}
	\end{align*} 
	The score function and observed fisher information is
	\begin{align*}
		U_{\psi}(\xi) &= \frac{\partial l_c(\xi, \psi_0)}{\partial \psi} |_{\psi_0=\psi}\\
		&= \psi - \frac{\sum_{y' \in S} S'_3 exp \left( S'_3 \psi - log y'_i! \right)}{\sum_{y' \in S} exp \left( S'_3 \psi - log y'_i! \right)}\\
		\frac{\partial^2 l_c(\xi, \psi_0)}{\partial \psi^2} &= \left[ \frac{\sum_{y' \in S} S'_3 exp \left( S'_3 \psi - log y'_i! \right)}{\sum_{y' \in S} exp \left( S'_3 \psi - log y'_i! \right)}\right]^2 - \frac{\sum_{y' \in S} S'^2_3 exp \left( S'_3 \psi - log y'_i! \right)}{\sum_{y' \in S} exp \left( S'_3 \psi - log y'_i! \right)}
	\end{align*}
	The newton-Raphson algorithm
	\begin{align*}
		\psi^{k+1} &= \psi^{k} - \left[\frac{\partial^2 l_c(\psi^{k})}{\partial \psi^2} \right]^{-1} U_{\psi}(\psi^{k})
	\end{align*}
	where $\frac{\partial^2 l_c(\psi^{k})}{\partial \psi^2}, U_{\psi}(\psi^{k})$ are from above equations.
	
	\item[(d)] Now suppose that we only have two random variables $y_1 \sim Poisson(\mu_1)$ and $y_2 \sim
	Poisson(\mu_2)$, where $y_1$ and $y_2$ are independent. We are interested in making inferences on the ratio $\psi = \mu_1/\mu_2$. Let $\xi = (\psi , \lambda)$, where $\lambda$ represents the nuisance parameter.
	\begin{itemize}
		\item [(i)] Show that the log-likelihood function of $\xi$ can be written as
		\begin{align*}
			l(\xi) &= (y_1 + y_2)\lambda + y_1 log (\psi) - exp(\lambda) (1+\psi)
		\end{align*}
		where $\lambda$ is a function of $\mu_2$. Explicitly state what $\lambda$ is.\\
		Write the joint distribution of $y_1, y_2$
		\begin{align*}
			P(y_1, y_2) &= \frac{\mu_1^{y_1} e^{-\mu_1}}{y_1!} \frac{\mu_2^{y_2} e^{-\mu_2}}{y_2!} \\
			log P(y_1, y_2) &= y_1 log \mu_1 - \mu_1 + y_2 \log \mu_2 - \mu_2 - log y_1! - log y_2!\\
			&= y_1 log \frac{\mu_1}{\mu_2} + y_1 log \mu_2 + y_2 log \mu_2 -\mu_1 - \mu_2 -log y_1! - log y_2!\\
			&= y_1 log \frac{\mu_1}{\mu_2} + (y_1+y_2) log \mu_2 - \mu_2(\mu_1/\mu_2 + 1) -log y_1! - log y_2!
		\end{align*}
		where 
		\begin{align*}
			\psi &=log \frac{\mu_1}{\mu_2} \\
			\lambda &= log \mu_2
		\end{align*}
		\item[(ii)] Derive the conditional likelihood of $\psi$  and write out a Newton-Raphson algorithm for obtaining the conditional maximum likelihood estimate of $\psi$ .\\
		From part (a), we see $y_1 + y_2$ is the sufficient statistics for $\lambda$, while $y_1 + y_2 \sim Poission (\mu_1+\mu_2)$ then we have conditional distribution of $\psi$ condition on $S = y_1 + y_2$.
		\begin{align*}
			Y(\psi|S= y_1+y_2,\lambda) &= \frac{exp \left[ y_1 \psi + (y_1+y_2) \lambda - exp(\lambda)(\psi + 1) -log y_1! - log y_2! \right] }{exp \left[ (y_1+y_2) log (\mu_1+\mu_2) - (\mu_1+\mu_2) -log (y_1+y_2)!  \right]}\\
			&= \frac{exp \left[ y_1 \psi + S \lambda - exp(\lambda)(\psi + 1) -log y_1! - log y_2! \right] }{exp \left[ S (\lambda + log(\psi + 1)) -  exp(\lambda)(\psi + 1) -log S!  \right]}\\
			&= \frac{exp \left[ y_1 \psi -log y_1! - log y_2! \right] }{exp \left[ (y_1+ S-y_1) log(\psi + 1)) -log S!  \right]}\\
			&= {S \choose y_1} \left( \frac{\psi}{1+\psi}\right)^{y_1} \left(\frac{1}{1+\psi} \right)^{S-y_1}
		\end{align*}
		The conditional distribution is a binomial, $B(S, \psi/(1+\psi))$.\\
		The score function and observed fisher information 
		\begin{align*}
			log Y(\psi|S,\lambda) &= y_1 log \psi -S log(1+\psi) + log {S \choose y_1} \\
			\partial_{\psi} log Y(\psi|S,\lambda) &= \frac{y_1}{\psi} - \frac{S}{1+\psi} = 0, \qquad \hat{\psi} = y_1/(S-y_1)\\
			\partial^2_{\psi} log Y(\psi|S,\lambda) &= -\frac{y_1}{\psi^2} + \frac{S}{(1+\psi)^2}
		\end{align*}
		The $CMLE = \hat{\psi} = y_1/(S-y_1)$. And the newton-Raphson equation 
		\begin{align*}
			\psi^{k+1} &= \psi^{k} - \left[\frac{\partial^2 l_c(\psi^{k})}{\partial \psi^2} \right]^{-1} U_{\psi}(\psi^{k})\\
			&= \psi^{k} - \left[ -\frac{y_1}{\psi^2} + \frac{S}{(1+\psi)^2}\right]^{-1} \left[\frac{y_1}{\psi} - \frac{S}{1+\psi} \right]|_{\psi = \psi^{k}}\\
			&=  \psi^{k} + \frac{y_1/\psi^{k} - S/(1+\psi^{k})}{y_1/{\psi^{k}}^2 - S/(1+\psi^{k})^2}
		\end{align*}
	\end{itemize}
\end{itemize}

\subsection{a}
Suppose that $y_1;... y_n$ are independent Bernoulli random variables, where $y_i  \sim Bernoulli(\pi)$, and we consider a logistic regression so that $logit(\pi) = x'_i\beta$, where $\beta = (\beta_1;... \beta_p)$. Our interest is inference on $(\beta_1; \beta_2)$, with all other parameters being treated as nuisance.
\begin{itemize}
	\item [(a)] Derive the conditional likelihood of $(\beta_1; \beta_2)$ and express it in the simplest possible form.\\
	The joint distribution of $y_1;... y_n$
	\begin{align*}
		p(Y) &= \prod_{i=0}^n p_i^{y_i} (1-p_i)^{(1-y_i)}\\
		log p(Y) &= \sum_{i=0}^n y_i log p_i + (1-y_i) log (1-p_i) = \sum_{i=0}^n y_i log \frac{p_i}{1-p_i}  + log (1-p_i) \\
		logit(pi) & = log \frac{p_i}{1-p_i} = x'_i\beta , \qquad p_i = \frac{exp(x'_i\beta )}{1+exp(x'_i\beta) } \\
		log p(Y) &= \sum_{i=0}^n y_i  x'_i\beta  - log (1+exp(x'_i\beta) ) \\
		&= \sum_{i=0}^n y_i  (x_{i1}\beta_1 + x_{i2}\beta_2 + x_{i3}\beta_3+.. x_{ip}\beta_p) - log (1+exp(x'_i\beta) ) 
	\end{align*}
	We can see that $\sum_{i=0}^n x_{i1}y_i$ is a sufficient and complete statistics for $\beta_1$. When only $(\beta_1; \beta_2)$ are the interest, and all other parameters being treated as nuisance. Then $s_j = \sum_{i=0}^n y_ix_{ij}$ is sufficient statistics for $\beta_j$. Let $S= (s_3, s_4,.. s_p)$
	
	\begin{align*}
		P(\beta_1, \beta_2| S)  &= \frac{exp \left[\sum_{i=0}^n (y_i  x_{i1})\beta_1 + (y_i  x_{i2})\beta_2 + .. (y_i  x_{ip})\beta_p - log (1+exp(x'_i\beta) ) \right]}{\sum_{t \in S} exp \left[ (t_i  x_{i1})\beta_1 + (t_i  x_{i2})\beta_2 +... (t_i  x_{ip})\beta_p - log (1+exp(x_{i}^T\beta ) \right]}  \\
		&= \frac{exp \left( \sum_{i=0}^n (y_i  x_{i1})\beta_1 + (y_i  x_{i2})\beta_2) \right)}{\sum_{t \in S} exp \left( (t_i  x_{i1})\beta_1 + (t_i  x_{i2})\beta_2)\right)}\\
		&= \frac{exp \left(S_1\beta_1 + S_2 \beta_2) \right)}{\sum_{S'} exp \left( S'_1\beta_1 + S'_2\beta_2)\right)}, \qquad S_j= \sum_{i=0}^n (y_i  x_{ij}), S'_j= \sum_{i=0}^n (t_i  x_{ij})
	\end{align*}
	
	\item[(b)] Derive the score equations for $(\beta_1; \beta_2)$ based on the conditional likelihood derived in part (a).\\
	The log conditional distribution is
	\begin{align*}
		l_c(\beta_1, \beta_2| S) &= log p(Y, \xi) - log p(s,\lambda, \psi_0) =log  P(\beta_1, \beta_2| S)\\
		l_c(\beta_1, \beta_2| S) &= log \frac{exp \left(S_1\beta_1 + S_2 \beta_2) \right)}{\sum_{S'} exp \left( S'_1\beta_1 + S'_2\beta_2)\right)} = S_1\beta_1 + S_2 \beta_2 - log \sum_{S'} exp \left( S'_1\beta_1 + S'_2\beta_2)\right)\\
		\frac{\partial l_c}{\partial \beta_1} &= S_1 - \frac{\sum_{S'} S'_1 exp \left( S'_1\beta_1 + S'_2\beta_2)\right)}{\sum_{S'} exp \left( S'_1\beta_1 + S'_2\beta_2)\right)} \\
		\frac{\partial l_c}{\partial \beta_2} &=S_2 - \frac{\sum_{S'} S'_2 exp \left( S'_1\beta_1 + S'_2\beta_2)\right)}{\sum_{S'} exp \left( S'_1\beta_1 + S'_2\beta_2)\right)} 
	\end{align*}  
	The score equations are setting the score function to 0
	\begin{align*}
		SCn = 0 &= \begin{bmatrix}
			S_1 - \frac{\sum_{S'} S'_1 exp \left( S'_1\beta_1 + S'_2\beta_2)\right)}{\sum_{S'} exp \left( S'_1\beta_1 + S'_2\beta_2)\right)}  \\
			S_2 - \frac{\sum_{S'} S'_2 exp \left( S'_1\beta_1 + S'_2\beta_2)\right)}{\sum_{S'} exp \left( S'_1\beta_1 + S'_2\beta_2)\right)}   \\
		\end{bmatrix} =\begin{bmatrix}
			0  \\
			0  \\
		\end{bmatrix}
	\end{align*}
	\item[(c)] Derive the asymptotic covariance matrix of the conditional maximum likelihood estimates of $(\beta_1; \beta_2)$.\\
	The Fisher information of $(\beta_1; \beta_2)$
	\begin{align*}
		\frac{\partial^2 l_c}{\partial \beta_1^2} &=  \left[\frac{\sum_{T} T_1 exp \left( T_1\beta_1 + T_2\beta_2\right)}{\sum_{T} exp \left( T_1\beta_1 + T_2\beta_2\right)} \right]^2 - \frac{\sum_{T} T_1^2 exp \left( T_1\beta_1 + T_2\beta_2\right)}{\sum_{T} exp \left( T_1\beta_1 + T_2\beta_2\right)}\\
		\frac{\partial^2 l_c}{\partial \beta_2^2} &= \left[\frac{\sum_{T} T_2 exp \left( T_1\beta_1 + T_2\beta_2\right)}{\sum_{T} exp \left( T_1\beta_1 + T_2\beta_2\right)} \right]^2 - \frac{\sum_{T} T_2^2 exp \left( T_1\beta_1 + T_2\beta_2\right)}{\sum_{T} exp \left( T_1\beta_1 + T_2\beta_2\right)}\\ 
		\frac{\partial^2 l_c}{\partial \beta_1 \beta_2} &=\frac{\left[ \sum_{T} T_1 exp \left( T_1\beta_1 + T_2\beta_2\right)\right] \left[ \sum_{T} T_2 exp \left( T_1\beta_1 + T_2\beta_2\right)\right]}{\left[ \sum_{T} exp \left( T_1\beta_1 + T_2\beta_2\right)\right]^2}  - \frac{\sum_{T} T_1 T_2 exp \left( T_1\beta_1 + T_2\beta_2\right)}{\sum_{T} exp \left( T_1\beta_1 + T_2\beta_2\right)}
	\end{align*}  
	Thus the asymptotic covariance matrix $Cov(\beta_1, \beta_2)$ is
	\begin{align*}
		Cov(\beta_1, \beta_2) &= I(\beta_1, \beta_2)^{-1}\\
		I(\beta_1, \beta_2) &= -E \left[ \frac{\partial^2 l_c}{\partial \beta^2} \right] =  -\lim_{n\to\infty} \frac{ I_n(\beta)}{n} \\
		I_n(\beta) &=- \begin{bmatrix}
			\frac{\partial^2 l_c}{\partial \beta_1^2}& \frac{\partial^2 l_c}{\partial \beta_1 \beta_2}\\
			\frac{\partial^2 l_c}{\partial \beta_1 \beta_2} &\frac{\partial^2 l_c}{\partial \beta_2^2}  \\
		\end{bmatrix}
	\end{align*}
	\item[(d)]Derive the conditional score test for testing $H_0: \beta_1= \beta_2 = 0$.\\
	\begin{align*}
		SCn &= \frac{\partial l_c}{\partial \Tilde{\beta}}^T I_n(\Tilde{\beta})^{-1} \frac{\partial l_c}{\partial \Tilde{\beta}} \sim \chi^2(1)
	\end{align*} 
	SCn is estimated under $H_0, \beta_1=\beta_2 = 0$. The SCn quadratic form is rank 1, so the degrees of freedom is 1.\\
	We will reject $H_0$ if $SCn > \chi^2(1, \alpha)$.
\end{itemize}





\chapter{Exponential Family}
\section{The Standard Exponential Distribution}

The standard exponential distribution family 

\begin{align*}
p(y| \theta) &= \phi \Big[ \exp \Big( y \theta - b(\theta) \Big) - c(y) \Big] - \frac{1}{2} s(y, \phi)
\end{align*}

We will explore the fun characteristics of the exponential family

\begin{itemize}
\item[(i)] Mean and Variance by derivatives

\begin{align*}
log  \int p(y| \theta) &=log  \int \phi \Big[ \exp \Big( y \theta - b(\theta) \Big) - c(y) \Big] - \frac{1}{2} s(y, \phi) dv = 0 \\
 log \int \exp \{( y \theta ) \} h(y) v(dy) &= b(\theta) \\
 \partial_{\theta} log \int \exp \{( y \theta ) \} h(y) v(dy) &= \partial_{\theta}  b(\theta) \\
\end{align*}

To proceed we need to move the gradient past the integral sign. In general derivatives can not be moved past integral signs (both are certain kinds of limits, and sequences of limits can differ depending on the order in which the limits are taken). However it turns out that the move is justified in this case by an appeal to the dominated convergence theorem. 

\begin{align*}
\partial_{\theta}  b(\theta) &= \partial_{\theta}  log \int \exp \{( y \theta ) \} h(y) v(dy)\\
 &=  \frac{\int y \exp \{( y \theta ) \} h(y) v(dy) }{\int \exp \{( y \theta ) \} h(y) v(dy)} \\
 &= \int y \exp \{ y \theta - b(\theta) \} h(x) v(dy) \\
 &= E[y] 
\end{align*}

Also we can see that the first derivative of $b(\theta)$ is equal to the mean of the sufficient statistics. Similar for the variance.

Another proof is to use the Bartlett's identities

Suppose that differentiation and integration are exchangeable and all the necessary expectations are finite. We have the following results:

\begin{align*}
E\_{\xi} \Big( \partial_j l_n \Big) &= 0,\\
E_{\xi} \Big( \partial^2_{j,k} l_n \Big) + E_{\xi} \Big( \partial_j l_n \partial_k l_n \Big) = 0 \\
\end{align*}

By the above two equations, we can get the expectation and variance. 


\end{itemize}



\section{The Bernoulli Distribution}

The standard exponential distribution family 

\begin{align*}
p(y| \theta) &= \phi \Big[ \exp \Big( y \theta - b(\theta) \Big) - c(y) \Big] - \frac{1}{2} s(y, \phi)
\end{align*}

For Bernoulli distribution,
\begin{align*}
p(x| \pi) &= \pi^{x} (1- \pi)^{1-x} \\
&= \exp \{ \log \Big( \frac{\pi}{1- \pi} \Big) x + \log (1 - \pi) \}
\end{align*}

We see that Bernoulli distribution is an exponential family distribution with 

\begin{align*}
\theta &= \log \Big( \frac{\pi}{1- \pi} \Big) \\
b(\theta)&=- \log (1 - \pi) =  \log \Big( 1 + \exp(\theta) \Big) x \\
\phi & = 1
\end{align*}

\subsection{Mean and Variance}

For a univariate random variable $Y$, in this case, all the $Y_i$ have the same $\pi$
\begin{align*}
\diffp{b(\theta)}{\theta} &= \frac{\exp(\theta)}{1 + \exp(\theta) } = \frac{1}{1 + \exp(-\theta)} = \mu = E(Y) \\
\diffp{b(\theta)}{\theta \theta}  &= \frac{\exp(\theta)}{\Big[ 1 + \exp(\theta) \Big]^2} = \mu(1-\mu) =Var(Y)
\end{align*}

In regression model, $logit (\pi) = X \beta$, which $\beta$ is a vector, then we will use the chain rule. And each individual $y_i$ has its own equation that $\pi_i$ is different.

\begin{align*}
\theta & = X \beta, \qquad \theta_i = x_i^{T} \beta \\
\partial_{\beta}{b(\theta_i)} &= \partial_{\theta_i}{b(\theta_i)} \partial_{\beta}{{\theta_i}} \\
&= \frac{\exp(\theta_i)}{1 + \exp(\theta_i) }  x_i= \frac{1}{1 + \exp(-\theta)} x_i= \mu_i x_i\\
\partial^2_{\beta}{{b(\theta_i)}} &= \frac{\exp(\theta_i)}{\Big[ 1 + \exp(\theta_i) \Big]^2} x_i^{\otimes 2}= \mu_i(1-\mu_i) x_i^{\otimes 2}
\end{align*}

And we will need to connect this with the Fisher Information or Newton-Raphson algorithm

\begin{align*}
\theta_i & = k \Big(x_i^{T} \beta \Big) = x_i^{T} \beta \\
\xi &= (\beta, \phi)\\
ln(\xi) &= \sum_{i=1}^n \phi \Big[ y_i k \Big(x_i^{T} \beta \Big) - b \Big( k \Big(x_i^{T} \beta \Big)  \Big) - c(y_i) \Big] - \frac{1}{2} s(y_i, \phi) \\
\dot{ln}(\beta) &= \diffp{ln(\beta) }{\beta} = \phi \sum_{i=1}^n \Big[ y_i - \dot{b} \Big( k \Big(x_i^{T} \beta \Big)  \Big)  \Big] \dot{k} \Big(x_i^{T} \beta \Big) x_i \\
&= \sum_{i=1}^n \Big[ y_i - \mu_i \Big] x_i \\
\ddot{ln}(\beta) &= \diffp{ln(\beta) }{\beta \beta} = -\phi \sum_{i=1}^n \ddot{b} \Big( k(x_i^T \beta) \Big) \dot{k}(x_i^T \beta)^2 x_i x_i^T + \phi \sum_{i=1}^n \Big[y_i - \dot{b}(k(x_i^T \beta)) \Big] \ddot{k}(x_i^T \beta) x_i x_i^T \\
&= -\sum_{i=1}^n \ddot{b} \Big(\theta_i \Big) x_i x_i^T = -\sum_{i=1}^n V(\theta_i) x_i x_i^T, \qquad \partial^2_{\beta}{{b(\theta_i)}} = V(\theta_i)
\end{align*}

let 
\begin{align*}
V(\theta) & = diag \{ V(\theta_i) \} , \qquad e_i = y_i - \mu_i\\
\sum_{i=1}^n V(\theta_i) x_i x_i^T &= X V(\theta) V^T\\
\mu_i &= \dot{b}(\theta_i), \qquad v_i = \ddot{b}(\theta_i)\\
\dot{\theta}_i &= \partial_{\beta} \theta_i = \dot{k}(x_i^T \beta) x_i, \qquad \ddot{\theta}_i = \partial^2_{\beta} \theta_i = \ddot{k}(x_i^T \beta) x_i x_i^T \\
\dot{b}(\theta_i) &= \partial_{\theta} b(\theta) \Big |_{\theta = \theta_i}, \dot{k}(\eta) = \partial_{\eta} k(\eta), \ddot{k}(\eta) = \partial^2_{\eta}(\eta)
\end{align*}

So
\begin{align*}
E \Big[ - \ddot{l}n(\beta) \Big] & = \phi \sum_{i=1}^n v_i \dot{\theta}_i^{\otimes 2}
\end{align*}

Another set is to use $E(y_i), Var(y_i)$ which is also used commonly as that are the information we generally get. It is used a lot in GEE. 
\begin{align*}
\partial_{\mu} \theta &= \partial_{\theta} \mu ^{-1}, \qquad \partial_{\mu} \mu = \partial_{\theta} \mu \partial_{\mu} \theta = 1\\
\partial_{\theta} \mu &= \partial_{\theta} b(\theta) = \ddot{b}(\theta) \\
\partial_{\mu} \theta &= \Big( \partial_{\theta} \mu \Big)^{-1} =  \ddot{b}(\theta)^{-1} \\
\end{align*}

Then we have the connection between the two system
\begin{align*}
\partial_{\beta} \theta &= \partial_{\beta} \mu_i \partial_{\mu_i} \theta_i = \partial_{\beta} \mu_i \Big[ \ddot{b}(\theta_i) \Big]^{-1} \\
\partial_{\beta}^2 \theta_i &= \Big( \partial^2_{\mu_i} \theta_i \Big) \Big( \partial_{\beta} \mu_i \Big)^{\otimes 2} + \partial_{\mu_i} \theta_i \Big( \partial_{\beta}^2 \mu_i \Big) \\
&= - \dddot{b}(\theta_i) \ddot{b}(\theta_i)^{-3} \Big( \partial_{\beta} \mu_i \Big)^{\otimes 2} + \Big[ \ddot{b}(\theta_i) \Big]^{-1} \Big( \partial^2_{\beta} \mu_i \Big)
\end{align*}

The generalized estimation model
\begin{align*}
V(\beta) &= \text{diag} \Big( v_1(\beta), …, v_n(\beta) \Big) \\
e(\beta) &= (y_1 - \mu_1(\beta), …, y_n- \mu_n(\beta))^{'} \\
D_{\theta} (\beta)^{'} &= \Big( \partial_{\beta} \beta_1(\beta),…,  \partial_{\beta} \beta_n(\beta)\Big)_{p \times n} \\
D (\beta)^{T} &= \Big( \partial_{\beta} \mu_1(\beta),…,  \partial_{\beta} \mu_n(\beta) \Big)_{p \times n} \\
\dot{l}_n(\beta) &= \phi D_{\theta}(\beta)^{T} e(\beta) = \phi D(\beta)^{'} V(\beta)^{-1} e(\beta) \\
E \Big[ -\ddot{l}_n(\beta) \Big] &= \phi D_{\theta}(\beta)^{'} V D_{\theta}(\beta) = \phi D(\beta)^{'} V(\beta)^{-1} D(\beta) 
\end{align*}




\chapter{Likelihood Functions}

\section{Conditional Distribution}

Conditional distribution is used in sufficient statistics (ie. show T(X) is sufficient which the distribution based on sufficient statistics does not depend on $\theta$), 
UMVUE $E[\theta | T(X)]$, nuisance parameter $p(\theta_1 | \theta_2,.. \theta_n, X)$, Bayesian statistics. 

Basically we can write the distribution of the based on statistics, if not, we will write the integral 



	\section{Multinomial distribution}
	Get the covariance matrix for cross-sectional, prospective, retrospective sampling method.\\
	
	\subsection{Likelihood for one random variable}
	To calculate the covariance matrix, we will use the MGF and take derivatives. Or use the cumulant function KGF to get the covariance.\\
	Use one random variable for the two way contingency table. While the Fisher information is the inverse of the covariance matrix, however we don't use Fisher information to calculate covariance matrix due to the math computation.\\
	For one random variable Y:
	\begin{align*}
		p(\theta) &= \prod_{i=1}^n \prod_{j=1}^J \pi_{j}^{I(Y_{i} = j)}, \qquad \theta = (\pi_1, \pi_2, ... \pi_J)'\\
		ln p(\theta) &= \sum_{i=1}^n \sum_{j=1}^J I(Y_{i}=j)log( \pi_{j}) = \sum_{j=1}^J n_j log(\pi_{j})\\
		M_X(t) &= E[exp(t^TX)] = E[exp(t^T(Y_1 + Y_2 +... Y_n))] = E[exp(t^TY_1 + t^TY_2 + ... t^TY_n)]\\
		&= E[\prod_{i=1}^n exp(t^TY_i)]\\
		&= \prod_{i=1}^n E[exp(t^TY_i)]  \qquad (\text{by independence})\\
		&= \prod_{i=1}^n M_{Y_i}(t) = \prod_{i=1}^n P(Y_i= 1) e^{ty_i}\qquad  \text{by MGF of discrete variable $Y_i$}\\
		&= \left( \sum_{j=1}^J \pi_j exp(t_j)\right)^n \qquad \text{by MGF of multinoulli}
	\end{align*}
	The MGF for bernoulli distribution
	\begin{align*}
		M_X(t) &= 1-p + p exp(t), \qquad K_X(t) = log (1-p + p exp(t))
	\end{align*}
	For multinomial distribution
	\begin{align*}
		M_X(t) &= (1-p + p exp(t))^n, \qquad K_X(t) = n log (1-p + p exp(t))\\
		E[n_j] &= n\pi_j, \qquad Var[n_j] = n\pi_j(1-\pi_j), \qquad Cov(n_j, n_k) = -n\pi_j\pi_k, {(j \neq k)}
	\end{align*}    
	Thus to compute covariance matrix
	\begin{align*}
		E(X_1 X_2) &= \frac{\partial^2 M_X(t)}{\partial t_i \partial t_j}|_{t_i = t_j = 0}\\
		&= \frac{\partial \left(n(\pi_ie^{t_i})(\sum_{k=1}^K \pi_ke^{t_k})^{n-1} \right)'}{\partial t_j}\\
		&= n(n-1)(\sum_{k=1}^K \pi_ke^{t_k})^{n-2}\pi_i\pi_j|_{t_i = t_j = 0} = n(n-1)\pi_i\pi_j\\
		E(X_i) &= n\pi_i\\
		Cov(X_i, X_j) &= E(X_i X_2) - E(X_i)E(X_j) = n(n-1)\pi_i\pi_j - n^2 \pi_i\pi_j = -n\pi_i\pi_j\\
		Var(X_i) &= E(X_i^2) - E(X_i)^2 \\
		E(X_i^2) &= \diffp{M(t)}{t t} = \frac{\partial \left(n(\pi_ie^{t_i})(\sum_{k=1}^K \pi_ke^{t_k})^{n-1} \right)'}{\partial t_i}\\
		&= n(\sum_{k=1}^K \pi_ke^{t_k})^{n-1}\pi_i e^{t_i}+ n(n-1)(\sum_{k=1}^K \pi_ke^{t_k})^{n-2}\pi_i\pi_i e^{2t_i}|_{t_i = 0} \\
		&= n\pi_i + n(n-1)\pi_i^2 = n\pi_i(1-\pi)\\
		Var(X_i/n) &= \frac{1}{n^2} Var(X_i) = \frac{1}{n}\pi_i(1-\pi_i)
	\end{align*}
	Thus the covariance matrix is
	\begin{align*}
		\Sigma &= \begin{bmatrix}
			\pi_1(1-\pi_1) &  -\pi_1\pi_2&  & -\pi_i\pi_j \\
			-\pi_j\pi_i&  \pi_i(1-\pi_i)&   &  \\
			..& ..&..&..
		\end{bmatrix}\\
		&= diag{(\pi_j) - \theta \theta^T}
	\end{align*}
	Here is the question, why do we think the covariance matrix of $X$ is the covariance matrix of $\pi$?
	\begin{align*}
		n^{-1} (n_1, n_2, ..n_I) &= n^{-1} \sum_{i=1}^n[ 1 (X_{i}=1), 1 (X_{i}=2), ..1 (X_{i}=I)] \\
		&= E[1 (X_{i}=1), 1 (X_{i}=2), ..1 (X_{i}=I) ] = [\pi_1, \pi_2, .. \pi_I] 
	\end{align*}


	\subsection{Pearson Statistics}
	Question: why the Pearson Statistics use the square of difference between sample mean and expected mean, then divided by the expected mean? \\
	
	We need to know what is the distribution of the Pearson Statistics. First, we start from the asymptotic distribution of the sample percentage $\hat{\pi} = \frac{n_i}{n}$.
	\begin{align*}
		\sqrt{n} (\frac{n_1}{n} - \pi_1, \frac{n_2}{n} - \pi_2, ..\frac{n_I}{n}-\pi_I) & \xrightarrow{L} N(0, \Sigma^{\ast})\\
		\Sigma^{\ast} &= diag\{ \pi\} - \pi \pi^T
	\end{align*}
We need to pay attention that, the $\pi_1, \pi_2, .. \pi_I$ are joint distributed. The Pearson statistics comes from a function of $(\frac{n_1}{n} - \pi_1, \frac{n_2}{n} - \pi_2, ..\frac{n_I}{n}-\pi_I)$, which could use delta method. The normal distribution is always associated with chi-square distribution. \\
	\begin{align*}
		\Gamma &= diag\{ \pi_1, \pi_2,... \pi_I \} \\
		\sqrt{n} \Gamma^{-1/2} \left(\frac{n_1}{n} - \pi_1, \frac{n_2}{n} - \pi_2, ..\frac{n_I}{n}-\pi_I \right) & \xrightarrow{L} N(0, \Gamma^{-1/2} \Sigma^{\ast} \Gamma^{-1/2})
	\end{align*}
	
	Because $\Gamma$ is a diagonal matrix, so it could be multiplied directly to the left or right of a matrix, and it only works on the diagonal element. \\
	\begin{align*}
		\Gamma^{-1/2} \Sigma^{\ast} \Gamma^{-1/2} &= \Gamma^{-1/2} \Gamma^{1/2} (I - \sqrt{\pi}^{\otimes 2}) \left( \Gamma^{-1/2} \Gamma^{1/2} \right)^T\\
		tr(I - \sqrt{\pi}^{\otimes 2}) & = I-1 \\
		tr(\Gamma^{-1/2} \Sigma^{\ast} \Gamma^{-1/2}) &= tr( \Sigma^{\ast} \Gamma^{-1/2} \Gamma^{-1/2}) = tr( \Sigma^{\ast} \Gamma^{-1}) \\
		&= tr( [\Gamma - \pi \pi^T] \Gamma^{-1}) = tr(\Gamma\Gamma^{-1}) - tr(\pi \pi^T \Gamma^{-1}) = I-1
	\end{align*}
	The Pearson Chi-square statistic is defined as
	\begin{align*}
		\chi^2 &= n \sum_{j=1}^I (\frac{n_j}{n} - \pi_j)^2/\pi_j = \left[ \sqrt{n} \Gamma^{-1/2} \left(\frac{n_1}{n} - \pi_1, \frac{n_2}{n} - \pi_2, ..\frac{n_I}{n}-\pi_I \right) \right]^{\otimes 2}
	\end{align*}
	which converge to $\chi^2(I-1)$ as $n \rightarrow \infty$.

\subsection{Odds ratio}
	The covariance of odds ratio by delta method. We simplify $2 \times 2$ table as $\pi_{11} = \pi_1, \pi_{12} = \pi_2, \pi_{21} = \pi_3, \pi_{22} = \pi_4$.
	\begin{align*}
		g(\pi) &= \frac{\pi_{22}\pi_{11}}{\pi_{12}\pi_{21}} \qquad \pi=(\pi_{11}, \pi_{12}, \pi_{21}, \pi_{22})\\
		\sqrt{n} \left( g(\hat{\pi}) - g({\pi}) \right) & \xrightarrow[]{d} N \left(0, \diffp*{g(\pi)}{\pi}{} \Sigma \diffp*{g(\pi)}{\pi}{}^T \right)\\
		\diffp{g(\pi)}{\pi}  &= \left( \frac{\partial g}{\partial \pi_{11}}, \frac{\partial g}{\pi_{12}}, \frac{\partial g}{\partial \pi_{21}}, \frac{\partial g}{\partial \pi_{22}} \right)^T\\
		& = \left( \frac{\pi_{22}}{\pi_{21}\pi_{12}}, \frac{-\pi_{11}\pi_{22}}{\pi_{21}\pi_{12}^2}, \frac{-\pi_{11}\pi_{22}}{\pi_{12}\pi_{21}^2}, \frac{\pi_{11}}{\pi_{21}\pi_{12}} \right)^T\\
		\Sigma^{\ast} &= g(\pi)^2(\frac{1}{\pi_{11}} + \frac{1}{\pi_{12}} + \frac{1}{\pi_{21}} + \frac{1}{\pi_{22}})
	\end{align*} 
	So that,
	\begin{align*}
		Var(\hat R) &=  \frac{1}{n} \Sigma^{\ast} 
	\end{align*} 
	We consider $log \hat R$ instead of $\hat R$, because $log \hat R$ converges rapidly to a normal distribution compared to $\hat R$.
	\begin{align*}
		log(\hat{R}) &= log \pi_1 + \log \pi_2 - \log \pi_3  \log \pi_4\\
		\diffp{g(\pi)}{\pi}  &= \left(\frac{1}{\pi_{11}} , -\frac{1}{\pi_{12}}, -\frac{1}{\pi_{21}}, \frac{1}{\pi_{22}} \right)^T\\
		Var(log(\hat{R})) &= \frac{1}{n} \Tilde{\Sigma} \\
		\Tilde{\Sigma} &= \diffp*{g(\pi)}{\pi}{}^T \Sigma \diffp*{g(\pi)}{\pi}{}\\
		log(\hat R) &=  \frac{1}{n}\left( \frac{1}{\hat \pi_{11}} + \frac{1}{\hat \pi_{12}} + \frac{1}{\hat \pi_{21}} + \frac{1}{\hat \pi_{22}} \right)\\
		s.e. log(\hat R) &=  \frac{1}{\sqrt{n}} \sqrt{\frac{1}{\hat \pi_{11}} + \frac{1}{\hat \pi_{12}} + \frac{1}{\hat \pi_{21}} + \frac{1}{\hat \pi_{22}}} 
	\end{align*} 
	
	




\section{Contingency Table}

We can use either multinomial distribution, poisson distribution to model the contingency table.

Consider a $I \times J$ contingency table of cell counts, where each cell count is denoted by $n_{ij}, i=1,..I, j=1,..J$, and thus $n_{ij}$ denotes the cell count of ith row and jth column, and $n_{ij} \sim Poisson (\mu_{ij})$ and independent. Further, let $n= \sum_{j=1}^J \sum_{i=1}^I n_{ij}$ denote the grand total.

\begin{itemize}

	\item [(a)] Derive the joint distribution of $(n_{11}, n_{12},... n_{ij})$ conditional on grand total n.
	
	$n_{ij} \sim Poisson(\mu_{ij})$, $n_{ij}$ are independent, $i=1,..I, j=1,..J$.
	
	By poisson distribution of each cell counts, $n= \sum_{i=1}^I \sum_{j=1}^J n_{ij} \sim Poisson(\sum_{i=1}^I \sum_{j=1}^J \mu_{ij})$
	\begin{align*}
		n &= \sum_{i=1}^I \sum_{j=1}^J n_{ij} \sim \frac{\exp(-\mu) \mu^n }{n!}, \qquad \mu= \sum_{i=1}^I \sum_{j=1}^J \mu_{ij}
	\end{align*}	
	
	Then the likelihood function
	\begin{align*}
	p(n_{11},..n_{ij}| n) &= \frac{P(n_{11}, …, n_{IJ}, \sum_{i=1}^I \sum_{j=1}^J n_{ij} = n)}{P(\sum_{i=1}^I \sum_{j=1}^J n_{ij} = n)} \\
		&= \frac{\prod_{i=1}^I \prod_{j=1}^J \frac{\exp(-\mu_{ij})  {\mu_{ij}}^{n_{ij}}}{n_{ij}!}}{\frac{exp(-\mu) \mu^n }{n!}} \\
		&= {n \choose n_{11} n_{12} ... n_{ij}} \frac{\prod_{i=1}^I \prod_{j=1}^J {\mu_{ij}}^{n_{ij}}}{\mu^n } \\
	&= {n \choose n_{11} n_{12} ... n_{ij}} \prod_{i=1}^I \prod_{j=1}^J \left( \frac{\mu_{ij}}{\mu } \right)^{n_{ij}}, \qquad \pi_{ij} = \frac{\mu_{ij}}{\sum_{i=1}^I \sum_{j=1}^J \mu_{ij}}
	\end{align*}
		
The joint distribution is Multinomial ($n, \pi_{11}, \pi_{12},.. \pi_{IJ}$), where 

	\item [(b)] Suppose all of the rows margins are assumed fixed. Derive the joint distribution of $(n_{11}, n_{12},... n_{ij})$.
	
	$n_{i .} = \sum_{j=1}^I n_{ij} \sim Poisson(\sum_{j=1}^J \mu_{ij}), i= 1,..I$. The conditional distribution will be built based on fixed row margins (the denominator will be the supposed known). 
	
\begin{align*}
	n_{i+} &= \sum_{j=1}^J n_{ij}\\
	n_{i+} & \sim Poisson (\sum_{j=1}^J \mu_{ij})\\
	p(n_{11},..n_{ij}|n_{i+}) &= \frac{P(n_{11}) P(n_{12}).. P(n_{IJ})}{P(n_{1.}) P(n_{2.})… P(n_{I.})} \\
	&= \prod_{i=1}^I \prod_{j=1}^J \frac{\exp(-\mu_{ij})  {\mu_{ij}}^{n_{ij}}}{n_{ij}!} \Bigg{/} \prod_{i=1}^I \frac{\exp(-\mu_i) \mu_i^{n_{i+}}}{n_{i+}!}\\
	&= \prod_{i=1}^I {n_{i+} \choose n_{ij}} \prod_{i=1}^I \prod_{j=1}^J \left( \frac{\mu_{ij}}{\sum_{j=1}^J \mu_{ij}} \right)^{n_{ij}}
\end{align*}

	\item [(c)] Suppose all of the columns margins are assumed fixed. Derive the joint distribution of $(n_{11}, n_{12},... n_{ij})$.
\begin{align*}
	n_{+j} &= \sum_{i=1}^I n_{ij}\\
	n_{+j} & \sim Poisson (\sum_{i=1}^I \mu_{ij})\\
	p(n_{11},..n_{ij}|n_{+j}) &= \prod_{i=1}^I \prod_{j=1}^J \frac{exp(-\mu_{ij})  {\mu_{ij}}^{n_{ij}}}{n_{ij}!} \Bigg{/} \prod_{j=1}^J \frac{exp(-\mu_i) \mu_i^{n_{+j}}}{n_{+j}!}\\
	&= \prod_{j=1}^J {n_{+j} \choose n_{ij}} \prod_{i=1}^I \prod_{j=1}^J \left( \frac{\mu_{ij}}{\sum_{i=1}^I \mu_{ij}} \right)^{n_{ij}}
\end{align*}


Consider a $I \times J$ contingency table of cell counts, where each cell count is denoted by $n_{ij}, i=1,..I, j=1,..J$, and thus $n_{ij}$ denotes the cell count of ith row and jth column, and $n_{ij} \sim Poisson (\mu_{ij})$ and independent. Further, let $n= \sum_{j=1}^J \sum_{i=1}^I n_{ij}$ denote the grand total.

\item[(d)] Conditional distribution of $n_{11}$

	Need to see that the variable transformation in this case. One skill I need to develop is to construct the probability distribution or likelihood function for each scenario.
	
	Suppose that $I=2$ and $J=2$, and both the rows margins and column margins are fixed. Derive the joint distribution of $(n_{11}|n_{1+}, n_{+1} n)$, where $n_{1+} = n_{11} + n_{12}, n_{+1} = n_{11}+ n_{21}$.
	
\begin{align*}
	p(n_{11} | n_{1+}, n_{+1} n) &= \frac{p(n_{11}, n_{1+}, n_{+1} n)}{p(n_{1+}, n_{+1} n)}\\
		p(n_{ij}) &= \prod_{i=1}^2 \prod_{j=1}^2 \frac{\exp(-\mu_{ij}) \mu_{ij}^{n_{ij}}}{n_{ij}!} \\
		&= \frac{\exp(-\mu_{11})\mu_{11}^{n_{11}} }{n_{11}!} \frac{\exp(-\mu_{12})\mu_{12}^{n_{12}}}{n_{12}!} \frac{\exp(-\mu_{21})\mu_{21}^{n_{21}}}{n_{21}!} \frac{\exp(-\mu_{22})\mu_{22}^{n_{22}}}{n_{22}!}\\
		n_{12} &= n_{1+} - n_{11}, \qquad n_{21} = n_{+1} - n_{11}, \\ n_{22} &= n - n_{12} - n_{21} - n_{11} = n- n_{1+} - n_{+1} + n_{11}\\
		p(n_{11}, n_{1+}, n_{+1} n) &= \frac{\exp(-\mu_{11})\mu_{11}^{n_{11}} }{n_{11}!} \frac{\exp(-\mu_{12})\mu_{12}^{n_{1+} - n_{11}}}{(n_{1+} - n_{11})!} \frac{\exp(-\mu_{21})\mu_{21}^{n_{+1} - n_{11}}}{(n_{+1} - n_{11})!} \frac{\exp(-\mu_{22})\mu_{22}^{n- n_{1+} - n_{+1} + n_{11}}}{(n- n_{1+} - n_{+1} + n_{11})!}
\end{align*}	
The Jacobian transformation matrix 
\begin{align*}
	J &=  \begin{pmatrix}
	\diffp{{n_{11}}}{{n_{11}}} & \diffp{{n_{11}}}{{n_{1+}}} & \diffp{{n_{11}}}{{n_{+1}}} & \diffp{{n_{11}}}{{n}}\\
	\diffp{{n_{12}}}{{n_{11}}} & \diffp{{n_{12}}}{{n_{1+}}} & \diffp{{n_{21}}}{{n_{+1}}} & \diffp{{n_{22}}}{{n}}\\
	\diffp{{n_{21}}}{{n_{11}}} & \diffp{{n_{21}}}{{n_{1+}}} & \diffp{{n_{21}}}{{n_{+1}}} & \diffp{{n_{22}}}{{n}}\\
	\diffp{{n_{22}}}{{n_{11}}} & \diffp{{n_{22}}}{{n_{1+}}} & \diffp{{n_{22}}}{{n_{+1}}} & \diffp{{n_{22}}}{{n}} \\
\end{pmatrix}= \begin{pmatrix}
1 & 0 & 0 & 0\\
-1 & 1 & 0 & 0\\
-1 & 0 & 1 & 0\\
1 & -1 & -1 & 1\\
\end{pmatrix}\\
\lVert J \rVert &= 1
\end{align*}
Then we can get the $p(n_{1+}, n_{+1}, n)$ by summing over $n_{11}$. We have $n_{11} <= n_{1+}, n_{11} <= n_{+1}$, and $n_{11} >= -n + n_{1+} + n_{+1}$. 		
\begin{align*}
	p(n_{11}, n_{1+}, n_{+1} n) &= \frac{exp(-\mu_{11})\mu_{11}^{n_{11}} }{n_{11}!} \frac{exp(-\mu_{12})\mu_{12}^{n_{1+} - n_{11}}}{(n_{1+} - n_{11})!} \frac{exp(-\mu_{21})\mu_{21}^{n_{+1} - n_{11}}}{(n_{+1} - n_{11})!} \frac{exp(-\mu_{22})\mu_{22}^{n- n_{1+} - n_{+1} + n_{11}}}{(n- n_{1+} - n_{+1} + n_{11})!}\\
	&= \frac{exp(-\sum_{i=1}^2 \sum_{j=1}^2 \mu_{ij}) \left( \frac{\mu_{11} \mu_{22}}{\mu_{12} \mu_{21}}\right) ^{n_{11}} \left(\frac{\mu_{12}}{\mu_{22}} \right)^{n_{1+}} \left(\frac{\mu_{21}}{\mu_{22}} \right)^{n_{+1}} \mu_{22}^{n}} {n_{11}! (n_{1+} - n_{11})! (n_{+1} - n_{11})! (n- n_{1+} - n_{+1} + n_{11})!}\\
	p(n_{1+}, n_{+1} n) &= \sum_{ \max{(0, -n + n_{1+} + n_{+1})}}^{\min{(n_{1+}, n_{+1})}} \frac{exp(-\sum_{i=1}^2 \sum_{j=1}^2 \mu_{ij}) \left( \frac{\mu_{11} \mu_{22}}{\mu_{12} \mu_{21}}\right) ^{n_{11}} \left(\frac{\mu_{12}}{\mu_{22}} \right)^{n_{1+}} \left(\frac{\mu_{21}}{\mu_{22}} \right)^{n_{+1}} \mu_{22}^{n}} {n_{11}! (n_{1+} - n_{11})! (n_{+1} - n_{11})! (n- n_{1+} - n_{+1} + n_{11})!}
\end{align*}
So we can have 
\begin{align*}
	p(n_{11}|n_{1+}, n_{+1} n) &= \frac{p(n_{11}, n_{1+}, n_{+1} n)}{p(n_{1+}, n_{+1} n)}\\
	 &= \frac{exp(-\sum_{i=1}^2 \sum_{j=1}^2 \mu_{ij}) \left( \frac{\mu_{11} \mu_{22}}{\mu_{12} \mu_{21}}\right) ^{n_{11}} \left(\frac{\mu_{12}}{\mu_{22}} \right)^{n_{1+}} \left(\frac{\mu_{21}}{\mu_{22}} \right)^{n_{+1}} \mu_{22}^{n}} {n_{11}! (n_{1+} - n_{11})! (n_{+1} - n_{11})! (n- n_{1+} - n_{+1} + n_{11})!} \\
	 & \Bigg{/} \sum_{ \max{(0, -n + n_{1+} + n_{+1})}}^{\min{(n_{1+}, n_{+1})}} \frac{exp(-\sum_{i=1}^2 \sum_{j=1}^2 \mu_{ij}) \left( \frac{\mu_{11} \mu_{22}}{\mu_{12} \mu_{21}}\right) ^{n_{11}} \left(\frac{\mu_{12}}{\mu_{22}} \right)^{n_{1+}} \left(\frac{\mu_{21}}{\mu_{22}} \right)^{n_{+1}} \mu_{22}^{n}} {n_{11}! (n_{1+} - n_{11})! (n_{+1} - n_{11})! (n- n_{1+} - n_{+1} + n_{11})!}
\end{align*}	
Which we can rewrite 
\begin{align*}
	p(n_{11}|n_{1+}, n_{+1} n) &= {n_{1+} \choose n_{11}} {n - n_{1+} \choose n_{+1}-n_{11}} \left( \frac{\pi_{11} \pi_{22}}{\pi_{12} \pi_{21}} \right)^{n_{11}}\\
	& \Bigg{/}  \sum_{x \in \max{(0, -n + n_{1+} + n_{+1})}}^{\min{(n_{1+}, n_{+1})}} {n_{1+} \choose x} {n - n_{1+} \choose n_{+1}-x} \left( \frac{\pi_{11} \pi_{22}}{\pi_{12} \pi_{21}}\right) ^x
\end{align*}

Thus $n_{11} | n_{1,}, n_{.1}, n$ is a non-central hypergeometric distribution.

\item[(e)] Let $\pi_{ij}$ denote the cell probability for the ith row and jth column of the table and assume that n is fixed. Consider testing the hypothesis $H_0: \pi_{ij} = \pi_{i.} \pi_{.j}, i=1,..,I, j= 1,..,J$. Derive the MLE's 
of $\pi_{ij}$ under $H_0$.

We need to derive the likelihood function of cell probability. Under the $H_0$, get the MLE of $\pi_{ij} = \pi_{i.} \pi_{.j}$. We could use the likelihood function to get the $\pi_{i.}, \pi_{.j}$ MLE estimates, then by the invariance of MLE, $\hat{p_{ij}} = \hat{\pi_{i.}} \hat{\pi_{.j}}$.

From part (a) we have the likelihood function of $\pi_{ij}$

\begin{align*}
	p(n_{11},..n_{ij}| n) &	= {n \choose n_{11} n_{12} ... n_{ij}} \prod_{i=1}^I \prod_{j=1}^J \left( \pi_{ij} \right)^{n_{ij}} \\
	\sum_{i=i}^I \sum_{j=1}^J \pi_{ij} &= 1\\
\end{align*}

Let $\pi_{ij}$ denote the cell probability and assume n is fixed. Consider testing $H_0: \pi_{ij} = \pi_{i+} \pi_{+j}, i=1,..I, j=1,..J$. Derive the MLE of $\pi_{ij}$ under $H_0$.

The $H_0$ could be written as 
\begin{align*}
	H_0 &: \pi_{ij} = \pi_{i+} \pi_{+j}
\end{align*}

The multinomial distribution of $\pi_{ij}$
\begin{align*}
	p(\pi_{ij}) &= {n \choose n_{11} n_{12} n_{21} n_{22}} \pi_{ij}^{n_{ij}} , \sum_{i=1}^I \sum_{j=1}^J \pi_{ij} = 1
\end{align*}
The log-likelihood function
\begin{align*}
	log p(\pi_{ij}) &= \log {n \choose n_{11} ,.., n_{IJ}} +  n_{ij} log \pi_{ij} , \sum_{i=1}^I \sum_{j=1}^J \pi_{ij} = 1
\end{align*}
Under $H_0$, the log-likelihood
\begin{align*}
	log p(\pi_{ij}) &= \log {n \choose n_{11} ,.., n_{IJ}}  +  n_{ij} log \pi_{i+} \pi_{+j} , \sum_{i=1}^I \pi_{i+} = 1, \sum_{j=1}^J \pi_{+j} = 1 
\end{align*}
By Lagrangian multiplier theorem,
\begin{align*}
	ln(\pi_{ij}) &=n \log {n \choose n_{11} ,.., n_{IJ}}  +\sum_{i=1}^I \sum_{j=1}^J n_{ij} log \pi_{i+} \pi_{+j} + \lambda ( \sum_{i=1}^I \sum_{j=1}^J \pi_{ij} - 1),\\
	&= n \log {n \choose n_{11} ,.., n_{IJ}}  +\sum_{i=1}^I \sum_{j=1}^J n_{ij} log \pi_{i+} + \sum_{j=1}^J \sum_{i=1}^I n_{ij} log \pi_{+j} - \lambda ( \sum_{i=1}^I \pi_{i+} - 1)
\end{align*}
Take first derivative of log-likelihood
\begin{align*}
	\diffp{ln}{{\pi_{i+}}} &= \frac{\sum_{j=1}^J n_{ij}}{\pi_{i+}} + \lambda = 0 \\
	\hat{\pi}_{i+} &= \frac{\sum_{j=1}^J n_{ij}}{\lambda}\\
	\sum_{i=1}^I \pi_{i+} &= 1, \qquad \lambda = \sum_{j=1}^J \sum_{i=1}^I n_{ij}\\
	\hat{\pi}_{i+} &= \frac{n_{i+}}{n}
\end{align*}
Similarly, we have $\hat{\pi}_{+j} = \frac{n_{+j}}{n}$, the MLE of $\pi_{ij}$ under $H_0$ is 
\begin{align*}
	\hat{\pi}_{ij} &= \hat{\pi}_{i+} \hat{\pi}_{+j} = \frac{n_{i+} n_{+j}}{n^2}
\end{align*}

Using Lagrange multiplier $\lambda$, the log-likelihood of $\pi_{1.},.., \pi_{I.}$ is

\begin{align*}
	l(\pi_{1.},..\pi_{I.}| n) &= \sum_{i=1}^I \sum_{j=1}^J {n_{ij}} \log \left( \pi_{i.} \pi_{.j} \right) + \lambda \Big( 1- \sum_{i=1}^I \pi_{i.} \Big)\\
	\dot{l} (\pi_{i.}) &= \frac{\sum_{j=1}^J n_{ij}}{\pi_{i.}} - \lambda \underset{set}{=} 0\\
	\hat{\pi_{i.}} &= \frac{\sum_{j=1}^J n_{ij}}{\lambda}, \qquad \lambda = \sum_{i=1}^I \sum_{j=1}^J {n_{ij}} = n \\
	\hat{\pi}_{i.} &= \frac{n_{i+}}{n} 
\end{align*}


\item[(f)] Derive the likelihood ratio test for the hypothesis in part (e) and derive its asymptotic distribution under $H_0$.

From part (e), we have the parameter estimates under $H_0$. While under alternative hypothesis, we have $\mu_{ij} = n_{ij}$. 
\begin{align*}
	LRT_n &= 2(LR(\pi_{H_1}) - LR(\pi_{H_0})) =2\left( \sum_{i=1}^I \sum_{j=1}^J n_{ij} log \pi_{ij} - \sum_{i=1}^I \sum_{j=1}^J n_{ij} log \pi_{i+} \pi_{+j} \right)\\
	&= 2\left( \sum_{i=1}^I \sum_{j=1}^J n_{ij} log \frac{\pi_{ij}}{\pi_{i+} \pi_{+j} }   \right)\\
	&= 2\left( \sum_{i=1}^I \sum_{j=1}^J n_{ij} log \frac{n_{ij} n}{n_{i+} n_{+j} }   \right) \sim \chi^2_{(I-1)(J-1)} 
\end{align*}
Note that the full model has $(IJ-1)$ parameters, and the null hypothesis has $(I-1)+ (J-1)$ parameters.
\begin{align*}
	df &= I \times J-1 - (I-1) - (J-1)\\
	&= (I-1)(J-1)
\end{align*}

The degrees of freedom refers to the number of random variables that used to estimate. When the mean of row or column are calculated, the mean will be used to center the data, then the free (random) data would be less 1. Here both the row and column will be less 1 when under $H_0$.

\end{itemize}




	\section{Sampling Method}
	 cross-sectional, prospective, retrospective sampling method.\\
	

	\subsection{Retrospective}
	In retrospective study, we know the response and further select patients accordingly, in other words, the columns margin is fixed. So we know the percentage of X in the Y categories. 
	
	For retrospective study, the Y is fixed. The likelihood function is constructed as below:
	
	\begin{align*}
		\theta &= p(X=1|Y=1) = \frac{\pi_{11}}{\pi_{11} + \pi_{21}}\\
		1- \theta &= p(X=0|Y=1) = \frac{\pi_{21}}{\pi_{11} + \pi_{21}}\\
		\gamma &= p(X=1|Y=0) = \frac{\pi_{12}}{\pi_{12} + \pi_{22}}\\
		1- \gamma &= p(X=0|Y=0) = \frac{\pi_{22}}{\pi_{12} + \pi_{22}}
	\end{align*} 
	
	$X | Y$ are binomial distribution, which is different from above multinomial distribution. And the $X|Y=0, X|Y=1$ are independent. 
	
	\begin{align*}
		p(\theta, \gamma) &= \theta^{n_{11}} (1-\theta)^{n_{21}} \gamma^{n_{12}} (1-\gamma)^{n_{22}}\\
		\log p(\theta, \gamma) &= n_{11} \log \theta + n_{21} \log (1-\theta) + n_{12} \log \gamma + n_{22} \log(1-\gamma)\\
		\frac{\partial ln}{\partial \theta} &= \frac{n_{11}}{\theta} - \frac{n_{21}}{1-\theta} = 0\\
		\hat{\theta} &= \frac{n_{11}}{n_{11}+ n_{21}}\\
		\frac{\partial ln}{\partial \gamma} &= \frac{n_{12}}{\gamma} - \frac{n_{22}}{1-\gamma} = 0\\
		\hat{\gamma} &= \frac{n_{12}}{n_{12}+ n_{22}}
	\end{align*} 
	By CLT,
	\begin{align*}
		\sqrt{n} \Bigg ( \begin{pmatrix} 
		\theta \\
		\gamma
		\end{pmatrix} -  \begin{pmatrix} 
		\hat{\theta} \\
		\hat{\gamma}
		\end{pmatrix} \Bigg ) & \xrightarrow{d}  N(0, \Sigma)
	\end{align*} 
		
	The covariance matrix, binomial distribution variance is $np(1-p)$
	
	\begin{align*}
		\sqrt{n} \left( \theta - \hat{\theta} \right) & \xrightarrow[]{d} N(0, \Sigma)\\
		\Sigma &= \begin{bmatrix}
			\theta(1-\theta) &  0 \\
			0 &  \gamma(1-\gamma) \\
		\end{bmatrix}\\
	\end{align*} 	
	
	The Fisher Information matrix is inverse of Covariance matrix.
	
	\begin{align*}
	  I(\theta, \gamma) &= \Sigma^{-1} \\
	  &= \begin{bmatrix}
			\theta^{-1}(1-\theta)^{-1} &  0 \\
			0 &  \gamma^{-1}(1-\gamma)^{-1} 
		\end{bmatrix}
	\end{align*} 		
	
	Then get covariance matrix by delta method for odds ratio, 
	
	\begin{align*}
		g(\theta) &= \frac{n_{11}n_{22}}{n_{21}n_{12}} = \frac{\theta/(1-\theta)}{\gamma/(1-\gamma)}\\
		\sqrt{n} \left( g(\hat\theta) - g({\theta}) \right) & \xrightarrow[]{d} N(0, g(\theta)' \Sigma^{New} g(\theta)'^T)\\  
		g(\theta)' &= \left( \frac{(1-\gamma)/\gamma}{1/(1-\theta)^2}, \frac{\theta/(1-\theta)}{-1/\gamma^2} \right)
	\end{align*} 
	
	The standard error for odds ratio in retrospective study
	
	\begin{align*}
		se(\hat R) &= \hat{R} \sqrt{\frac{1}{n_{.1}\hat{\pi}_{X=2|Y=1}\hat{\pi}_{X=1|Y=1} } + \frac{1}{n_{.2}\hat{\pi}_{X=2|Y=2} \hat {\pi}_{X=1|Y=2} } }\\
		\hat{\pi}_{X=2|Y=1} &= \frac{n_{21}}{n_{11}+ n_{21}}\\
		\hat{\pi}_{X=1|Y=1} &= \frac{n_{11}}{n_{11}+ n_{21}}\\
		\hat{\pi}_{X=2|Y=2} &=  \frac{n_{12}}{n_{12} + n_{22}}\\
		\hat {\pi}_{X=1|Y=2} &= \frac{n_{12}}{n_{12} + n_{22}}\\
		n_{.1} = n_{11}+ n_{21}, \quad n_{.2}=n_{12} + n_{22}\\
		se(\hat R) &= \frac{n_{22}n_{11}}{(n_{21}n_{12})} \sqrt{\frac{n_{11}+n_{21}}{n_{11}n_{21}} + \frac{n_{12}+n_{22}}{n_{12}n_{22}} }\\
		&= \frac{{n_{22}n_{11}}}{(n_{21}n_{12})} \sqrt{\frac{1}{n_{11}} + \frac{1}{n_{12}} + \frac{1}{n_{21}} + \frac{1}{n_{22}}}\\
	\end{align*}
	
\subsection{Prospective}
In prospective study, the row margin is fixed. We will need to figure out how many parameters are needed to construct the likelihood function.

We have
	\begin{align*}
	P(Y=1|X=1) &= \frac{\pi_{11}}{\pi_{11} + \pi_{12}} = \theta\\
	P(Y=0|X=1) &= \frac{\pi_{12}}{\pi_{11} + \pi_{12}} = 1 - P(Y=1|X=1)\\
	P(Y=1|X=  0) &= \frac{n_{21}}{n_{21}+ n_{22}} = \gamma \\
	P(Y= 0|X=  0) &=  \frac{n_{22}}{n_{21}+ n_{22}} = 1 - P(Y=1|X=  0)
\end{align*}

So the likelihood function
\begin{align*}
	P(\theta, \gamma) &= \theta^{n_{11}} (1-\theta)^{n_{12}} \gamma^{n_{21}} (1-\gamma)^{n_{22}}
\end{align*}

The standard error for odds ratio in prospective study
	\begin{align*}
		se(\hat R) &= \hat{R} \sqrt{\frac{1}{n_{1.}\hat{\pi}_{Y=2|X=1}\hat{\pi}_{Y=1|X=1} } + \frac{1}{n_{2.}\hat{\pi}_{Y=2|X=2} \hat {\pi}_{Y=1|X=2} } }\\
		\hat{\pi}_{Y=2|X=1} &= \frac{n_{12}}{n_{11}+ n_{12}}\\
		\hat{\pi}_{Y=1|X=1} &= \frac{n_{11}}{n_{11}+ n_{12}}\\
		\hat{\pi}_{Y=2|X=2} &=  \frac{n_{22}}{n_{21} + n_{22}}\\
		\hat {\pi}_{Y=1|X=2} &= \frac{n_{21}}{n_{21} + n_{22}}\\
		n_{1.} = n_{11}+ n_{12}, \quad n_{2.}=n_{21} + n_{22}\\
		se(\hat R) &= \frac{n_{22}n_{11}}{(n_{21}n_{12})} \sqrt{\frac{n_{11}+n_{12}}{n_{11}n_{12}} + \frac{n_{21}+n_{22}}{n_{21}n_{22}} }\\
		&= \frac{{n_{22}n_{11}}}{(n_{21}n_{12})} \sqrt{\frac{1}{n_{11}} + \frac{1}{n_{12}} + \frac{1}{n_{21}} + \frac{1}{n_{22}}}\\
	\end{align*}
	
	\subsection{Cross-Sectional}
	For cross-sectional study, we only have the total n fixed. That is the difference for each scenario. 
	To calculate the covariance matrix, we will use the MGF and take derivatives. Or use the cumulant function KGF to get the covariance.
	Use one random variable for the two way contingency table. While the Fisher information is the inverse of the covariance matrix, however we don't use Fisher information to calculate covariance matrix due to the math computation.\\
	
	Show that the sample odds ratio $\hat R = n_{22}n_{11}/(n_{21}n_{12})$ has the same standard error for cross-sectional, prospective and retrospective studies.
	
	
	The standard error for odds ratio in cross sectional study\\
	\begin{align*}
		se(\hat R) &= \frac{\hat{R}}{\sqrt{n}} \sqrt{\frac{1}{\hat{\pi_{11}}} + \frac{1}{\hat{\pi_{12}}} + \frac{1}{\hat{\pi_{21}}} + \frac{1}{\hat{\pi_{22}}}}\\
		&= \frac{{n_{22}n_{11}}}{(n_{21}n_{12})} \sqrt{\frac{1}{n_{11}} + \frac{1}{n_{12}} + \frac{1}{n_{21}} + \frac{1}{n_{22}}}\\
	\end{align*}

	
	By comparing the above standard errors in three types of studies, we see that they have same standard errors. Odds ratio is invariant in terms of sampling method. 
Similarly the coefficient of a particular covariate is associated with the odds ratio of the covariate, which is invariant with prospective and retrospective studies. Check out p747.


\subsection{Odds ratio}
	The covariance of odds ratio by delta method. We simplify $2 \times 2$ table as $\pi_{11} = \pi_1, \pi_{12} = \pi_2, \pi_{21} = \pi_3, \pi_{22} = \pi_4$.
	\begin{align*}
		g(\pi) &= \frac{\pi_{22}\pi_{11}}{\pi_{12}\pi_{21}} \qquad \pi=(\pi_{11}, \pi_{12}, \pi_{21}, \pi_{22})\\
		\sqrt{n} \left( g(\hat{\pi}) - g({\pi}) \right) & \xrightarrow[]{d} N \left(0, \diffp*{g(\pi)}{\pi}{} \Sigma \diffp*{g(\pi)}{\pi}{}^T \right)\\
		\diffp{g(\pi)}{\pi}  &= \left( \frac{\partial g}{\partial \pi_{11}}, \frac{\partial g}{\pi_{12}}, \frac{\partial g}{\partial \pi_{21}}, \frac{\partial g}{\partial \pi_{22}} \right)^T\\
		& = \left( \frac{\pi_{22}}{\pi_{21}\pi_{12}}, \frac{-\pi_{11}\pi_{22}}{\pi_{21}\pi_{12}^2}, \frac{-\pi_{11}\pi_{22}}{\pi_{12}\pi_{21}^2}, \frac{\pi_{11}}{\pi_{21}\pi_{12}} \right)^T\\
		\Sigma^{\ast} &= g(\pi)^2(\frac{1}{\pi_{11}} + \frac{1}{\pi_{12}} + \frac{1}{\pi_{21}} + \frac{1}{\pi_{22}})
	\end{align*} 
	So that,
	\begin{align*}
		Var(\hat R) &=  \frac{1}{n} \Sigma^{\ast} 
	\end{align*} 
	We consider $log \hat R$ instead of $\hat R$, because $log \hat R$ converges rapidly to a normal distribution compared to $\hat R$.
	\begin{align*}
		log(\hat{R}) &= log \pi_1 + \log \pi_2 - \log \pi_3  \log \pi_4\\
		\diffp{g(\pi)}{\pi}  &= \left(\frac{1}{\pi_{11}} , -\frac{1}{\pi_{12}}, -\frac{1}{\pi_{21}}, \frac{1}{\pi_{22}} \right)^T\\
		Var(log(\hat{R})) &= \frac{1}{n} \Tilde{\Sigma} \\
		\Tilde{\Sigma} &= \diffp*{g(\pi)}{\pi}{}^T \Sigma \diffp*{g(\pi)}{\pi}{}\\
		log(\hat R) &=  \frac{1}{n}\left( \frac{1}{\hat \pi_{11}} + \frac{1}{\hat \pi_{12}} + \frac{1}{\hat \pi_{21}} + \frac{1}{\hat \pi_{22}} \right)\\
		s.e. log(\hat R) &=  \frac{1}{\sqrt{n}} \sqrt{\frac{1}{\hat \pi_{11}} + \frac{1}{\hat \pi_{12}} + \frac{1}{\hat \pi_{21}} + \frac{1}{\hat \pi_{22}}} 
	\end{align*} 


\section{Conditional Probability}

Suppose that $\pi_{11}, \pi_{12}$ are parameters of interest and the rest of the parameters are treated as nuisance. Derive the conditional likelihood of $(\pi_{11}, \pi_{12})$ and the conditional MLE's of  $(\pi_{11}, \pi_{12})$.
If not specified, we treat as general contingency table that total n is fixed. If only $\pi_{11}, \pi_{12}$ are parameters of interest and the rest of the parameters are treated as nuisance, then we will set the rest of the parameters as one parameter, and get its distribution, which is to find the sufficient statistics for rest of the parameters.
Write the Multinomial distribution in exponential family distribution.\\
We can find marginal distribution by summing over along all possible values of $(n_{11}, n_{12})$. Note that $n_{11} \leq \min{n_{1+} - n_{12}, n_{+1}}$ for a given value of $n_{12}$. Similarly, $n_{12} \leq \min{n_{1+}- n_{11}, n_{+1}}$ for a given value of $n_{11}$. \\
Additionally,
\begin{align*}
	n & \geq n_{1+} + n_{+1} + n_{+2} - n_{11} - n_{12} \\
	n_{11} + n_{12} & \geq \max{ 0, n_{+1} + n_{1+} + n_{+2}}
\end{align*}
Let
\begin{align*}
	S(n_{11}, n_{12}) &= \{(n_{11}, n_{12}): n_{11} + n_{12} \geq \max{ 0, n_{+1} + n_{1+} + n_{+2}},\\
	&  n_{11} \leq \min{(n_{1+} - n_{12}, n_{+1})}, n_{12} \leq \min{(n_{1+}- n_{11}, n_{+1})}   \} 
\end{align*}

The conditional distribution
\begin{align*}
	p(n_{11}, n_{12}|n_{13}, ...n_{IJ}, n) &= \frac{p(n_{ij}}{p(S_n)}\\
	&= \frac{\frac{1}{n_{11}! n_{12}! } \pi_{11}^{n_{11}} \pi_{12}^{n_{12}}}{\sum_{(x, y \in S_n)} \frac{1}{x! y!} \pi_{11}^x \pi_{12}^y}
\end{align*}
And $\hat{\pi}_{11}, \hat{\pi}_{12}$ are the CMLE that maximize $p(n_{11}, n_{12}|n_{13}, ...n_{IJ}, n)$.


\subsection{Contingency table}
\begin{itemize}
	\item [(a)] Get MLE of $\pi$ and prove CLT.\\
	The multinomial distribution based on total n. 
	\begin{align*}
		p(\theta) &=n! \prod_{i=0}^1 \prod_{j=0}^1  \frac{\pi_{ij}^{n_{ij}}}{n_{ij}!}, \qquad \theta = (\pi_{00}, \pi_{01}, \pi_{10}, \pi_{11})^T\\
		ln p(\theta) &=log n!+ \sum_{i=0}^1 \sum_{j=0}^1 n_{ij}log( \pi_{ij}) - log n_{ij}! \\
		&= log n!+ n_{00}log \pi_{00}  + n_{01}log \pi_{01}  + n_{10}log \pi_{10}  + n_{11}log (1-\pi_{00}-\pi_{01} - \pi_{10})  
	\end{align*}
	The MLE of the $\theta$ by taking derivative to the log-likelihood
	\begin{align*}
		\frac{\partial ln(\theta)}{\partial \pi_{00}} &= \frac{n_{00}}{\pi_{00}} - \frac{n_{11}}{1-\pi_{00}-\pi_{01}-\pi_{10}} = 0\\  
		\frac{\partial ln(\theta)}{\partial \pi_{01}} &=\frac{n_{01}}{\pi_{01}} - \frac{n_{11}}{1-\pi_{00}-\pi_{01}-\pi_{10}} = 0 \\  
		\frac{\partial ln(\theta)}{\partial \pi_{10}} &= \frac{n_{10}}{\pi_{10}} - \frac{n_{11}}{1-\pi_{00}-\pi_{01}-\pi_{10}} = 0\\ 
		\hat{\pi_{00}} & = \frac{n_{00}}{n}\\
		\hat{\pi_{01}} & = \frac{n_{01}}{n}\\
		\hat{\pi_{10}} & = \frac{n_{10}}{n}\\
		\hat{\pi_{11}} & = \frac{n_{11}}{n}, \qquad n= n_{00} + n_{01} + n_{10} + n_{11}
	\end{align*}
	Let $Z_i= I(X=x, Y=y) \sim $ multi $(1, \pi_{00}, \pi_{01}, \pi_{10}, \pi_{11})$.
	\begin{align*}
		Z_1 &= I[(X,Y)= (0,0)]\\
		Z_2 &= I[(X,Y)= (0,1)]\\
		Z_3 &= I[(X,Y)= (1,0)]\\
		Z_4 &= I[(X,Y)= (1,1)]\\
		p(\theta) &= \prod_k \pi_{k}^{I(Z_k=1)}\\
		M_Z(t) &= E[exp(t^TZ)] = E[exp(t^T(Z_1 + Z_2 +... Z_n))] = E[exp(t^TZ_1 + t^TZ_2 + ... t^TZ_n)]\\
		&= E[\prod_{i=1}^n exp(t^TZ_i)]\\
		&= \prod_{i=1}^n E[exp(t^TZ_i)]  \qquad (\text{by independence})\\
		&= \prod_{i=1}^n M_{Z_i}(t) = \prod_{i=1}^n P(Z_i= 1) e^{tz_i}\qquad  \text{by MGF of discrete variable $Z_i$}\\
		&= \left( \sum_{j=1}^J \pi_j exp(t_j)\right)^n \qquad \text{by MGF of multinoulli}
	\end{align*}  
	Then the covariance matrix of $\theta$ could be calculated by MGF.
	\begin{align*}
		E(Z_1 Z_2) &= \frac{\partial^2 M_Z(t)}{\partial Z_i \partial Z_j}|_{t_i = t_j = 0}\\
		&= \frac{\partial \left(n(\pi_ie^{t_i})(\sum_{k=1}^K \pi_ke^{t_k})^{n-1} \right)'}{\partial t_j}\\
		&= n(n-1)(\sum_{k=1}^K \pi_ke^{t_k})^{n-2}\pi_i\pi_j|_{t_i = t_j = 0} = n(n-1)\pi_i\pi_j\\
		E(X_i) &= n\pi_i\\
		Cov(Z_i, Z_j) &= E(Z_i Z_2) - E(Z_1)E(Z_j) = n(n-1)\pi_i\pi_j - n^2 \pi_i\pi_j = -n\pi_i\pi_j\\
		Var(Z_i) &= E(Z_i^2) - E(Z_i)^2 \\
		E(Z_i^2) &=  \frac{\partial \left(n(\pi_ie^{t_i})(\sum_{k=1}^K \pi_ke^{t_k})^{n-1} \right)'}{\partial t_i}\\
		&= n(\sum_{k=1}^K \pi_ke^{t_k})^{n-1}\pi_i e^{t_i}+ n(n-1)(\sum_{k=1}^K \pi_ke^{t_k})^{n-2}\pi_i\pi_i e^{2t_i}|_{t_i = 0} \\
		&= n\pi_i + n(n-1)\pi_i^2 = n\pi_i(1-\pi)\\
		Var(Z_i/n) &= \frac{1}{n^2} Var(Z_i) = \frac{1}{n}\pi_i(1-\pi_i)
	\end{align*}
	Thus the covariance matrix is
	\begin{align*}
		\Sigma &= \begin{bmatrix}
			\pi_{00}(1-\pi_{00}) &  -\pi_{00}\pi_{01}&  -\pi_{00}\pi_{10} &  -\pi_{00}\pi_{11}\\
			-\pi_{01}\pi_{00} & \pi_{01}(1-\pi_{01}) & -\pi_{01}\pi_{10}   & -\pi_{01}\pi_{11}  \\
			-\pi_{10}\pi_{00} & -\pi_{10}\pi_{01} &  \pi_{10}(1-\pi_{10})  & -\pi_{10}\pi_{11}  \\
			-\pi_{11}\pi_{00} &  -\pi_{11}\pi_{01} & -\pi_{11}\pi_{10}   & \pi_{11}(1-\pi_{11})  \\
		\end{bmatrix}= diag{(\pi_{ij}) - \theta \theta^T}
	\end{align*}
	By Central limit theroem, 
	\begin{align*}
		\sqrt{n} (\hat{\pi_{00}} - \pi_{00}, \hat{\pi_{01}}- \pi_{01}, \hat{\pi_{10}} - \pi_{10}, \hat{\pi_{11}}- \pi_{11} )^T & \xrightarrow[]{d} N(0, \Sigma)
	\end{align*}
	\item[(b)] Let R denote the odds ratio. Find the maximum likelihood estimate of log(R) and
	derive its asymptotic distribution.\\
	By invariance of MLE:
	\begin{align*}
		R & =  \frac{\pi_{00}\pi_{11}}{\pi_{01}\pi_{10}}\\
		g(R) &= log R = log \pi_{00} + log \pi_{11}- log \pi_{01}- log \pi_{10}\\
		log \hat{R} & = log \hat{\pi_{00}} + log \hat{\pi_{11}}- log \hat{\pi_{01}}- log \hat{\pi_{10}}\\
		&= log \frac{n_{00}n_{11}}{n_{01}n_{10}}
	\end{align*}
	
	By Central limit theorem, we have 
	\begin{align*}
		\sqrt{n} \left(\hat{g(R)} - g(R) \right) & \xrightarrow[]{d} N \left(0, \frac{\partial g(R)}{\partial \theta} \Sigma   \frac{\partial g(R)}{\partial \theta}^T \right) \\
	\end{align*}
	By delta method,
	\begin{align*}
		\frac{\partial g(R)}{\partial \theta} &= \left(
		\frac{1}{R} \frac{\partial R}{\partial \pi_{00}} ,  \frac{1}{R}\frac{\partial R}{\partial \pi_{01}},   \frac{1}{R}\frac{\partial R}{\partial \pi_{10}} ,  \frac{1}{R} \frac{\partial R}{\partial \pi_{11}} \right)\\
		& = \left( \frac{1}{\pi_{00}},  -\frac{1}{\pi_{01}},  -\frac{1}{\pi_{10}}, \frac{1}{\pi_{11}} \right)\\
		\Sigma^{R} &= \frac{\partial g(R)}{\partial \theta} \Sigma \frac{\partial g(R)}{\partial \theta}' \\
		&= \left( \frac{1}{\pi_{00}},  -\frac{1}{\pi_{01}},  -\frac{1}{\pi_{10}}, \frac{1}{\pi_{11}} \right) \begin{bmatrix}
			\pi_{00}(1-\pi_{00}) &  -\pi_{00}\pi_{01}&  -\pi_{00}\pi_{10} &  -\pi_{00}\pi_{11}\\
			-\pi_{01}\pi_{00} & \pi_{01}(1-\pi_{01}) & -\pi_{01}\pi_{10}   & -\pi_{01}\pi_{11}  \\
			-\pi_{10}\pi_{00} & -\pi_{10}\pi_{01} &  \pi_{10}(1-\pi_{10})  & -\pi_{10}\pi_{11}  \\
			-\pi_{11}\pi_{00} &  -\pi_{11}\pi_{01} & -\pi_{11}\pi_{10}   & \pi_{11}(1-\pi_{11})  \\
		\end{bmatrix} \begin{bmatrix}
			\frac{1}{\pi_{00}} \\
			-\frac{1}{\pi_{01}}   \\
			-\frac{1}{\pi_{10}}  \\
			\frac{1}{\pi_{11}}  \\
		\end{bmatrix}\\
		&= (\frac{1}{\pi_{00}} + \frac{1}{\pi_{01}} + \frac{1}{\pi_{10}} + \frac{1}{\pi_{11}})\\
	\end{align*}
	We have the asymptotic distribution of $log(R)$
	\begin{align*}
		\sqrt{n} (log\hat{R} - logR) & \xrightarrow[]{d} N \left(0, (\frac{1}{\pi_{11}} + \frac{1}{\pi_{12}} + \frac{1}{\pi_{21}} + \frac{1}{\pi_{22}}) \right) 
	\end{align*}
	\item[(c)] Construct an approximate 95$\%$ confidence interval for the odds ratio R.\\
	From part (b), we have the asymptotic normal distribution of $log R$. We have the asymptotic distribution of $R$.
	\begin{align*}
		f &= exp(g) = R, \qquad f(g)' = R\\
		\sqrt{n} (\hat{f(g)} - f(g)) & \xrightarrow[]{d} N \left(0, f(g)'(\frac{1}{\pi_{11}} + \frac{1}{\pi_{12}} + \frac{1}{\pi_{21}} + \frac{1}{\pi_{22}}) f(g)'^T \right)\\
		\sqrt{n} (\hat{R} - R) & \xrightarrow[]{d} N \left(0, R^2(\frac{1}{\pi_{11}} + \frac{1}{\pi_{12}} + \frac{1}{\pi_{21}} + \frac{1}{\pi_{22}}) \right)\\
		(\hat{R} - R) & \xrightarrow[]{d} N \left(0, \frac{1}{n} R^2(\frac{1}{\pi_{11}} + \frac{1}{\pi_{12}} + \frac{1}{\pi_{21}} + \frac{1}{\pi_{22}}) \right)
	\end{align*}
	The 95$\%$ confidence interval for the odds ratio R
	\begin{align*}
		\{R &: \hat{R} - 1.96\hat{R} \sqrt{\frac{1}{\pi_{11}} + \frac{1}{\pi_{12}} + \frac{1}{\pi_{21}} + \frac{1}{\pi_{22}}} \leq  R \leq \hat{R} + 1.96\hat{R} \sqrt{\frac{1}{\pi_{11}} + \frac{1}{\pi_{12}} + \frac{1}{\pi_{21}} + \frac{1}{\pi_{22}}} \}
	\end{align*}
	
	\item[(d)] Under the assumptions of part (a), further assume that$ \pi_{1+} = \pi_{11} + \pi_{10} = \frac{exp(\alpha)}{1+\exp(\alpha)} $ and $ \pi_{+1} = \pi_{11} + \pi_{01} = \frac{exp(\alpha + \beta)}{1+\exp(\alpha + \beta)} $ . Derive the maximum likelihood estimates of $(\alpha, \beta)$, denoted by $(\hat{\alpha}; \hat{\beta})$.\\
	\begin{align*}
		\pi_{01} + \pi_{11} & = \frac{exp(\alpha)}{1+\exp(\alpha)} \\
		exp(\alpha) &= \frac{\pi_{10} + \pi_{11}}{\pi_{01} + \pi_{00}}, \qquad \alpha = log \left( \frac{\pi_{10} + \pi_{11}}{\pi_{01} + \pi_{00}}\right)\\
		\pi_{10}+ \pi_{11} & = \frac{exp(\alpha + \beta)}{1+\exp(\alpha + \beta)} \\
		\alpha + \beta &= log \left( \frac{\pi_{01} + \pi_{11}}{\pi_{10} + \pi_{00}} \right)\\
		\beta &= log \left( \frac{\pi_{01} + \pi_{11}}{\pi_{10} + \pi_{00}} \right) - log \frac{\pi_{10} + \pi_{11}}{\pi_{01} + \pi_{00}}, \qquad \beta &= log \left(\frac{(\pi_{01} + \pi_{11})(\pi_{01} + \pi_{00})}{(\pi_{10} + \pi_{00}) (\pi_{10} + \pi_{11})} \right)
	\end{align*}
	By invariance of MLE,
	\begin{align*}
		\hat\alpha &= log \left( \frac{\hat{\pi_{10}} + \hat{\pi_{11}}}{\hat{\pi_{01}} + \hat{\pi_{00}}}\right) = log \left(\frac{n_{10} + n_{11}}{n_{01} + n_{00}} \right)\\
		\hat\beta &= log \left(\frac{(\hat\pi_{01} + \hat\pi_{11})(\hat\pi_{01} + \hat\pi_{00})}{(\hat\pi_{10} + \hat\pi_{00}) (\hat\pi_{10} + \hat\pi_{11})} \right) = log \left(\frac{(n_{01} + n_{11})(n_{01} + n_{00})}{(n_{10} + n_{00}) (n_{10} + n_{11})} \right)
	\end{align*}
	\item[(e)] Using the assumptions of part (d), derive the asymptotic distribution of $(\alpha, \beta)$ (properly normalized).\\
	By Central limit theorem and delta method,
	\begin{align*}
		\xi &= (\alpha, \beta)^T \\
		g(\xi) &= \{ log \left( \frac{\pi_{10} + \pi_{11}}{\pi_{01} + \pi_{00}}\right), log \left(\frac{(\pi_{01} + \pi_{11})(\pi_{01} + \pi_{00})}{(\pi_{10} + \pi_{00}) (\pi_{10} + \pi_{11})} \right)\}^T \\
		\sqrt{n} (\hat{g(\xi)} - g(\xi)) & \xrightarrow[]{d} N \left(0, \Sigma^{N} \right) \\
		\Sigma^{N} &= \frac{\partial g(\xi)}{\partial \pi} \Sigma \frac{\partial g(\xi)}{\partial \pi}^T
	\end{align*}
	
	$\Sigma^{N}$ is calculated by delta method,
	\begin{align*}
		\frac{\partial g(\alpha)}{\partial \pi_{00}} &= -\frac{1}{(\pi_{01} + \pi_{00})} = -\frac{1}{\pi_{0+}} \\
		\frac{\partial g(\alpha)}{\partial \pi_{01}} &= -\frac{1}{(\pi_{01} + \pi_{00})} = -\frac{1}{\pi_{0+}}\\
		\frac{\partial g(\alpha)}{\partial \pi_{10}} &= \frac{1}{(\pi_{10} + \pi_{11})}= \frac{1}{\pi_{1+}}\\
		\frac{\partial g(\alpha)}{\partial \pi_{11}} &= \frac{1}{(\pi_{10} + \pi_{11})}= \frac{1}{\pi_{1+}}\\
		\frac{\partial g(\beta)}{\partial \pi_{00}} &= \frac{(\pi_{10}-\pi_{01})}{(\pi_{01} + \pi_{00})(\pi_{00} + \pi_{10})} = -\frac{1}{(\pi_{10} + \pi_{00})} +\frac{1}{(\pi_{01} + \pi_{00})} = -\frac{1}{\pi_{+0} }  +\frac{1}{\pi_{0+}}\\
		\frac{\partial g(\beta)}{\partial \pi_{01}} &= \frac{1}{(\pi_{01} + \pi_{11})} + \frac{1}{(\pi_{01} + \pi_{00})}  \\
		\frac{\partial g(\beta)}{\partial \pi_{10}} &=- \frac{1}{(\pi_{10} + \pi_{00})} - \frac{1}{(\pi_{10} + \pi_{11})}\\
		\frac{\partial g(\beta)}{\partial \pi_{11}} &= \frac{(\pi_{10}-\pi_{01})}{(\pi_{10} + \pi_{11})(\pi_{01} + \pi_{11})} = - \frac{1}{(\pi_{10} + \pi_{11})} +\frac{1}{(\pi_{01} + \pi_{11})} \\
		\frac{\partial g(\xi)}{\partial \pi} &=\begin{bmatrix}
			-\frac{1}{\pi_{0+}} &  -\frac{1}{\pi_{0+}} &  \frac{1}{\pi_{1+}} &  \frac{1}{\pi_{1+}}\\
			\frac{1}{\pi_{0+} }  -\frac{1}{\pi_{+0}} & \frac{1}{\pi_{0+} } + \frac{1}{\pi_{+1}} & - \frac{1}{\pi_{+0} } - \frac{1}{\pi_{1+}} & \frac{1}{\pi_{+1} } -\frac{1}{\pi_{1+}}    \\
		\end{bmatrix}\\
		\Sigma^{N} &= \frac{\partial g(\xi)}{\partial \pi}\Sigma \frac{\partial g(\xi)}{\partial \pi}^T\\
		&= \left(\frac{1}{\pi_{11}} + \frac{1}{\pi_{12}} + \frac{1}{\pi_{21}} + \frac{1}{\pi_{22}} \right) 
	\end{align*}
	\item[(f)] Under the model of part (d), show that $(\pi_{1+}\pi_{0+})^{-1} + (\pi_{+1}\pi_{+0})^{-1} \leq (\pi_{1+}\pi_{+0})^{-1} + (\pi_{+1}\pi_{0+})^{-1}$.\\
	\begin{align*}
		&(\pi_{1+}\pi_{+0})^{-1} + (\pi_{+1}\pi_{0+})^{-1} - (\pi_{1+}\pi_{0+})^{-1} - (\pi_{+1}\pi_{+0})^{-1}\\
		&= \frac{\pi_{0+}- \pi_{+0}}{\pi_{1+}\pi_{+0}\pi_{0+}} + \frac{\pi_{+0} - \pi_{0+}}{\pi_{+1}\pi_{0+}\pi_{+0}}\\
		&= \frac{(\pi_{0+}-\pi_{+0})(\pi_{+1}-\pi_{1+})}{\pi_{1+}\pi_{+0}\pi_{0+}\pi_{+1}}\\
		&=  \frac{(\pi_{01}-\pi_{10})^2}{\pi_{1+}\pi_{+0}\pi_{0+}\pi_{+1}} \geq 0
	\end{align*}
	From above, we have $(\pi_{1+}\pi_{0+})^{-1} + (\pi_{+1}\pi_{+0})^{-1} \leq (\pi_{1+}\pi_{+0})^{-1} + (\pi_{+1}\pi_{0+})^{-1}$.
\end{itemize}


\backmatter

\import{./}{bibliography.tex}


\end{document}
