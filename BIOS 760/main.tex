\documentclass[a4paper,11pt]{book}
\usepackage{import}
\usepackage{amsmath}
\usepackage{mathtools}
\usepackage[utf8]{inputenc} % set input encoding (not needed with XeLaTeX)
\usepackage[thinc]{esdiff}
%\usepackage{example}

%%% Examples of Article customizations
% These packages are optional, depending whether you want the features they provide.
% See the LaTeX Companion or other references for full information.

%%% PAGE DIMENSIONS
\usepackage{geometry} % to change the page dimensions
\geometry{a4paper} % or letterpaper (US) or a5paper or....
% \geometry{margin=2in} % for example, change the margins to 2 inches all round
% \geometry{landscape} % set up the page for landscape
%   read geometry.pdf for detailed page layout information

\usepackage{graphicx} % support the \includegraphics command and options

% \usepackage[parfill]{parskip} % Activate to begin paragraphs with an empty line rather than an indent

%%% PACKAGES
\usepackage{booktabs} % for much better looking tables
\usepackage{array} % for better arrays (eg matrices) in maths
\usepackage{paralist} % very flexible & customisable lists (eg. enumerate/itemize, etc.)
\usepackage{verbatim} % adds environment for commenting out blocks of text & for better verbatim
\usepackage{subfig} % make it possible to include more than one captioned figure/table in a single float
% These packages are all incorporated in the memoir class to one degree or another...

%%% HEADERS & FOOTERS
\usepackage{fancyhdr} % This should be set AFTER setting up the page geometry
\pagestyle{fancy} % options: empty , plain , fancy
\renewcommand{\headrulewidth}{0pt} % customise the layout...
\lhead{}\chead{}\rhead{}
\lfoot{}\cfoot{\thepage}\rfoot{}

%%% SECTION TITLE APPEARANCE
\usepackage{sectsty}
\allsectionsfont{\sffamily\mdseries\upshape} % (See the fntguide.pdf for font help)
% (This matches ConTeXt defaults)

%%% ToC (table of contents) APPEARANCE
\usepackage[nottoc,notlof,notlot]{tocbibind} % Put the bibliography in the ToC
\usepackage[titles,subfigure]{tocloft} % Alter the style of the Table of Contents
\renewcommand{\cftsecfont}{\rmfamily\mdseries\upshape}
\renewcommand{\cftsecpagefont}{\rmfamily\mdseries\upshape} % No bold!
\usepackage[english]{babel}
\usepackage{amsthm}
\theoremstyle{definition}
\newtheorem{definition}{Definition}[section]
\newtheorem{theorem}{Theorem}[section]
\newtheorem{corollary}{Corollary}[theorem]
\newtheorem{lemma}[theorem]{Lemma}
\theoremstyle{remark}
\newtheorem*{remark}{Remark}


\usepackage{makeidx}
\makeindex

\begin{document}

\frontmatter
\import{./}{title.tex}
\clearpage
%\thispagestyle{empty}

\tableofcontents

\mainmatter

\chapter{Distributions}

%\import{sections/}{section1-1.tex}
%\clearpage
\import{sections/}{section1-2.tex}
\clearpage
\import{sections/}{section1-3-BiNormal.tex}
\clearpage
\import{sections/}{section1-3-MVN.tex}
\clearpage
\import{sections/}{section1-4.tex}
\clearpage
\import{sections/}{section1-5.tex}

\chapter{Distribution Transformation}
\section{Conditional Distribution}

Conditional distribution is used in sufficient statistics (ie. show T(X) is sufficient which the distribution based on sufficient statistics does not depend on $\theta$), 
UMVUE $E[\theta | T(X)]$, nuisance parameter $p(\theta_1 | \theta_2,.. \theta_n, X)$, Bayesian statistics. 

Basically we can write the distribution of the based on statistics, if not, we will write the integral 




\chapter{Parameter Estimates}

\section{The Standard Exponential Distribution}

The standard exponential distribution family 

\begin{align*}
p(y| \theta) &= \phi \Big[ \exp \Big( y \theta - b(\theta) \Big) - c(y) \Big] - \frac{1}{2} s(y, \phi)
\end{align*}

We will explore the fun characteristics of the exponential family

\begin{itemize}
\item[(i)] Mean and Variance by derivatives

\begin{align*}
log  \int p(y| \theta) &=log  \int \phi \Big[ \exp \Big( y \theta - b(\theta) \Big) - c(y) \Big] - \frac{1}{2} s(y, \phi) dv = 0 \\
 log \int \exp \{( y \theta ) \} h(y) v(dy) &= b(\theta) \\
 \partial_{\theta} log \int \exp \{( y \theta ) \} h(y) v(dy) &= \partial_{\theta}  b(\theta) \\
\end{align*}

To proceed we need to move the gradient past the integral sign. In general derivatives can not be moved past integral signs (both are certain kinds of limits, and sequences of limits can differ depending on the order in which the limits are taken). However it turns out that the move is justified in this case by an appeal to the dominated convergence theorem. 

\begin{align*}
\partial_{\theta}  b(\theta) &= \partial_{\theta}  log \int \exp \{( y \theta ) \} h(y) v(dy)\\
 &=  \frac{\int y \exp \{( y \theta ) \} h(y) v(dy) }{\int \exp \{( y \theta ) \} h(y) v(dy)} \\
 &= \int y \exp \{ y \theta - b(\theta) \} h(x) v(dy) \\
 &= E[y] 
\end{align*}

Also we can see that the first derivative of $b(\theta)$ is equal to the mean of the sufficient statistics. Similar for the variance.

Another proof is to use the Bartlett's identities

Suppose that differentiation and integration are exchangeable and all the necessary expectations are finite. We have the following results:

\begin{align*}
E\_{\xi} \Big( \partial_j l_n \Big) &= 0,\\
E_{\xi} \Big( \partial^2_{j,k} l_n \Big) + E_{\xi} \Big( \partial_j l_n \partial_k l_n \Big) = 0 \\
\end{align*}

By the above two equations, we can get the expectation and variance. 


\end{itemize}



\clearpage
\section{The Bernoulli Distribution}

The standard exponential distribution family 

\begin{align*}
p(y| \theta) &= \phi \Big[ \exp \Big( y \theta - b(\theta) \Big) - c(y) \Big] - \frac{1}{2} s(y, \phi)
\end{align*}

For Bernoulli distribution,
\begin{align*}
p(x| \pi) &= \pi^{x} (1- \pi)^{1-x} \\
&= \exp \{ \log \Big( \frac{\pi}{1- \pi} \Big) x + \log (1 - \pi) \}
\end{align*}

We see that Bernoulli distribution is an exponential family distribution with 

\begin{align*}
\theta &= \log \Big( \frac{\pi}{1- \pi} \Big) \\
b(\theta)&=- \log (1 - \pi) =  \log \Big( 1 + \exp(\theta) \Big) x \\
\phi & = 1
\end{align*}

\subsection{Mean and Variance}

For a univariate random variable $Y$, in this case, all the $Y_i$ have the same $\pi$
\begin{align*}
\diffp{b(\theta)}{\theta} &= \frac{\exp(\theta)}{1 + \exp(\theta) } = \frac{1}{1 + \exp(-\theta)} = \mu = E(Y) \\
\diffp{b(\theta)}{\theta \theta}  &= \frac{\exp(\theta)}{\Big[ 1 + \exp(\theta) \Big]^2} = \mu(1-\mu) =Var(Y)
\end{align*}

In regression model, $logit (\pi) = X \beta$, which $\beta$ is a vector, then we will use the chain rule. And each individual $y_i$ has its own equation that $\pi_i$ is different.

\begin{align*}
\theta & = X \beta, \qquad \theta_i = x_i^{T} \beta \\
\partial_{\beta}{b(\theta_i)} &= \partial_{\theta_i}{b(\theta_i)} \partial_{\beta}{{\theta_i}} \\
&= \frac{\exp(\theta_i)}{1 + \exp(\theta_i) }  x_i= \frac{1}{1 + \exp(-\theta)} x_i= \mu_i x_i\\
\partial^2_{\beta}{{b(\theta_i)}} &= \frac{\exp(\theta_i)}{\Big[ 1 + \exp(\theta_i) \Big]^2} x_i^{\otimes 2}= \mu_i(1-\mu_i) x_i^{\otimes 2}
\end{align*}

And we will need to connect this with the Fisher Information or Newton-Raphson algorithm

\begin{align*}
\theta_i & = k \Big(x_i^{T} \beta \Big) = x_i^{T} \beta \\
\xi &= (\beta, \phi)\\
ln(\xi) &= \sum_{i=1}^n \phi \Big[ y_i k \Big(x_i^{T} \beta \Big) - b \Big( k \Big(x_i^{T} \beta \Big)  \Big) - c(y_i) \Big] - \frac{1}{2} s(y_i, \phi) \\
\dot{ln}(\beta) &= \diffp{ln(\beta) }{\beta} = \phi \sum_{i=1}^n \Big[ y_i - \dot{b} \Big( k \Big(x_i^{T} \beta \Big)  \Big)  \Big] \dot{k} \Big(x_i^{T} \beta \Big) x_i \\
&= \sum_{i=1}^n \Big[ y_i - \mu_i \Big] x_i \\
\ddot{ln}(\beta) &= \diffp{ln(\beta) }{\beta \beta} = -\phi \sum_{i=1}^n \ddot{b} \Big( k(x_i^T \beta) \Big) \dot{k}(x_i^T \beta)^2 x_i x_i^T + \phi \sum_{i=1}^n \Big[y_i - \dot{b}(k(x_i^T \beta)) \Big] \ddot{k}(x_i^T \beta) x_i x_i^T \\
&= -\sum_{i=1}^n \ddot{b} \Big(\theta_i \Big) x_i x_i^T = -\sum_{i=1}^n V(\theta_i) x_i x_i^T, \qquad \partial^2_{\beta}{{b(\theta_i)}} = V(\theta_i)
\end{align*}

let 
\begin{align*}
V(\theta) & = diag \{ V(\theta_i) \} , \qquad e_i = y_i - \mu_i\\
\sum_{i=1}^n V(\theta_i) x_i x_i^T &= X V(\theta) V^T\\
\mu_i &= \dot{b}(\theta_i), \qquad v_i = \ddot{b}(\theta_i)\\
\dot{\theta}_i &= \partial_{\beta} \theta_i = \dot{k}(x_i^T \beta) x_i, \qquad \ddot{\theta}_i = \partial^2_{\beta} \theta_i = \ddot{k}(x_i^T \beta) x_i x_i^T \\
\dot{b}(\theta_i) &= \partial_{\theta} b(\theta) \Big |_{\theta = \theta_i}, \dot{k}(\eta) = \partial_{\eta} k(\eta), \ddot{k}(\eta) = \partial^2_{\eta}(\eta)
\end{align*}

So
\begin{align*}
E \Big[ - \ddot{l}n(\beta) \Big] & = \phi \sum_{i=1}^n v_i \dot{\theta}_i^{\otimes 2}
\end{align*}

Another set is to use $E(y_i), Var(y_i)$ which is also used commonly as that are the information we generally get. It is used a lot in GEE. 
\begin{align*}
\partial_{\mu} \theta &= \partial_{\theta} \mu ^{-1}, \qquad \partial_{\mu} \mu = \partial_{\theta} \mu \partial_{\mu} \theta = 1\\
\partial_{\theta} \mu &= \partial_{\theta} b(\theta) = \ddot{b}(\theta) \\
\partial_{\mu} \theta &= \Big( \partial_{\theta} \mu \Big)^{-1} =  \ddot{b}(\theta)^{-1} \\
\end{align*}

Then we have the connection between the two system
\begin{align*}
\partial_{\beta} \theta &= \partial_{\beta} \mu_i \partial_{\mu_i} \theta_i = \partial_{\beta} \mu_i \Big[ \ddot{b}(\theta_i) \Big]^{-1} \\
\partial_{\beta}^2 \theta_i &= \Big( \partial^2_{\mu_i} \theta_i \Big) \Big( \partial_{\beta} \mu_i \Big)^{\otimes 2} + \partial_{\mu_i} \theta_i \Big( \partial_{\beta}^2 \mu_i \Big) \\
&= - \dddot{b}(\theta_i) \ddot{b}(\theta_i)^{-3} \Big( \partial_{\beta} \mu_i \Big)^{\otimes 2} + \Big[ \ddot{b}(\theta_i) \Big]^{-1} \Big( \partial^2_{\beta} \mu_i \Big)
\end{align*}

The generalized estimation model
\begin{align*}
V(\beta) &= \text{diag} \Big( v_1(\beta), …, v_n(\beta) \Big) \\
e(\beta) &= (y_1 - \mu_1(\beta), …, y_n- \mu_n(\beta))^{'} \\
D_{\theta} (\beta)^{'} &= \Big( \partial_{\beta} \beta_1(\beta),…,  \partial_{\beta} \beta_n(\beta)\Big)_{p \times n} \\
D (\beta)^{T} &= \Big( \partial_{\beta} \mu_1(\beta),…,  \partial_{\beta} \mu_n(\beta) \Big)_{p \times n} \\
\dot{l}_n(\beta) &= \phi D_{\theta}(\beta)^{T} e(\beta) = \phi D(\beta)^{'} V(\beta)^{-1} e(\beta) \\
E \Big[ -\ddot{l}_n(\beta) \Big] &= \phi D_{\theta}(\beta)^{'} V D_{\theta}(\beta) = \phi D(\beta)^{'} V(\beta)^{-1} D(\beta) 
\end{align*}



\chapter{Convergence Theorem}

\section{Conditional MLE}
The conditional maximum likelihood estiamte (CMLE) of $\psi$ is not calculated directly from the conditional distribution of $\psi$. While we get it from $p(n_{11}| n_{1+}, n_{+1}, n, \psi)$.

We should be able to get the MLE from the log-likelihood conditional.

$\hat{\psi}_c$ is the solution to 
\begin{align*}
	n_{11} &= P_1(\hat{\psi}_c)/P_0(\hat{\psi}_c) = \mu\\
	log P(n_{11}| n_{1+}, n_{+1}, n, \psi) &=  n_{11} log \psi - log P_0(\psi) + c\\
	\diffp{log P}{\psi} &= \frac{n_{11}}{\psi} -\frac{P_0(\psi)'}{P_0(\psi)} = 0\\
	n_{11} &= P_1(\hat{\psi}_c)/P_0(\hat{\psi}_c)
\end{align*}
The variance of $\hat{\psi}_c$ can be approximated by the inverse of the Fisher information matrix $I_n(\hat{\psi}_c)$, which is given
\begin{align*}
	I_n(\hat{\psi}_c) &= E\{[ \partial_{\psi} log P(n_{11}| n_{1+}, n_{+1}, n, \hat{\psi}_c)]^2 \} = \frac{Var(n_{11}| n_{1+}, n_{+1}, n, \hat{\psi}_c)}{\hat{\psi}_c^2}
\end{align*}


\item[(g)] Suppose that $\pi_{11}, \pi_{12}$ are parameters of interest and the rest of the parameters are treated as nuisance. Derive the conditional likelihood of $(\pi_{11}, \pi_{12})$ and the conditional MLE's of  $(\pi_{11}, \pi_{12})$.
If not specified, we treat as general contingency table that total n is fixed. If only $\pi_{11}, \pi_{12}$ are parameters of interest and the rest of the parameters are treated as nuisance, then we will set the rest of the parameters as one parameter, and get its distribution, which is to find the sufficient statistics for rest of the parameters.
Write the Multinomial distribution in exponential family distribution.\\
We can find marginal distribution by summing over along all possible values of $(n_{11}, n_{12})$. Note that $n_{11} \leq \min{n_{1+} - n_{12}, n_{+1}}$ for a given value of $n_{12}$. Similarly, $n_{12} \leq \min{n_{1+}- n_{11}, n_{+1}}$ for a given value of $n_{11}$. \\
Additionally,
\begin{align*}
	n & \geq n_{1+} + n_{+1} + n_{+2} - n_{11} - n_{12} \\
	n_{11} + n_{12} & \geq \max{ 0, n_{+1} + n_{1+} + n_{+2}}
\end{align*}
Let
\begin{align*}
	S(n_{11}, n_{12}) &= \{(n_{11}, n_{12}): n_{11} + n_{12} \geq \max{ 0, n_{+1} + n_{1+} + n_{+2}},\\
	&  n_{11} \leq \min{(n_{1+} - n_{12}, n_{+1})}, n_{12} \leq \min{(n_{1+}- n_{11}, n_{+1})}   \} 
\end{align*}

The conditional distribution
\begin{align*}
	p(n_{11}, n_{12}|n_{13}, ...n_{IJ}, n) &= \frac{p(n_{ij}}{p(S_n)}\\
	&= \frac{\frac{1}{n_{11}! n_{12}! } \pi_{11}^{n_{11}} \pi_{12}^{n_{12}}}{\sum_{(x, y \in S_n)} \frac{1}{x! y!} \pi_{11}^x \pi_{12}^y}
\end{align*}
And $\hat{\pi}_{11}, \hat{\pi}_{12}$ are the CMLE that maximize $p(n_{11}, n_{12}|n_{13}, ...n_{IJ}, n)$.

\end{itemize}


\clearpage
\section{Sample Variance Distribution}

Suppose that $X_1,..X_n$ are i.i.d. with $E(X_i) = \mu, Var(X_i) = \sigma^2$, and $Var\Big[ (X_i- \mu)^2 \Big] = \tau < \infty$. Define $S_n^2 = \frac{1}{n} \sum_{i=1}^n (X_i - \bar{X}_n)^2$. Find the asymptotic distribution of $S_n^2 $.

We have several methods to get the asymptotic distribution:

1. Direct method: Get the mean and variance, then use CLT;

2. Delta method: when there is a function of the known distribution, ie. We know $X$ distribution, then we would like to know $f(X)$ distribution. The key is to get the mean and variance of the new function, which we times the derivative

In this problem, we could use the given information to use CLT method:

WRONG METHOD:
\begin{align*}
S_n^2 &= \frac{1}{n} \sum_{i=1}^n (X_i - \bar{X}_n)^2 = \frac{1}{n} \sum_{i=1}^n (X_i - E(X_i) + E(X_i) - \bar{X}_n)^2, \qquad \text{construct the variance component} \\
&= \frac{1}{n}\sum_{i=1}^n  \Big(X_i - E(X_i) \Big)^2 + \frac{1}{n}\sum_{i=1}^n \Big(E(X_i) - \bar{X}_n \Big)^2 \\
&= \sigma^2 +  \frac{1}{n}\sum_{i=1}^n \Big(E(X_i) - \bar{X}_n \Big)^2  \qquad \text{THIS IS WRONG}
\end{align*}

CAN'T SAY $\frac{1}{n}\sum_{i=1}^n  \Big(X_i - E(X_i) \Big)^2 =  \sigma^2 $, as this is not expectation, it is sample mean/variance. 


CORRECT METHOD:

We need to use the information $Var\Big[ (X_i- \mu)^2 \Big] = \tau < \infty$, to construct a CLT with variance component of $(X_i- \mu)$. 
To simplify, let $Y_i= X_i - \mu, \bar{Y}_n= \bar{X}_n - \mu$,

\begin{align*}
S_n^2 &=  \frac{1}{n} \sum_{i=1}^n Y_i^2 - \bar{Y}_n^2 \\
Var \Big( Y_i^2 \Big) &= \tau^2, \qquad E \Big( Y_i^2 \Big) = E \Big( (X_i - \mu)^2 \Big) = \sigma^2 \\
\bar{Y}_n^2 & \xrightarrow{d} 0
\end{align*}

It is very tricky to notice this, we only see the mean has asymptotic distribution based on CLT, while other series does not have asymptotic distribution. $\bar{Y}_n^2$ is only a number.
By slutsky theorem,

\begin{align*}
\sqrt{n} \Big(S_n^2 - \sigma^2 \Big) &= \sqrt{n} \Big(\frac{1}{n} \sum_{i=1}^n Y_i^2 - \sigma^2 \Big) + \sqrt{n} \bar{Y}_n^2 \\
\sqrt{n} \Big(S_n^2 - \sigma^2 \Big) & \xrightarrow{d} N(0, \tau^2)
\end{align*}




\clearpage
\section{UMVUE}

\subsection{Sufficient Statistics}
We learn sufficient statistics from a specific example, then extend to generalized case. So the first sufficient statistics that easily to look at is the sufficient statistics for mean. For every distribution which we can write in exponential family, the sufficient statistics for mean is the statistics with the parameter. 

Similarly, we can extend this finding to other statistics(ie. variance) by factorization.

The denotation of a distribution, $f(x | \theta)$ write in this form as the sample data X is known, while the $\theta$ is unknown. So the distribution is only available when the $\theta$ is known. 

\subsection{Factorization Theorem}

Let  $f(x | \theta)$ denote the joint pdf or pmf of a sample X. A statistic $T(X)$ is a sufficient statistic for $\theta$ if and only if there exist functions  

\begin{align*}
	f(x | \theta) &=g(T(x)| \theta) h(x)
\end{align*}

such that, for all sample points X and all parameter points  $\theta$,


To understand this theorem, we can see that in the exponential family distribution

\begin{align*}
	f(x | \theta) &= \phi exp( \theta - b(\theta) - c(y))
\end{align*}

So we can write the distribution in factorization form, one part involves $\theta$ and the other part does not.


The proof needs to use the definition of the sufficient statistics that the distribution based on sufficient statistics does not depend on $\theta$. So we will construct a conditional distribution, which the nominator is the joint distribution of the sample X, and denominator is the distribution of the sufficient statistics.

When we prove the theorem using the specific distribution (ie. normal distribution), we can easily write out the joint distribution and distribution of $T(x)$. But now we are using the general form of distribution.

So the proof needs to find the distribution of $T(x)$. We need to see if there is any relationship between $g(T(x)| \theta)$ and $P_{\theta}(T(x) = t)$. Need to pay attention that, the distribution of $T(X)$ is actually the sum of the distributions with any $y = T(X)$. We use the concept in the nuisance parameter, which need to find the distribution of statistics.


First, suppose $T(x)$ is sufficient statistics. 

As it is the general case, so we just choose 
$g(t| \theta) = P_{\theta}(T(x) = t)$.

Then $h(x) = P(X=x | T(X))$, Because $T(X)$ is sufficient, the conditional probability defining  
$h(x)$ dose not depend on $\theta$. Thus, the choice of $h(x)$ and $g(t|\theta)$ is legitimate.

For this choice,

\begin{align*}
	f(x | \theta) &= P_{\theta}(X = x) \\
	&=  P_{\theta}(X = x, T(X)= T(x)) \\
	&= P_{\theta}(X = x | T(X)= T(x)) P_{\theta}(T(X) = T(x)) \\
	&= g(T(x)| \theta) h(x)
\end{align*}

Now assume factorization exists. Let $q(t|\theta)$ be the pmf of $T(X)$. To show that $T(X)$ is sufficient we exam the ratio $f(x|\theta) / q(T(X)|\theta)$. Define $A_{T(x)} = y: T(y) = T(x)$. Then

\begin{align*}
	\frac{f(x | \theta)}{q(T(X)|\theta)} &= \frac{g(T(x)| \theta) h(x)}{q(T(X)|\theta)} \\
	&=  \frac{g(T(x)| \theta) h(x)}{\sum_{A_{T(x)} } g(T(y)| \theta) h(y)} \\
	&=  \frac{g(T(x)| \theta) h(x)}{g(T(x)| \theta) \sum_{A_{T(x)} }  h(y)} \\
	&= \frac{h(x)}{\sum_{A_{T(x)} }  h(y)}
\end{align*}

Since the ratio does not depend on $\theta$, by Theorem, $T(X)$ is a sufficient statistic for  
$\theta$. We factor the joint pdf into two parts, one part not depending on $\theta$, which is  
$h(x)$ function. The other part depends on $\theta$, depends on the sample x only through some function $T(x)$ and this function is a sufficient statistic for $\theta$.


\subsubsection{Uniform Sufficient Statistic}
Let $X_1, X_2,..., X_n$ be i.i.d. observations from the discrete uniform distribution on $1,..\theta$. The pmf of $X_i$ is 

\begin{align*}
	f(x|\theta) &= \begin{cases}
		\frac{1}{\theta} & x=1,2, .. \theta \\
		0 & \text{otherwise}
	\end{cases}
\end{align*}

Thus the joint pmf of $X_1, ...X_n$ is 

\begin{align*}
	f(x|\theta) &= \begin{cases}
		\frac{1}{\theta^n} & x_i \in \{1,2, .., \theta\} \theta \\
		0 & \text{otherwise}
	\end{cases}
\end{align*}

Let $N= \{1,2.. \}$ be the set of positive integers and let $N_{\theta} = \{1,2,.. \theta\}$. Then the joint pmf of $X_1, ... X_n$ is

\begin{align*}
	f(x|\theta) &= \prod_{i=1}^n \theta^{-1} I_{N_{\theta}} (x_i) = \theta^{-n} \prod_{i=1}^n I_{N_{\theta}} (x_i) 
\end{align*}

Defining $T(x) = max_i x_i$ then

\begin{align*}
	\prod_{i=1}^n I_{N_{\theta}} (x_i) = (\prod_{i=1}^n I_{N} (x_i)) I_{N_{\theta}} (T(x))
\end{align*}

The equation here is similar as the $I_{x_1 < x_2 <.. x_{max}} (x_m < \theta)$.

Thus, we have the factorization
\begin{align*}
	f(x|\theta) &= \theta^{-n} \prod_{i=1}^n I_{N_{\theta}} (T(x))  (\prod_{i=1}^n I_{N} (x_i))
\end{align*}

The uniform distribution statistics is difficult to understand, it is not explicit using the identity function. 

\subsubsection{Normal Sufficient Statisitc for Both Parameters}
Assume $X_1,..X_n$ are i.i.d. $N(\mu, \simga^2)$ with both parameters unknown. When using Theorem 5.1, any part of the joint pdf that depends on either must be include in the g function. Define  Then


\subsection{Minimum Sufficient Statistics}
Let $X \sim Gamma(\alpha, 1)$ and $Y \sim Gamma(\beta, 1)$ where the paramaterization is such that $\alpha$ is the shape parameter. Then 
\begin{align*}
	\frac{X}{X+Y} \sim Beta(\alpha, \beta)
\end{align*}

\begin{itemize}
	\item [(i)] How do we create the transformed variables $U= \frac{X}{X+Y} $, V, such that it is easy to get the distribution of U and V, and easy to integrate out V to get the margin distribution of U? This requires familiarity of Beta and Gamma distribution.
	
	One proposal is $V= Y$, and the other is $V= X+Y$. Compare between the two and see which one is better.
	
	\textbf{Note:} The way to choose new parameter is easier to get the original parameter by simple arithmetic calculation or just itself.
	If we are using the first set of new parameter, it would get the distribution very complicated, the Beta distribution has the form of $U, 1-U$, need to keep in mind of the achieving the product of two distribution form. 
	
	\item[(ii)] We could see that $V=X+Y$ is better as U and V are independent from each other.
\begin{align*}
	f(X) &= \frac{1}{\Gamma{(\alpha)}} x^{\alpha-1} e^{-x}\\
	f(Y) &= \frac{1}{\Gamma{(\beta)}} y^{\beta-1} e^{-y}
\end{align*}	
Let
\begin{align*}
	U &= \frac{X}{X+Y}, \qquad V = X + Y
\end{align*}	
Then
\begin{align*}
	X &= UV, \qquad Y = V - UV
\end{align*}
	 	
The Jacobian transformation matrix
\begin{align*}
	J &= \begin{pmatrix}
		\diffp{X}{U} & \diffp{X}{V} \\
		\diffp{Y}{U} & \diffp{Y}{V} 
	\end{pmatrix} =  \begin{pmatrix}
	V & U \\
	-V & 1-U
\end{pmatrix}\\
|J| &= V
\end{align*}
X and Y are independent, so the joint distribution of (X, Y) 
\begin{align*}
	f(X, Y) &= \frac{1}{\Gamma{(\alpha)} \Gamma{(\beta)}} x^{\alpha -1} e^{-x} y^{\beta -1} e^{-y}\\
	f(U, V) &= \frac{1}{\Gamma{(\alpha)} \Gamma{(\beta)}} U^{\alpha -1} (1-U)^{\beta -1} V^{\alpha + \beta -1} e^{-V}
\end{align*}

We don't always need to integrate out the other parameter, if we can write the distribution in the form that we can recognize, then will directly get the distribution. 

$V \sim Gamma (\alpha + \beta, 1)$ and $U \sim Beta(\alpha, \beta)$.

\end{itemize}

\subsection{Complete Statistics}

The completeness ensures that the distributions corresponding to different values of the parameters are distinct. It is closely related to the idea of identifiability (unique, distinct distribution).

\begin{definition}
Consider a random variable X whose probability distribution belongs to a parametric model $P_{\theta}$ parametrized by $\theta$. Say T is a statistic, that is, the composition of a measurable function with a random sample $X_1,.. X_n$. The statistic T is said to be complete for the distribution of X if, for every measurable function g; 
if $E_{\theta} (g(T)) = 0$ for all $\theta$ then $P_{\theta}(g(T) = 0) = 1$ for all $\theta$. 
\end{definition}

For some parametric families, a complete sufficient statistic does not exist. For example, if you take a sample sized $n > 2$ from a $N(\theta, \theta^2)$ distribution, then $\Big( \sum_{i=1}^n X_i, \sum_{i=1]^n X_i^2 \Big)$ is a minimal sufficient statistic and is a function of any other minimal sufficient statistic, but $2 \Big( \sum_{i=1]^n X_i^2 \Big) - (n+1) \sum_{i=1]^n X_i^2$ has an expectation of 0 for all $\theta$, so there can not be a complete statistic. 

\subsubsection{Importance of completeness}

Lehmann-Scheffe theorem

Completeness occurs in the Lehmann-Scheffe theorem, which states that if a statistic that is unbiased, complete and sufficient for some parameter $\theta$, then it is the best mean-unbiased estimator for $\theta$. In other words, this statistic has a smaller expected loss for any convex loss function, in many practical applications with the squared loss-function, it has a smaller mean squared error among an estimators with the same expected value. 

Basu's theorem


\backmatter

\import{./}{bibliography.tex}


\end{document}
